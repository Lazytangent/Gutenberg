\documentclass[oneside,12pt]{book}

\usepackage{mathtools,amsthm,amssymb,setspace}
\usepackage{fancyhdr,array,graphicx,multicol,titlesec}
\usepackage[margin=1in]{geometry}
\usepackage[utf8]{inputenc}
\usepackage[english,greek]{babel}
\usepackage[T1]{fontenc}

\onehalfspace 
\setlength{\parskip}{12pt}
%\setlength{\parindent}{0pt}

\titleformat{\chapter}[hang]{}{}{0pt}{\Huge\bfseries}[]

\begin{document}
    
\selectlanguage{english}
\frontmatter

\begin{titlepage}
    \centering
    {\Huge \textbf{GEOMETRICAL SOLUTIONS}\par} 
    {\Large \textbf{DERIVED FROM} \par}
    {\Huge \textbf{MECHANICS} \par}
    {\LARGE A TREATISE OF ARCHIMEDES \par}
    \vspace{0.25cm}
    {\large RECENTLY DISCOVERED\\AND TRANSLATED FROM THE GREEK BY\\DR. J. L. HEIBERG \par}
    {\normalsize PROFESSOR OF CLASSICAL PHILOSOPHY\\AT THE UNIVERSITY OF COPENHAGEN \par}
    \vspace{0.25cm}
    {\large WITH AN INTRODUCTION BY \par}
    {\Large DAVID EUGENE SMITH \par}
    {\normalsize PRESIDENT OF TEACHERS COLLEGE,\\COLUMBIA UNIVERSITY, NEW YORK \par}
    \vspace{0.15cm}
    {\large ENGLISH VERSION TRANSLATED FROM THE GERMAN\\BY LYDIA G. ROBINSON\\AND REPRINTED FROM ``THE MONIST,'' APRIL, 1909 \par} 
    
    {\large CHICAGO\\THE OPEN COURT PUBLISHING COMPANY\\LONDON AGENTS\\KEGAN PAUL, TRENCH, TR\"UBNER \& CO., LTD.\\1909 \par}
\end{titlepage}

\chapter{Introduction}
If there ever was a case of appropriateness in discovery, the finding of this manuscript in the summer of 1906 was one. In the first place it was appropriate that the discovery should be made in Constantinople, since it was here that the West received its first manuscripts of the other extant works, nine in number, of the great Syracusan. It was futhermore appropriate that the discovery should be made by Professor Heiberg, \textit{facilis princeps} among all workers in the field of editing the classics of Greek mathematics, and an indeftigable searcher of the libraries of Europe for manuscripts to aid him in perfecting his labors. And finally it was most appropriate that this work should appear at a time when the affiliation of pure and applied mathematics is becoming so generally recognized all over the world. We are sometimes led to feel, in considering osilated cases, that the great contributors of the past have worked in the field of pure mathematics alone, and the saying of Plutarch that Archimedes felt that ``every kind of art connected with daily needs was ignoble and vulgar''\footnote{Marcellus, 17} may have stengthened this feeling. It therefore assists us in properly orientating ourselves to read another treatise from the greatest mathematician of antiquity that sets clearly before us his indebtedness to the mechanical applications of his subject. \par 

Not the least interesting of the passages in the manuscript is the first line, the greeting to Eratosthenes. It is well known, on the testimony of Diodoros his countryman, that Archimedes studied in Alexandria, and the latter frequently makes mention of Konon of Samos whom he knew there, probably as a teacher, and to whom he was indebted for the suggestion of the spiral that bears his name. It is also related, this time by Proclos, that Eratosthenes was a contemporary of Archimedes, and if the testimony of so late a writer as Tzetzes, who lived in the twelfth century, may be taken as valid, the former was eleven years the junior of the great Sicilian. Until now, however, we have had nothing definite to show that the two were ever acquainted. The great Alexandrian savant,--poet, geographer, arithmetician,--affectionately called by the students Pentathlos, the champion in five sports,\footnote{His nickname of \textit{Beta} is well known, possibly because his lecture room was number 2.} selected by Ptolemy Euergetes to succeed his master, Kallimachos the poet, as head of the great Library,--this man, the most renowned of his time in Alexandria, could hardly have been a teacher of Archimedes, nor yet the fellow student of one who was so much his senior. It is more probable that they were friends in the later days when Archimedes was received as a savanta rather than as a learner, and this is borne out by the statement at the close of proposition I which refers to one of his earlier works, showing that this particular treatise was a late one. This reference being to one of the two works dedicated Dositheos of Kolonos,\footnote{We know little of his works, none of which are extant. Geminos and Ptolemy refer to certain observations made by him in 200 B.C., twelve years after the death of Archimedes. Pliny also mentions him.} and one of these (\textit{De lineis spiralibus}) referring to an earlier treatise sent to Konon,\footnote{\selectlanguage{greek}Τῶν ποπὶ Κόνωνα άπυσταλέντων θεωρημάτων.} we are led to believe that this was one of the latest works of Archimedes and that Eratosthenes was a friend of his mature years, although one of long standing. The statement that the preliminary propositions were sent ``some time ago'' bears out this idea of a considerable duration of friendship, and the idea that more or less correspondence had resulted from this communication may be inferred by the statement that he saw, as he had previously said, that Eratosthenes was ``a capable scholar and a prominent teacher of philosophy,'' and also that he understood ``how to value a mathematical method of investigation when the opportunity offered.'' We have, then, new light upon the relations between these two men, the leaders among the learned of their day. \par 

A second feature of much interest in the treatise is the intimate view that we have into the workings of the mind of the author. It must always be remembered that Archimedes was primarily a discoverer, and not primarily a compiler as were Euclid, Apollonios, and Nicomachos. Therefore to have him follow up his first communication of theorems to Eratosthenes by a statement of his mental processes in reaching his conclusions is not merely a contribution to mathematics but one to education as well. Particularly is this true in the following statement, which may well be kept in mind in the present day: ``I have thought it well to analyse and lay down for you in this same book a peculiar method by means of which it will be possible for you to derive instruction as to how certain mathematical questions may be investigated by means of mechanics. And I am convinced that this is equally profitable in demonstrating a proposition itself; for much that was made evident to me through the medium of mechanics was later proved by means of geometry, because the treatment by the former method had not yet been established by way of a demonstration. For of course it is easier to establish a proof if one has in this way previously obtained a conception of the questions, than for him to seek it without such a preliminary notion \dots. Indeed I assume that some one among the investigators of to-day or in the future will discover by the method here set forth still other propositions which have not yet occurred to us.'' Perhaps in all the history of mathematics no such prophetic truth was ever put into words. It would almost seem as if Archimedes must have seen as in a vision the methods of Galileo, Cavalieri, Pascal, Newton, and many of the other great makers of the mathematics of the Renaissance and the present time. \par 

The first proposition concerns the quadrature of the parabola, a subject treated at length in one of his earlier communications to Dositheos.\footnote{\selectlanguage{greek}Τετραγωνισμδς παραβολῆς.} He gives a digest of the treatment, but with the warning that the proof is not complete, as it is in his special work upon the project. He has, in fact, summarized propositions VII-XVII of his communication to Dositheos, omitting the geometric treatment of propositions XVIII-XXIV. One thing that he does not state, here or in any of his works, is where the idea of center of gravity\footnote{{\selectlanguage{greek}Κέντρα βαρῶν,} for ``barycentric'' is a very old term.} started. It was certainly a common notion in his day, for he often uses it without defining it. It appears in Euclid's\footnote{At any rate in the anonymous fragment \textit{De levi et ponderoso}, sometimes attributed to him.} time, but how much earlier we cannot as yet say. \par 

Proposition II states no new fact. Essentially it means that if a sphere, cylinder, and cone (always circular) have the same radius, $r$, and the altitude of the cone is $r$ and that of the cylinder $2r$, then the volumes will be as 4:1:6, which is true, since they are respectively $\frac{4}{3}\pi r^3,\ \frac{1}{3}\pi r^3,\ \text{and } 2\pi r^3$. The interesting thing, however, is the method pursued, the derivation of geometric truths from principles of mechanics. There is, too, in every sentence, a little suggestion of Cavalieri, an anticipation by nearly two thousand years of the work of the greatest immediate precursor of Newton. And the geometric imagination that Archimedes shows in the lasat sentence is also noteworthy as one of the interesting features of this work: ``After I had thus perceived that a sphere is fourtimes as large as the cone\dots it occurred to me that the surface of a sphere is four times as great as its largest circle, in which I proceeded from the idea that just as a circle is qual to a triangle whose base is the periphery of the circle, and whose altitude is equal to its radius, so a sphere is equal to a cone whose base is the same as the surface of the sphere and whose altitude is equal to the radius of the sphere.'' As a bit of generalization this throws a good deal of light on the workings of Archimedes's mind. \par 

In proposition III he considers the volume of a spheroid, which he had already treated more fully in one of his letters to Dositheos,\footnote{\selectlanguage{greek}Περὶ κωνοειδεῶν και σφαιροειδεῶν.} and which contains nothing new from a mathematical standpoint. Indeed it is the method rather than the conclusion that is interesting in such of the subsequent propositions as relate to mensuration. Proposition V deals with the center of gravity of a segment of a conoid, and propositions VI with the center of gravity of a hemisphere, thus carrying into solid geometry the work of Archimedes on the equilibrium of planes and on their centers of gravity.\footnote{\selectlanguage{greek}Έπιπέδων ὶσορροπιῶν ἧ κέντρα βαρῶν έπιπέδων.} The general method is that already known in the treatise mentioned, and this is followed through proposition X. \par 

Porosition XI is the interesting case of a segment of a right cylinder cut off by a plane through the center of the lower base and tangent to the upper one. He shows this to equal one-sixth of the square prism that circumscribes the cylinder. This is well known to us through the formula $v=2r^2h/3$, the volume of the prism being $4r^2h$, and requires a knowledge of the center of gravity of the cylindric section in question. Archimedes is, so far as we know, the first to state this result, and he obtains it by his usual method of the skilful balancing of sections. There are several lacunaue in the demonstration, but enough of it remains to show the ingenuity of the general plan. The culminating interest from the mathematical standpoint lies in proposition XIII, where Archimedes reduces the whole quetsion to that of the quadrature of the parabola. He shows that a fourth of the circumscribed prism is to the segment of the cylinder as the semi-base of the prism is to the parabola inscribed inthe semi-base; that is, that $\frac{1}{4}p\ :\ v=\frac{1}{2}b\ :\ (\frac{2}{3}\cdot \frac{1}{2}b)$, whence $v=\frac{1}{6}p$. Proposition XIV is incomplete, but it is the conclusion of the two preceding propositions. \par 

In general, therefore, the greatest value of the work lies in the following: \par 

1. It throws light upon the hitherto only syspected relations of Archimedes and Eratosthenes. \par 

2. It shows the working of the mind of Archimedes in the discovery of mathematical truths, showing that he often obtained his resuylts by intuition or even my measurement, rahter than by an analytic form of reasoning, verifying these results later by strict analysis. \par 

3. It expresses definitely the fact that Archimedes was the discoverer of these properties relating to the sphere and cylinder that have been attributed to him and that are given in his other works without a definite statement of their authorship. \par 

4. It shows that Archimedes was the first to state the volume of the cylinder segment mentioned, and it gives an interesting description of the mechanical method by which he arrived at his result. \par 

\begin{flushright}
    \textsc{David Eugene Smith.\\Teachers College, Columbia University.}
\end{flushright}

\mainmatter
\chapter{Geometrical Solutions Derived from Mechanics}
Archimedes to Eratosthenes, Greeting: \par 

Some time ago I sent you some theorems I had discovered, writing down only the propositions because I wished you to find their demonstrations which had not been given. The propositions of the theorems which I sent you were the following: \par 

1. If in a perpendicular prism with a parallelogram\footnote{This must mean a square.} for base a cylinder is inscribed which has its bases in the opposite parallelograms\footnotemark[1] and its surface touching the other planes of the prism, and if a plane is passed through the center of the circle that is the base of the cylinder and one side of the square lying in the opposite plane, then that plane will cut off from the cylinder a section whichis bounded by two planes, the intersecting plane and the one in which the base of the cylinder lies, and also by as much of the surface of the cylinder as lies between these same planes; and the detached section of the cylinder is $\frac{1}{6}$ of the whole prism. \par 

2. If in a cube a cylinder is inscribed whose bases lie in opposite parallelograms\footnotemark[1] and whose surface touches the other four planes, and if in the same cube a second sylinder is inscribed whose bases lie in two other parallelograms\footnotemark[1] and whose surface touches the four other planes, then the body enclosed by the surface of the cylinder and comprehended within both cylinders will be equal to $\frac{2}{3}$ of the whole cube. \par 

These propositions differ essentially from those formerly discovered; for then we compared those bodies (conoids. spheroids and their segments) with the volume of cones and cylinders but none of them was found to be equal to a body enclosed by planes. Each of these bodies, on the other hand, which are enclosed by two planes and cylindrical surfaces is found to be equal to a body enclosed by planes. The demonstration of these propositions I am accordingly sending to you in this book. \par 

\textbf{10.} 
\end{document}