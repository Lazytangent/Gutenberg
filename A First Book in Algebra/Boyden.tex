\documentclass[12pt,oneside]{book}

\usepackage{mathtools,amssymb,fancyhdr}
\usepackage{fullpage,amsthm,graphicx,multicol}
\usepackage{setspace}

\onehalfspace 
\setlength{\parskip}{12pt plus 1pt minus 2pt}
%\setlength{\parindent}{0pt}

\begin{document}
    
\frontmatter
\noindent The Project Gutenberg EBook of A First Book in Algebra, by Wallace C. Boyden

\noindent This eBook is for the use of anyone anywhere at no cost and with almost no restrictions whatsoever.
You may copy it, give it away or re-use it under the terms of the Project Gutenberg License uncluded 
with this eBook or online at www.gutenberg.net

\begin{titlepage}
    \centering
    \vspace*{4cm}
    {\Huge A FIRST BOOK IN ALGEBRA\par}
    \vspace{2cm}
    {\large BY\par}
    \vspace{2cm}
    {\LARGE WALLACE C. BOYDEN, A.M.\par}
    \vspace{2cm}
    {\Large SUB-MASTER OF THE BOSTON NORMAL SCHOOL\par}
    \vspace{1cm}
    {\normalsize 1895}
\end{titlepage}

\chapter{Preface}
In preparing this book, the author had especially in mind classes in the upper grades of grammar schools,
though the work will be found equally well adapted to the needs of any classes of beginners. \par 

The ideas which have guided in the treatment of the subject are the following:
The study of algebra is a continuation of what the pupil has been doing for years,
but it is expected that this new work will result in a knowledge of \textit{general truths}
about numbers, and an increased power of clear thinking.
All the differences between this work and that pursued in arithmetic may be traced to the introduction
of two new elements, namely, negative numbers and the representation of numbers by letters.
The solution of problems is one of the most valuable portions of the work, 
in that it serves to develop the thought-power of the pupil at the same time that it broadens 
his knowledge of nubmers and their relations. Powers are developed and habits formed only by persistent, 
long-continued practice. \par 

Accordingly, in this book, it is takn for granted that the pupil knows what he may be reasonably expected
to have learned from his study of arithmetic; abundant practice is given in the representation of numbers by letters,
and great care is taken to make clear the meaning of the minus sign as applied to a single number,
together with the modes of operating upon negative numbers; problems are given in every exercise in the book;
and, instead of making a statement of what the child is to see in the illustrative example, questions are asked 
which shall lead him to find for himself that which he is to learn from the example. \par 

BOSTON, MASS., December, 1893. \par

\tableofcontents

\mainmatter

\chapter{Algebraic Notation}
Algebra is so much like arithmetic that all that you know about addition, subtraction, multiplication, and 
division, the signs that you have been using and the ways of working out problems, will be very useful
to you in this study. There are two things the introduction of which really makes all the differences between
arithmetic and algebra. One of these is the use of \textit{letters to represent numbers}, and you will see in 
following exercises that his change makes the solution of problems much easier. \par 

\section{Exercises}
{\large \textbf{Exercise I.}} \par 
\textit{Illustrative Example}. The sum of two numbers is 60, and the greater is four times the less.
What are the numbers? \par

\begin{center}
    \textbf{Solution} \par 
        \begin{align*}
        &\text{Let} & x&=\text{ the less number;}\\
        &\text{then} & 4x&=\text{ the greater number,}\\
        &\text{and} & 4x+x&=60,\\
        &\text{or} & 5x&=60;\\
        &\text{therefore} & x&=12,\\
        &\text{and} & 4x&=48. \text{ The numbers are 12 and 48.}\\
        \end{align*}
\end{center}

\begin{enumerate}
    \item The greater of two numbers is twice the less, and the sum of the numbers is 129. What are the numbers?
    \item A man bought a horse and carriage for \$500, paying three times as much for carriage as for the horse.
    How much did each cost?
    \item Two brothers, counting their money, found that together they had \$186, and that John had five times as
    much as Charles. How much had each? 
    \item Divide the number 64 into two parts so that one part shall be seven times the other. 
    \item A man walked 24 miles in a day. If he walked twice as far in the forenoon as in the afternoon, how far did
    he walk in the afternoon? 
\end{enumerate}

\end{document}