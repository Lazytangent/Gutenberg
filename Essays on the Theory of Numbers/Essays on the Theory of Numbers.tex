\documentclass[oneside,12pt]{book}

\usepackage{mathtools,amsthm,amssymb,setspace}
\usepackage{fancyhdr,fullpage,array,graphicx,multicol,enumerate}

\onehalfspace 
\setlength{\parskip}{6pt}
%\setlength{\parindent}{0pt}

\newcommand{\iit}[1]{\textit{#1}}

\begin{document}
    
\frontmatter

{\large Transcriber's Note:} The symbol \emph{3} is used as an approximation to the author's Part-of symbol, not to be confused with the digit 3. Internal page references have been
adjusted to fit the pagination of this edition. A few typographical errors have been corrected - these ar noted at the very end of the text. 

\begin{titlepage}
    \uppercase{
    \centering
    {\Huge ESSAYS \par}
    {\large ON THE \par}
    {\Huge THEORY OF NUMBERS \par}
    \vspace{1cm}
    {\Large I. continuity and Irrational Numbers\\II. The Nature and Meaning of Numbers \par}
    \vspace{1cm}
    {\large by \par}
    {\Large Richard Dedekind \par}
    \vspace{0.5cm}
    {\large authorised translation by \par}
    {\Large Wooster Woodruff Beman \par}
    {\large Professor of Mathematics in the University of Michigan \par}
    \vfill
    {\Large Chicago \par}
    {\large The Open Court Publishing Company \par}
    {\normalsize London Agents \par}
    {\small Kegan Paul, Trench, Tr\"ubner \& Co., Ltd.\\1901 \par}
    }
\end{titlepage}

\mainmatter

\chapter{Continuity and Irrational Numbers}
My attention was first directed toward the considerations which form the subject of this pamphlet in the autumn of 1858. As professor in the Polytechnic School in Z\"urich
I found myself for the first time obliged to lecture upon the elements of the differential calculus and felt more keenly thean ever before the lack of a really scientific 
foundation for arithmetic. In discussing the notion of the approach of a variable magnitude to a fixed limiting value, and especially in proving the theorem that every
magnitude which grows continually, but not beyond all limits, must certainly approach a limiting value, I had recourese to geometric evidences. Even now such resort to 
geometric intuition in a first presentation of the differential calculus, I regard as exceedingly useful, from the didactic standopint, and indeed indispensable, if one 
does not wish to lose too much time. But that his form of introduction into the differential calculus can make no claim to being scientific, no one will deny. For myself 
this feeling of dissatisfaction was so overpowering that I made the fixed resolve to keep meditating on the question till I should find a purely arithmetic and perfectly rigorous 
foundation for the principles of infinitesimal analysis. The statement is so ferquently made that the differential calculus deals with coninuous magnitude, and yet
an explanation of this continuity is nowhere given; even the most rigorous expositions of the differential calculus do not base their proofs uopn continuity but, with more or less
consciousness of the fact, they either appeal to geometric notions or those suggested by geometry, or depend upon theorems which are never established in a purely arithmetic manner.
Among these, for example, belongs the above-mentioned theorem, and a more careful investigation convinced me that this theorem, or any one equivalent to it, can be regarded in 
some way as a sufficient basis for infinitesimal analysis. It thenonly remained to discover its true origin in the elements of arithmetic and thus at the same time to secure a real 
definition of the essence of continuity. I succeeded Nov. 24, 1858, and a few days afterward I communicated the results of my meditations to my dear friend Dur\'ege with whom I had a long
and lively discussion. Later I explained these views of a scientific basis of arithmetic to a few of my pupils, and here in Braunschweig read a paper upon the subject before the 
scientific club of professors, but I could not make up my mind to its publication, because, in the first place, the presentation did not seem altogether simple, and further, the 
theory itself had little promise. Nevertheless I had already half determined to select this theme as subject for this occasion, when a few days ago, March 14, by the kindness
of the author, the paper \textit{Die Elemente der Funktionenlehre} by E. Heine (\textit{Crelle's Journal}, Vol. 74) came into my hands and cofirmed me in my decision. In the main
I fully agree with the substance of this memoir, and indeed I could hardly do otherwise, but I will frankly acknowledge that my own presentation seems to me to be simpler in form
and to bring out the vital point more clearly. While writing this preface (March 20, 1872), I am just in receipt of the interesting paper \textit{Ueber die Ausdehnung eines Satzes aus 
der Theorie der trignometrischen Reihen}, by G. Cantor (\textit{Math. Annalen}, Vol. 5), for which I owe the ingenious author my hearty thanks. As I find on a hasty perusal, the
axiom given in Section II. of that paper, aside from the form of presentation, agress with what I designate in Section III. as the essence of continuity. But what 
advantage will be gained by even a purely abstract definition of real numbers of a higher type, I am as yet unable to see, conceiving as I do of the domain of real numbers
as complete in itself. \par 

\section{Propertis of Rational Numbers}
\label{S1}
The development of the arithmetic of rational numbers is here presupposed, but still I think it worth while to call attention to certain important matters without discussion, 
so as to show at the outset the standopint assumed in what follows. I regard the whole of arithmetic as a necessary, or at least natural, consequence of the simplest arithmetic act,
that of counting, and counting itself as nothing else than the successive creation of the infinite series of positive integers in which each individual is defined by the one immediately prededing;
the simplest act is the passing from an already-formed individual to the consecutive new one to be formed. The chain of these numbers forms in itself an exceedingly useful instrument 
for the human mind; it presents an inexhaustible wealth of remarkable laws obtained by the introduction of the four fundamental operations of arithmetic. Addition is the combination of 
any arbitrary repetitions of the above-mentioned simplest act into a single act; from it in a similar way arises multiplication. While the performance of these two operations 
is always possible, that of the inverse operstions, subtraction and division, proves to be limited. Whatever the immediate occasion may have been, whatever comparisons or analogies 
with experience, or intuition, may have led thereto; it is certainly true that just this limitation in performing the indirect operations has in each case been the real motive for a new 
creative act; thus negative and fractional numbers have been created by the human mind; and in the system of all rational numbers there ahs been gained an instrument of infinitely greater
perfection. This system, which I shall denote by $R$, possesses first of all a completeness and self-containedness which I have designated in another place\footnote{\textit{Vorlesungen \"uber Zahlentheorie}, by P. G. Lejeune Dirichlet. 2d ed. \S 159} as characteristic of a \textit{body of numbers}
[Zahlk\"orper] and which consists in this that the four fundamental operations are always performable with any two indviduals in $R$, i.e., the result is always an individual of $R$,
the single case of division by the number zero being excepted. \par 

For our immediate purpose, however, another property of the system $R$ is still more important; it may be expressed by saying that the system $R$ forms a well-arranged domain of one
dimension extending to infinity on two opposite sides. What is meant by this is sufficiently indicated by my use of expressions borrowed from geometric ideas; but just for this reason
it will be necessary to bring out clearly the corresponding purely arithmetic properties in order to avoid even the appearance as if arithmetic were in need of ideas foreign
to it. \par 

To express that the symbols $a$ and $b$ represent one and the same rational number we put $a=b$ as well as $b=a$. The fact that two rational numbers $a$, $b$ are different appears 
in this that the difference $a-b$ has either a positive or negative value. In the former case $a$ is said to be \textit{greater} than $b$, $b$ \textit{less} than $a$; this is also
indicated by the symbols $a>b,\ b<a$.\footnote{Hence in what follows the so-called ``algebraic'' greater and less are understoof unless the word ``absolute'' is added.} As in the latter case $b-a$ has a positive value it follows that $b>a,\ a<b$. In regard to these two ways in which two numbes may differ the 
following laws will hold: \par 

\begin{enumerate}[I.]
    \item If $a>b$, and $b>c$. then $a>c$. Whenever $a$, $c$ are two different (or unequal) numbers, and $b$ is greater than the one and less than the other, 
    we shall, without hesitation because of the suggestion of geometric ideas, express this briefly by saying: $b$ lies between two numbers $a$, $c$.
    \item If $a$, $c$ are two different numbers, there are infinitely many different numbers lying between $a$, $c$.
    \item If $a$ is any definite number, then all numbers of the system $R$ fall into two classes, $A_1$ and $A_2$, each of which contains infinitely many individuals; the first class $A_1$ comprises all numbers $a_1$ that are $<a$, the second class $A_2$ comprises all numbers $a_2$ that are $>a$; the number $a$ itself may be assigned at pleasure to the first or second class, being respectively the greatest number of the first class or the least of the second. In every case the separation of the system $R$ into the two classes $A_1$, $A_2$ is such that every number of the first class $A_1$ is less than every number of the second class $A_2$.
\end{enumerate}

\section{Comparison of the Rational Numbers\\with the Points of a Straight Line}
\label{S2}
The above-mentioned properties of rational numbers recall the corresponding relations of position of the points of a straight line $L$. If the two opposite directions existing upon it are distinguished by ``right'' and ``left,'' and $p$, $q$ are two different points, then either $p$ lies to the right of $q$, and at the same time $q$ to the left of $p$, or conversely $q$ lies to the right of $p$ and at the same time $p$ to the left of $q$. A third case is impossible, if $p$, $q$ are actually different points. In regard to this difference in position the following laws hold: \par 

\begin{enumerate}[I.]
    \item If $p$ lies to the right of $q$, and $q$ to the right of $r$, then $p$ lies to the right of $r$; and we say that $q$ lies between the points $p$ and $r$.
    \item If $p$, $r$ are two different opints, the there always exist infinitely many points that lie between $p$ and $r$.
    \item If $p$ is a definite point in $L$, then all points in $L$ fall into two classes, $P_1$, $P_2$, each of which contains infinitely many individuals; the first class $P_1$ contains all the points $p_1$, that lie to the left of $P$, and the second class $P_2$ contains all the points $p_2$ that lie to the right of $p$; the point $p$ itself may be assigned at pleasure to the first or second class. In every case the separation of the straight line $L$ into the two classes or portions $P_1$, $P_2$, is of such a character that every point of the first class $P_1$ lies to the left of every point of the second class $P_2$.
\end{enumerate}

This analogy between rational numbers and the points of a straight line, as is well known, becomes a real correspondence when we select upon the straight line a definite origin or zero-point $o$ and a definite unit of length for the measurement of segments. With the aid of the latter to every rational number $a$ a corresponding length can be constructed and if we lay this off upon the straight line to the right or left of $o$ according as $a$ is positive or negative, we obtain a definite end-point $p$, which may be regarded as the point corresponding to the number $a$; to the rational number zero corresponds the point $o$. In this way to every rational number $a$, i.e., to every indivudla in $R$, corresponds one and only one point $p$, i.e., an individual in $L$. To the two numbers $a$, $b$ respectively correspond the two points, $p,\ q$, and if $a>b$, then $p$ lies to the right of $q$. To the laws \textsc{i, ii, iii} of the previous Section correspond completely the laws \textsc{i, ii, iii} of the present. \par 

\section{Continuity of the Straight Line}
Of the greatest importance, however, is the fact that in a straight line $L$ there are infinitely many points which correspond to no rational number. If the point $p$ corresponds to the rational number $a$, then, as is well known, the length $op$ is commensurable with the invariable unit of measure used in the construction, i.e., there exists a third length, a so-called common measure, of which these two lengths are integral multiples. But the ancient Greeks already knew and had demonstrated that there are lengths incommensurable with a given unit of length, e.g., the diagonal of the square whose side is the unit of length. If we lay off such a length from the point $o$ upon the line we obtain an end-point which corresponds to no rational number. Since further it can be easily shown that there are infinitely many lengths which are incommensurable with the unit of length, we may affirm: The straight line $L$ is infinitely richer in point-individuals than the domain $R$ of rational numbers in number-individuals.\par 

If now, as is our desire, we try to follow up arithmetically all phenomena in the straight line, the domain of rational numbers is insufficient and it becomes absolutely necessary that the instrument $R$ constructed by the creation of the rational numbers be essentially improved by the creation of new numbers such that the domain of numbers shall gain the same completeness, or as we may say at once, the same \textit{continuity}, as the straight line. \par 

The previous considerations are so familiar and well known to all that many will regard their repetition quite superfluous. Still I regarded this recapitulation as necessary to prepare properly for the main question. For, the way in which the irrational numbers are usually introduced is based directly upon the conception of extensive magnitudes--which itself is nowhere carefully defined--and explains number as the result of measuring such a magnitude by another of the same kind.\footnote{The apparent advantage of the generality of this definition of number disappears as soon as we consider complex numbers. According to my view, on the other hand, the notion of the ratio between two numbers of the same kind can be clearly developed only after the introduction of irrational numbers.} Instead of this I demand that arithmetic shall be developed out of itself. \par 

That such comparisons with non-arithmetic notions have furnished the immediate occasion for the extension of the number-concept may, in a general way, be granted (though this was certainly not the case in the introduction of complex numbers); but this surely is no sufficient ground for introducing these foreign notions into arithmetic, the science of numbers. Just as negative and fractional rational numbers are formed by a new creation, and as the laws of operating with these numbers must and can be reduced to the laws of operating with positive integers, so we must endeavor completely to define irrational numbers by means of the rational numbers alone. The question only remains how to do this. \par 

The above comparison of the domain $R$ of rational numbers with a straight line has led to the recognition of the existence of gaps, of a certain incompleteness or discontinuity of the former, while we ascribe to the straight line completeness, absence of gaps, or continuity. In what then does this continuity consist? Everything must depend on the answer to this question, and only through it shall we obtain a scientific basis for the investigation of \textit{all} continuous domains. By vague remarks upon the unbroken connection in the smallest parts obviously nothing is gained; the problem is to indicate a precise characteristic of continuity that can serve as the basis for valid deductions. For a long time I pondered over this in vain, but finally I found what I was seeking. This discovery will, perhaps, be differently estimated by different people; the majority may find its substance very commonplace. It consists of the following. In the preceding section attention was called to the fact that every point $p$ of the straight line produces a separation of the same into two portions such that every point  of one portion lies to the left of every point of the other. I find the essence of continuity in the converse, i.e., in the following principle: \par

\begin{quote}
    ``If all points of the straight line fall into two classes such that every point of the first class lies to the left of every point of the second class, then there exists one and only one point which produces this division of all points into two classes, this severing of the straight line into two portions.''
\end{quote}

As already said I think I shall not err in assuming that every one will at once grant the truth of this statement; the majority of my readers will be very much disappointed in learning that by this commonplace remark the secret of continuity is to be revealed. To this I may say that I am glad if every one finds the above principle so obvious and so in harmony with his own ideas of a line; for I am utterly unable to adduce any proof of its correctness, nor has any one the power. The assumption of this property of the line is nothing else than an axiom by which we attribute to the line its continuity, by which we find continuity in the line. If space has at all a real existence it is \textit{not} necessary for it to be continuous; many of its properties would remain the same even were it discontinuous. And if we knew for certain that space was discontinuous there would be nothing to prevent us, in case we so desired, from filling up its gaps, in thought, and thus making it continuous; this filling up would consist in a creation of new point-individuals and would have to be effected in accordance with the above principle. \par 

\section{Creation of Irrational Numbers}
From the last remarks it is sufficiently obvious how the discontinuous domain $R$ of rational numbers may be rendered complete so as to form a continuous domain. In Section I it was pointed out that every rational number $a$ effects a separation of the system $R$ into two classes such that every number $a_1$ of the first class $A_1$ is less than every number $a_2$ of the second class $A_2$; the number $a$ is either the greatest number of the class $A_1$ or the least number of the class $A_2$. If now any separation of the system $R$ into two classes $A_1,\ A_2$ is given which possesses only \textit{this} characteristic property that every number $a_1$ in $A_1$ is less than every number $a_2$ in $A_2$, the nfor brevity we shall call such a separation a \textit{cut} [Schnitt] and designeate it by ($A_1,\ A_2$). We can then say that every rational number $a$ produces one cut or, strictly speaking, two cuts, which, however, we shall not look upon as essentially different; this cut possesses, \textit{besides}, the property that either among the numbers of the first class there exists a greatest or among the numbers of the second class a least number. And conversely, if a cut possesses this property, then it is produced by this greatest or least rational number. \par 

But it is easy to show that there exist infinitely many cuts not produced by rational numbers. The following example suggests itself most readily. \par 

Let $D$ be a positive integer but not the square of an integer, then there exists a positve integer $\lambda$ such that 
$$\lambda^2<D<(\lambda+1)^2.$$ \par 

If we assign to the second class $A_2$, every positive rational number $a_2$ whose square is $>D$, to the first class $A_1$ all other rational numbers $a_1$, this separation forms a cut ($A_1,\ A_2$), i.e., every number $a_1$ is less than every number $a_2$. For if $a_1=0$, or is negative, then on that ground $a_1$ is less than any number $a_2$, because, by definition, this last is positive; if $a_1$ is positive, then is its square $\leqq D$, and hence $a_1$ is less than any positive number $a_2$ whose square is $>D$. \par 

But this cut is produced by no rational number. To demonstrate this it must be shown first of all that there exists no rational number whose square $=D$. Although this kis known from the first elements of the theory of numbers, still the following indirect proof may find place here. If there exist a rational number whose square $=D$, then there exist two positive integers $t,\ u$, that satisfy the equation 
$$t^2-Du^2=0,$$
and we may assume that $u$ is the \textit{least} positive integer possessing the property that its square, by multiplication by $D$, may be converted into the square of an integer $t$. Since evidently
$$\lambda u<t<(\lambda +1)u,$$
the number $u'=t-\lambda u$ is a positive integer certainly \textit{less} than $u$. If further we put 
$$t'=Du-\lambda t,$$
$t'$ is likewise a positive integer, and we have
$$t'^2-Du'^2=(\lambda -D)(t^2-Du^2)=0,$$
which is contrary to the assumption respecting $u$. \par 

Hence the square of every rational number $x$ is either $<D$ or $>D$. From this it easily follows that there is neither in the class $A_1$ a greatest, nor in the class $A_2$ a least number. For if we put 
$$y=\frac{x(x^2+3D)}{3x^2+D},$$
we have
$$y-x=\frac{2x(D-x^2)}{3x^2+D}$$
and 
$$y^2-D=\frac{(x^2-D)^3}{(3x^2+D)^2}.$$
If in this we assume $x$ to be a positive number from the class $A_1$, then $x^2<D$, and hence $y>x$ and $y^2<D$. Therefore $y$ likewise belongs to the class $A_1$. But if we assume $x$ to be a number from the class $A_2$, then $x^2>D$, and hence $y<x,\ y>0,\ \text{and } y^2>D$. Therefore $y$ likewise belongs to the class $A_2$. This cut is therefore produced by no rational number. \par 

In this property that not all cuts are produced by rational numbers consists the incompleteness or discontinuity of the domain $R$ of all rational numbers. \par 

Whenever, then, we have to do with a cut $(A_1,A_2)$ produced by no rational number, we create a new, an \textit{irrational} number $\alpha$, which we regard as completely defined by this cut $(A_1,A_2)$; we shall say that the number $\alpha$ corresponds to this cut, or that it produces this cut. From now one, therefore, to every definite cut there corresponds a definite rational or irrational number, and we regard two numbers as \textit{different} or \textit{unequal} always and onlye when they correspond to essentially different cuts. \par 

In order to obtain a basis for the orderly arrangement of all \textit{real}, i.e., of all rational and irrational numbers we must investigate the relation between any two cuts $(A_1,A_2)$ and $(B_1,B_2)$ produced by any two numbers $\alpha$ and $\beta$. Obviously a cut $(A_1,A_2)$ is given compoletely when one of the two classes, e.g., the first $A_1$ is known, because the second $A_2$ consists of all rational numbers not contained in $A_1$, and the characteristic property of such a first class lies in this that if the number $a_1$ is contained in it, it also contains all numbers less than $a_1$. If now we compare two such first classes $A_1,\ B_1$ with each other, it may happen \par 

1. That they are perfectly identical, i.e., that every number contained in $A_1$ is also contained in $B_1$, and that every number contained in $B_1$ is also contained in $A_1$. In this case $A_2$ is necessarily identical with $B_2$, and the two cuts are perfectly identical, which we denote in symbols by $\alpha = \beta \text{or} \beta = \alpha$. \par 

But if the two classes $A_1,\ B_1$ are not identical, then there exists in the one, e.g., in $A_1$, a number $a'_1=b'_2$ not contained in the other $B_1$ and consequently found in $B_2$; hence all numbers $b_1$ contained in $B_1$ are certainly less than this number $a'_1=b'_2$ and therefore all numbers $b_1$ are contained in $A_1$. \par 

2. If now this number $a'_1$ is the only one in $A_1$ that is not contained in $B_1$, then is every other number $a_1$ contained in $A_1$ also contained in $B_1$ and is consequently $<a'_1$, i.e., $a'_1$ is the greatest among all the numbers $a_1$, hence the cut $(A_1,A_2)$ is produced by the rational number $a=a'_1=b'_2$. Concerning the other cut $(B_1,B_2)$ we know already that all numbers $b_1$ in $B_1$ are also contained in $A_1$ and are less than the number $a'_1=b'_2$ which is contained in $B_2$; every other number $b_2$ contained in $B_2$ must, however, be greater than $b'_2$, for otherwise it would be less than $a'_1$, therefore contained in $A_1$ and hence in $B_1$; hence $b'_2$ is the least among all numbers contained in $B_2$, and consequently the cut $(B_1,B_2)$ is produced by the same rational number $\beta =b'_2=a'_1=\alpha$. The two cuts are then only unessentially different. \par 

3. If however, there exist in $A_1$ at least two differnt numbers $a'_1=b'_2$ and $a''_1=b''_2$, which are not contained in $B_1$, then there exist infinitely many of them, because all the infinitely many numbers lying between $a'_1$ and $a''_1$ are obviously contained in $A_1$ (Sections \ref{S1} and \ref{S2}) but not in $B_1$. In this case we say that the numbers $\alpha \text{ and } \beta$ corresponding to these two essentially different cuts $(A_1,A_2)$ and $(B_1,B_2)$ are \textit{different}, and further that $\alpha$ is \textit{greater} than $\beta$, that $\beta$ is \textit{less} than $\alpha$, which we express in symbols by $\alpha > \beta$ as well as $\beta < \alpha$. It is to be noticed that this definition coincides completely with the one given earlier, when $\alpha,\ \beta$ are rational. \par 

The remaining possible cases are these: \par 

4. If there exists in $B_1$ one and only one number $b'_1=a'_2$, that is not contained in $A_1$ then the two cuts $(A_1,A_2)$ and $(B_1,B_2)$ are only unessentially different and they are produced by one and the same rational number $\alpha = a'_2=b'_1= \beta$. \par 

5. But if there are in $B_1$ at least two numbers which are not contained in $A_1$, then $\beta > \alpha,\ \alpha < \beta$. \par 

As this exhausts the possible cases, it follows that of two different numbers one is necessarily the greater, the other the less, which gives two possibilities. A third case is impossible. This was indeed involved in the use of the \textit{comparative} (greater, less) to designate the relation between $\alpha,\ \beta$; but this use has only now been justified. In just such investigations one needs to exercise the greatest care so that even with the best intention to be honest he shall not, through a hasty choice of expressions borrowed from other notions already developed, allow himself to be led into the use of inadmissable transfers from one domain to the other. \par 

If now we consider again somewhat carefullyt the case $\alpha > \beta$ it is obvious that the less number $\beta$, if rational, certainly belongs to the class $A_1$; for since there is in $A_1$ a number $a'_1=b'_2$ which belongs to the class $B_2$, it follows that the number $\beta$, whether the greatest number in $B_1$ or thet least in $B_2$ is certainly $\leqq a'_1$ and hence contained in $A_1$. Likewise it is obvious from $\alpha > \beta$ that the greater number $\alpha$, if rational, certainly belongs to the class $B_2$, because $\alpha \geqq a'_1$. Combining these two considerations we get hte following result: If a cut is produced by the number $\alpha$ then any rational number belongs to the class $A_1$ or to the class $A_2$ according to it is less or greater than $\alpha$; if the number $\alpha$ is itself rational it may belong to either class. \par 

From this we obtain finally the following: If $\alpha > \beta$, i.e., if there are infinitely many numbers in $A_1$ not contained in $B_1$ then there are infinitely many such numbers that at the same time are differnt from $\alpha$ and from $\beta$; every such rational number $c$ is $< \alpha$, because it is cvontained in $A_1$ and at the same time it is $> \beta$ because contained in $B_2$. \par 

\section{Continuity of the Domain of Real Numbers}
In consequence of the distinctions just established the system $\mathfrak{R}$ of all real numbers forms a well-arranged domain of one dimension; this is to mean merely that the following laws prevail: \par 

\textsc{i.} If $\alpha > \beta,\ \text{and } \beta > \gamma,\ \text{then is also } \alpha > \gamma$. We shall say that the number $\beta$ lies between $\alpha$ and $\gamma$. \par

\textsc{ii.} If $\alpha,\ \gamma$ are any two different numbers, then there exist infinitely many different numbers $\beta$ lying between $\alpha,\ \gamma$. \par 

\textsc{iii.} If $\alpha$ is any definite number then all numbers of the system $\Re$ fall into two classes $\mathfrak{A}_1 \text{ and } \mathfrak{A}_2$ each of which contains infinitely many individuals; the first class $\mathfrak{A}_1$ comprises all the numbers $\alpha_1$ that are less than $\alpha$, the second $\mathfrak{A}_2$ comprises all the numbers $\alpha_2$ that are greater than $\alpha$; the number $\alpha$ itself maybe assigned at pleasure to the first class or to the second, and it is respectively the greatest of the first or the least of the second class. In each case the separation of the system $\Re$ into the two classes $\mathfrak{A}_1,\mathfrak{A}_2$ is such that every number of first class $\mathfrak{A}_1$ is smaller than every number of the second class $\mathfrak{A}_2$ and we say that this separation is produced by the number $\alpha$. \par 

For brevity and in order not to weary the reader I suppress the proofs of these theorems which follow immediately from the definitions of the previous section. \par 

Beside these properties, however, the domain $\mathfrak{R}$ possesses also \iit{continuity}; i.e., the following theorem is true: \par 

\textsc{iv.} If the system $\mathfrak{R}$ of all real numbers breaks up into two classes $\mathfrak{A}_1,\mathfrak{A}_2$ such that every number $\alpha_1$ of the class  $\mathfrak{A}_1$ is less than every number $\alpha_2$ of the class $\mathfrak{A}_2$ then there exists one and only one number $\alpha$ by which this separation is produced. \par 

\iit{Proof.} By the separation or the cut of $\Re$ into $\mathfrak{A}_1$ and $\mathfrak{A}_2$ we obtain at the same time a cut $(A_1,A_2)$ of the system $R$ of all rational numbers which is defined by this that $A_1$ contains all rational numbers of the class $\mathfrak{A}_1$ and $A_2$ all other rational numbers, i.e., all rational numbers of the class $\mathfrak{A}_2$. Let $\alpha$ be the perfectly definite number which produces this cut $(A_1,A_2)$. If $\beta$ is any number different from $\alpha$, there are always infinitely many rational numbers $c$ lying between $\alpha$ and $\beta$. If $\beta<\alpha$, then $c<\alpha$; hence $c$ belongs to the class $A_1$ and consequently also to the class $\mathfrak{A}_1$, and since at the same time $\beta<c$ then$\beta$ also belongs to the same class $\mathfrak{A}_1$, because every number in $\mathfrak{A}_2$ is greater than every number $c$ in $\mathfrak{A}_1$. But if $\beta>\alpha$, then $c>\alpha$; hence $c$ belongs to the class $A_2$ and consequently also to the class $\mathfrak{A}_2$, and since at the same time $\beta>c$, then $\beta$ also belongs to the same class $\mathfrak{A}_2$, because every number in $\mathfrak{A}_1$ is less than every number $c$ in $\mathfrak{A}_2$. Hence every number $\beta$ different from $\alpha$ belongs to the class $\mathfrak{A}_1$ or to the class $\mathfrak{A}_2$ according as $\beta<\alpha$ or $\beta>\alpha$; consequently $\alpha$ itself is either the greatest number in $\mathfrak{A}_1$ or the least number in $\mathfrak{A}_2$, i.e., $\alpha$ is one and obviously the only number by which the separation of $R$ into the classes $\mathfrak{A}_1$, $\mathfrak{A}_2$ is produced. Which was to be proved. \par 

\section{Operations with Real Numbers}
To reduce any operation with two real numbers $\alpha,\ \beta$ to operations with rational numbers, it is only necessary from the cuts $(A_1,A_2)$, $(B_1,B_2)$ produced by the numbers $\alpha$ and $\beta$ in the system $R$ to define the cut $(C_1,C_2)$ which is to correspond to the result of the operation, $\gamma$. I confine myself here to the discussion of the simplest case, that of addition. \par 

If $c$ is any rational number, we put it into the class $C_1$, provided there are two numbers one $a_1$ in $A_1$ and one $b_1$ in $B_1$ such that their sum $a_1+b_1\geqq c$; all other rational numbers shall be put into the class $C_2$. This separation of all rational numbers into the two classes $C_1,C_2$ evidently forms a cut, since every number $c_1$ in $C_1$ is less than ever number $c_2$ in $C_2$. If both $\alpha$ and $\beta$ are rational, then every number $c_1$ contained in $C_1$ is $\leqq\alpha+\beta$, because $a_1\leqq\alpha,\ b_1\leqq\beta$, and therefore $a_1+b_1\leqq\alpha+\beta$; further, if there were contained in $C_2$ a number $c_2<\alpha+\beta$, hence $\alpha+\beta=c_2+p$, where $p$ is a positive rational number, then we would have 
\begin{equation*}
    c_2=(\alpha-\frac{1}{2}p)+(\beta-\frac{1}{2}p),
\end{equation*}
which contradicts the definition of the number $c_2$, because $\alpha-\frac{1}{2}p$ is a number in $A_1$, and $\beta-\frac{1}{2}p$ a number in $B_1$; consequently every number $c_2$ contained in $C_2$ is $\geqq\alpha+\beta$. Therefore in this case the cut $(C_1,C_2)$ is produced by the sum $\alpha+\beta$. Thus we shall not violate the definition which holds in the arithmetic of rational numbers if in all cases we understand by the sum $\alpha+\beta$ of any two real numbers $\alpha,\ \beta$ that number $\gamma$ by which the cut $(C_1,C_2)$ is produced. Further, if only one of the two numbers $\alpha,\ \beta$ is rational, e.g., $\alpha$, it is easy to see that it makes no difference with the sum $\gamma=\alpha+\beta$ whether the number $\alpha$ is put into the class $A_1$ or into the class $A_2$. \par 

Just as addition is defined, so can the other opeartions of the so-called elementary arithmetic be defined, viz., the formation of differences, products, quotients, powers, roots, logarithms, and in this way we arrive at real proofs of theorems (as, e.g., $\sqrt{2}\cdot\sqrt{3}=\sqrt{6}$), which to the best of my knowledge have never been established before. The excessive length that is to be feared in the definitions of the more complicated operations is partly inherent in the nature of the subject but can for the most part be avoided. Very useful in this connection is the notion of an \iit{interval}, i.e., a system $A$ of rational numbers possessing the following characteristic property: if $a$ and $a'$ are numbers of the system $A$, then are all rational numbers lying between $a$ and $a'$ contained in $A$. The system $R$ of all rational numbers, and also the two classes of any cut are intervals. If there exist a rational number $a_1$ which is less and a rational number $a_2$ which is greater than every number of the interval $A$, then $A$ is called a finite interval; there then exist infinitely many numbers in the same condition as $a_1$ and infinitely many in the same condition as $a_2$; the whole domain $R$ breaks up into three parts $A_1,\ A,\ A_2$ and there enter two perfectly definite rational or irrational numbers $\alpha_1,\ \alpha_2$ which may be called respectively the lower and upper (or the less and greater) \iit{limits} of the interval; the lower limit $\alpha_1$ is determined by the cut for which the system $A_1$ forms the second class. Of every rational or irrational number $\alpha$ lying between $\alpha_1 \text{ and } \alpha_2$ it may be said that it lies \iit{within} the interval $A$. If all numbers of an interval $A$ are also numbers of an interval $B$, then $A$ is called a portion of $B$. \par 

Still lengthier considerations seem to loom up when we attempt to adapt the numerous theorems of the arithmetic of rational numbers (as, e.g., the theorem $(a+b)c=ac+bc$) to any real numbers. This, however, is not the case. It is easy to see that it all reduces to showing that the arithmetic operations possess a certain continuity. What I mean by this statement may be expressed in the form of a general theorem: \par 

``If the number $\lambda$ is the result of an opertaion performed on the numbers $\alpha,\ \beta,\ \gamma,\ \dots$ and $\lambda$ lies within the interval $L$, then intervals $A,\ B,\ C,\ \dots$ can be taken within which lie the numbers $\alpha,\ \beta,\ \gamma,\ \dots$ such that the result of the same operation in which the numbers $\alpha,\ \beta,\ \gamma,\ \dots$ are replaced by arbitrary numbers of the intervals $A,\ B,\ C,\ \dots$ is always a number lying within the interval $L$.'' The forbidding clumsiness, however, which marks the statement of such a theorem convinces us that something must be brought in as an aid to expression; this is, in fact, attained in the most satisfactory way by introducing the ideas of \iit{variable magnitudes, functions, limiting values}, and it would be best to base the definitions of even the simplest arithmetic operations upon these ideas, a matter which, however, cannot be carried further here. \par 

\section{Infinitesimal Analysis}
Here at the close we ought to explain the connection between the preceding investigations and certain fundamental theorems of infinitesimal analysis. \par 

We say that a variable magnitude $x$ which passes through successive definite numerical values appoaches a fixed limiting value $\alpha$ when in the course of the process $x$ lies finally between two numbers between which $\alpha$ itself lies, or, what amounts to the same, when the difference $x-\alpha$ taken absolutely becomes finally less than any given value different from zero. \par 

One of the most important theorems may be stated in the following manner: ``If a magnitude $x$ grows continually but not beyond all limits it approaches a limiting value.'' \par 

I prove it in the following way. By hypothesis there exists one and hence there exist infinitely many numbers $\alpha_2$ such that $x$ remains continually $<\alpha_2$; I designate by $\mathfrak{A}_2$ the system of all these numbers $\alpha_2$, by $\mathfrak{A}_1$ the system of all other numbers $\alpha_1$; each of the latter possesses the property that in the course of the process $x$ becomes finally $\geqq\alpha_1$, hence every number $\alpha_1$ is less than every number $\alpha_2$ and consequently there exists a number $\alpha$ which is either the greatest in $\mathfrak{A}_1$ or the least in $\mathfrak{A}_2$ (V, \textsc{iv}). The former cannot be the case since $x$ never ceases to grow, hence $\alpha$ is the least number in $\mathfrak{A}_2$. Whatever numnber $\alpha_1$ be taken we shall have finally $\alpha_1<x<\alpha$, i.e., $x$ approaches the limiting value $\alpha$. \par 

This theorem is equivalent to the principle of continuity, i.e., it loses its validity as soon as we assume a single real number not to be contained in the domain $\Re$; or otherwise expressed: if this theorem is correct, then is also theorem \textsc{iv.} in V. correct. \par 

Another theorem of infinitesimal analysis, likewise equivalent to this, which is still oftener employed, may be stated as follows: ``If in the variation of a magnitude $x$ we can for every given positive magnitude $\delta$ assign a corresponding position from and after which $x$ changes by less than $\delta$ then $x$ approaches a limiting value.'' \par 

This converse of the easily demonstrated theorem that every variable magnitude which approaches a limiting value finally changes by less than anyu given positive magnitude can be derived as well from the preceding theorem as directly from the principle of continuity. I take the latter course. Let $\delta$ be any positive magnitude (i.e., $\delta>0$), then by hypothesis a time will come after which $x$ will change by less than $\delta$, i.e., if at this time $x$ has the value $a$, then afterwards we shall continually have $x>a-\delta$ and $x<a+\delta$. I now for a moment lay aside the original hypothesis and make use only of the theorem just demonstrated that all later values of the variable $x$ lie between two assignable finite values. Upon this I vase a double separation of all real numbes. To the system $\mathfrak{A}_2$ I assign a number $\alpha_2$ (e.g., $a+\delta$) when in the course of the process $x$ becomes finally a $\leqq\alpha_2$; to the system $\mathfrak{A}_1$ I assign every number not contained in $\mathfrak{A}_2$; if $\alpha_1$ is such a number, then, however far the process may have advanced, it will still happen infinitely many times that $x>\alpha_2$. Since every number $\alpha_1$ is less than every number $\alpha_2$ there exists a perfectly definite number $\alpha$ which produces this cut ($\mathfrak{A}_1,\mathfrak{A}_2$) of the system $\Re$ and which I will call the upper limit of the variable $x$ which always remains finite. Likewise as a result of the behavior of the variable $x$ a second cut ($\mathfrak{B}_1,\mathfrak{B}_2$) of the system $\Re$ is produced; a number $\beta_2$ (e.g., $a-\delta$) is assigned to $\mathfrak{B}_2$ when in the course of the process $x$ becomes finally $\geqq\beta$; every other number $\beta_2$, to be assigned to $\mathfrak{B}_2$, has the property that $x$ is never finally $\geqq\beta_2$; therefore infinitely many times $x$ becomes $\leqq\beta_2$; the nubmer $\beta$ by which this cut is produced I call the lower limiting value of the variable $x$. The two numbers $\alpha,\ \beta$ are obviously characterised by the following property: if $\epsilon$ is an arbitrarily small positive magnitude then we have always finally $x<\alpha+\epsilon$ and $x>\beta-\epsilon$, but nevert finally $x<\alpha-\epsilon$ and never finally $x>\beta+\epsilon$. Now two cases are possible. If $\alpha$ and $\beta$ are different from each other, then necessarily $\alpha>\beta$, since continually$\alpha_2\geqq\beta_2$; the variable $x$n oscillates, and, however far the process advances, always undergoes changes whose amount surpasses the value $(\alpha-\beta)-2\epsilon$ where $\epsilon$ is an arbitrarily small positive magnitude. The original hypothesisto which I now return contradicts this consequence; there remains only the second case $\alpha=\beta$ since it has already been shoiwn that, however small be the positive magnitude $\epsilon$, we always have finally $x<\alpha+\epsilon$ and $x>\beta-\epsilon$, $x$ approaches the limiting value $\alpha$, which was to be proved. \par 

These examples may suffice to bring out the connection between the principle of continuity and infinitesimal analysis. \par 

\chapter{The Nature and Meaning of Numbers}
\end{document}