\documentclass[12pt,oneside]{book}

\usepackage{fullpage}
\usepackage{fancyhdr}
\usepackage{setspace,titlesec}

\onehalfspace 
\setlength{\parskip}{12pt plus 1pt minus 2pt}
\setlength{\parindent}{0pt}
\titleformat{\chapter}[hang]{\Huge\bfseries}{\thechapter.\hspace{20pt}}{0pt}{\Huge\bfseries}

\begin{document}
    
\frontmatter

\noindent The Project Gutenberg EBook of Beethoven's Letters 1790-1826, Vol. 1 of 2 \\
by Lady Wallace

This eBook is for the use of asnyone anywhere at not cost and with almost no restrictions whatsoever. 
You may copy it, give it away or re-use it under the terms of the Project Gutenberg License included with this eBook or online at www.gutenberg.net

Produced by Juliet Sutherland, John Williams and the Online Distributed Proofreading Team.

\begin{titlepage}
    \centering
    \vspace*{4cm}
    {\Huge \textbf{Beethoven's Letters} \par}
    \vspace{2cm}
    {\LARGE Lady Wallace \par}
    \vspace{1cm}
    {\large 1790-1826 \par }
\end{titlepage}

\chapter{Translator's Preface}
Since undertaking the translation of Dr.~Ludwig Nohl's valuable edition of ``Beethoven's Letters,''~
an additional collection has been published by Dr.~Ludwig Ritter von K\"ochel,
consisting of many interesting letters addressed by Beethoven to his illustrious pupil, H.R.H the Archduke Rudolph,
Cardinal-Archbishop of Olm\"utz.
These I have inserted in chronological order, and marked with the letter K., in order to distinguish them from the correspondence edited by Dr.~Nohl.
I have only omitted a few brief notes, consisting merely of apologies for non-attendance on the Archduke. \par 

The artistic value of these newly discovered treasues will no doubt be as highly appreciated in this country as in the great \textit{maestro's} Father-land. \par 

I must also express my gratitude to Dr.~Th.G.~v.~Karajan,
for permitting an engraving to be made expressly for this work,
from an original Beethoven portrait in his possession,
now for the first time given to the public.
The grad and thoughtful countenance forms a fitting introduction to letters so truly depicting the brilliant, fitful genious of the sublime master, 
as well as the touching sadness and gloom pervading his life,
which his devotion to Art alone brightened,
through many bitter trials and harassing cares. \par 

The love of Beethoven's music is now become so universal in England, that I make no doubt his Letters will receive a hearty welcome from all those whose spirits have been elevated and soothed by the genius of this illustrious man. \par 
\begin{flushright}
    GRACE WALLACE.
\end{flushright}
AINDERBY HALL, March 28, 1866.
\pagebreak

\chapter{Preface by Dr.~Ludwig Nohl to the Letters of Ludwig van Beethoven}
In accompanying the present edition of the Letters of Ludwig van Beethoven with a few introductory remarks,
I at once acknowledge that the compilation of these letters has cost me no sligh sacrifices.
I must also, however, mention that an unexpected Christmas donation, generously bestowed upon me with a view to further my efforts to promote
the science of music, enabled me to undertake one of the journeys necessary for my purpose, and also to 
complete the revision of the Letters and of the press, in the milder air and repose of a country residence,
long since recommended to me for the restoration of my health, undermined by overwork. \par

That, in spite of every effort, I have not succeeded in seeing the original of each letter, or even discovering
the place where it exists, may well be excused, taking into consideration the slender capabilities of an individual,
and the astonishing manner in which Beethoven's Letters are dispersed all over the world. At the same time, I must state that not only have 
the hitherto inaccessible treasures of Anton Schindler's ``Beethoven's Nachless'' been placed at my disposal,
but also other letters from private sources, owing to various happy chances, and the kindness and complaisance of 
collectors of autographs. I know better, however, than most people--being in a position to do so--that in the 
present work there can be no pretension to any thing approaching to a complete collection of Beethoven's Letters.
The master, so fond of writing, though he often rather amusingly accuses himself of being a lazy correspondent, 
may very probably have sent forth at least double the amount of the letters here given, and there is no doubt whatever 
that a much larger number are still extant in the originals. The only thing that can be done at this moment, however,
is to make the attempt to bring to light, at all events, the letters that could be discovered in Germany. The mass 
of those which I gradually accumulated, and now offer to the public (with the exception of some insignificant notes), 
appeared to me sufficiently numerous and important to interest the world, and also to form a substantial nucleus for
any letters that may hereafter be discovered. On the other hand, as many of Beethoven's Letters slumber in foreign lands,
especially in the unapproachable cabinets of curiosities belonging to various close-fisted English collectors, an entire 
edition of the correspondence could only be effected by a most disproportionate outlay of time and expense. \par 

When revising the text of the Letters, it seemed to me needless perpetually to impair the pleasure of the reader
by retaining the mistakes in orthography; but enough of the style of writing of that day is adhered to, to prevent 
its peculiar charm being entirely destroyed. Distorted and incorrect as Beethoven's mode of expression sometimes is,
I have not presumed to alter his grammar, or rather syntax, in the smallest degree:~who would presume to do so with
an individuality which, even amid startling clumsiness of style, displays those inherent intellectual powers 
that often did violence to language as well as to his fellow-men? Cyclopean masses of rock are here hurled with 
Cyclopean force; but hard and massive as they are, the man is not to be nvied whose heart is not touched by these glowing 
fragments, flung apparently at random right and left, like meteors, by a mightly intellectual being, however perverse 
the treatment language may have received from him. \par 

The great peculiarity, however, in this strange mode of expression is, that even such incongruous language 
faithfully reflects the mind of the man whose nature was of prophetic depth and heroic force;
and who that knows anything of the creative genius of a Beethoven can deny him these attributes?\par

The antique dignity pervading the whole man, the ethical contemplation of life forming the basis of his nature,
prevented even a momentary wish on my part to efface a single word of the oft-recurring expressions so painfully
harsh, bordering on the unaesthetic, and even on the repulsive, provoked by his wrath against the meanness of men.
In the last part of these genuine documents, we learn with a feeling of sadness, and with almost a tragic sensation,
how low was the standard of moral worth, or rather how great was the positive unworthiness, of the intimate society
surrounding the master, and with what difficulty he could maintain the purity of the nobler part of his being in such
an atmosphere. The manner, indeed, in which he strives to do so, fluctuating between explosions of harshness and almost
weak yieldingness, while striving to master the base thoughts and conduct of these men, though never entirely succeeding
in doing so, is often more a diverting than an offensive spectacle. In my opinion, nevertheless, even this less pleasing
aspect of the Letters ought not to be in the lsightest degree softened (which it has hitherto been, owing to false 
views of propriety and morality), for it is no moral deformity here displayed. Indeed, even when the irritable 
master has recourse to expressions repgnant to our sense of conventionality, and which may well be called harsh and rough, still the wrath that seizes on our hero is a just and righteous wrath, and we disregard it, just as in Nature, whose grandeur constantly elevates us above the inevitable stains of an earthly soil. The coarseness and ill-breeding, which would claim toleration because this great man now and then showed such feelings, must beware of doing so, being certain to make shipwreck when coming in contact with the massive rock of true morality on which, with all his faults and deficiencies, Beethoven's being was surely grounded. Often, indeed, when absorbed in the unsophisticated and genuine utterances of this great man, it seems as if these peculiarities and strange asperities were the results of some mysterious law of Nature, so tha we are inclined to adopt the paradox by which a wit once described the singular groundwork of our nature,-``The faults of man are the night in which he rests from his virtues.'' \par 

Indeed, I think that the lofty morality of such natures is not fully evident until we are obliged to confess with regret, that even the great ones of the earth must pay their tribute to humanity, and really do pay it (which is the distinction between them and base and petty characters), without being ever entirely hurled from their pedestal of dignity and virtue. The soul of that man cannot fail to be elevated, who can seize the real spirit of the scattered pages that a happy chance has preserved for us. If not fettered by petty feelings, he will quickly surount the causal obstacles and stumbling-blocks which the first perusal of these Letters may seem to present, and quickly feel himself transported at a single stride into a stream, where a strange roaring and rushing is heard, but above which loftier tones resound with magic and exciting power. For a peculiar life breathes in these lines; an undercurrent runs through their apparently unconnected import, uniting them as with an electric chain, and with firmer links than any mere coherence of subjects could have effected. I experienced this myself, to the most remarkable degree, when I first made the attempt to arrange, in accordance with their period and substance, the hundreds of individual pages bearing neither date nor address, and I was soon convinced that a connected text (such as Mozart's Letters have, and ought to have) would be here entirely superfluous, as even the best biographical commentary would be very dry work, interrupting the electric current of the whole, and thus destroying its peculiar effect. \par 

And now, what is this spirit which, for an intelligent minds, binds together these scattered fragments into a whole, and what is its actual power? I cannot tell; but I feel to this day just as I felt to the innermost depths of my heart in the days of my youth when I first heard a symphony of Beethoven's,--that a spirit breathes from it bearing us aloft with giant power out of the oppressive atmosphere of sense, stirring to its inmost reesses the heart of man, bringing him to the full consciousness of his loftier being, and of the undying within him. And even more distinctly than when a new world was thus disclosed to his youthful feelings is the \textit{man} fully conscious that not only was this a new world to him, but a new world of feeling in itself, revealing to the spirit phases of its own, which, till Beethoven appeared, had never before been fathomed. Call it by what name you will, when one of the great works of the sublime master is heard, whether indicative of proud self-consciousness, freedom, spring, love, storm, or batte, it grasps the soul with singular force, and enlarges the laboring breast. Whether a man understands music or not, every one who has a heart beating within his breast will feel with enchantment that here is concentrated the utmost promused to us by the most imaginative of our poets, in bright visions of happiness and freedom. Even the only great hero of action, who in those memorable days is worthy to stand beside the great master of harmony, having diffused among mankind new and priceless earthly treasures, sinks in the scale when we compare these with the celestial treasures of a purified and deeper feeling, and a more free, enlarged, and sublime view of the world, struggling gradually and distinctly upwards out of the mere frivolity of an art devoid of words to express itself, and impressing its stamp on the spirit of the age. They convey, too, the knowledge of this brightest victory of genuine German intellect to those for whom the sweet Musc of Music is as a book with seven seals, and reveal, likewise, a more profound sense of Beethoven's bing to many who already, through the sweet tones they have imbibed, enjoy some dawning conviction of the master's grandeur, and who now more and more eagerly lend a listening ear to the intellectual clearly worded strains so skilfully interwoven, thus soon to arrive at the full and blissful comprehension of those grand outpourings of the spirit, and finally to add another bright delight to the enjoyment of those who already know and love Beethoven. All these may be regarded as the objects I had in view when I undertook to edit his Letters, which ahve also bestowed on myself the best recompense of my labors, in the humble conviction that by this means I may have vividly reawakened in the remembrance of many the might mission which our age is called on to perform for the development of our race, even in the realm of harmony,--more especially in our Father-land. \par 
\begin{flushright}
    LUDWIG NOHL.
\end{flushright}
LA TOUR DE PERLZ--LAKE OF GENEVA, March, 1865.

\tableofcontents 

\mainmatter

\part{Life's Joys and Sorrows\\1783 to 1815}

\chapter[To the Elector of Cologne,\\Frederick Maximilian]{To the Elector of Cologne,\\Frederick Maximilian\raisebox{.3\baselineskip}{\normalsize\footnotemark}}
\footnotetext{The dedication affixed to this work, ``Three Sonatas for the Piano, dedicated to my illustrious master, Maximilian Friedrich, Archbishop and Elector of Cologne, by Ludwig van Beethoven in his eleventh year,'' is probably not written by the boy himself, but is given here as an amusic contrast to his subsequent ideas with regard to the homage due to rank.}

ILLUSTRIOUS PRINCE,--

Music from my fourth year has ever been my favorite pursuit. Thus early introduced to the sweet Muse, who attuned
my soul to pure harmony, I loved her, and sometimes ventured to think that I was beloved by her in return. I have now attained my 
eleventh year, and my Muse often whispered to me in hours of inspiration,--Try to write down the harmonies in your soul. 
Only eleven yearas old! thought I; does the character of an author befit me? and what would more mature artists say?
I felt some trepidation; but my Muse willed it--so I obeyed, and wrote. \par
\begin{flushright}
    LUDWIG VAN BEETHOVEN.
\end{flushright}


\chapter{To Dr. Shade,--Augsburg}
\begin{flushright}
    Bonn, 1787. Autumn
\end{flushright}
MY MOST ESTEEMED FRIEND,-- \par 

I can easily image what you must think of me, and I cannot deny that you have too good grounds for an unfavorable opinion. I shall not, however, attempt to justify myself, until I have explained to you the reasons why my apologies should be accepted. I must tell you that from the time I left Augsburg\footnote{On his return from Vienna, whither Max Franz had sent him for the further cultivation of his talents.} my cheerfulness, as well as my health, began to decline; the nearer I came to my native city, the more frequent were the letters from my father, urging me to travel with all possible speed, as my mother's health was in a most precarious condition. I therefore hurried forwards as fast as I could, although myself far from well. My longing once more to see my dying mother overcame every obstacle, and assisted me in surmounting the greatest difficulties. I found my mother indeed still alive, but in the most deplorable state; her disease was consumption, and about seven weeks ago, after much pain and suffering, she died [July 17]. She was indeed a kind, loving mother to me, and my best friend. Ah! who was happier than I, when I could still utter the sweet name of mother, and it was heard? But to whom can I now say it? Only to the silent form resembling her, evoked by the power of imagination. I have passed very few pleasant hours since my arrival here, having during the whole time been suffering from asthma, which may, I fear, eventually turn to consumption; to this is added melancholy,--almost as great an evil as my malady itself. Imagine yourself in my place, and then I shall hope to receive your forgiveness for my long silence. You showed me extreme kindness and friendship by lending me three Carolins in Augsburg, but I must entreat your indulgence for a time. My journey cost me a great deal, and I have not the smallest hopes of earning anything here. Fate is not propitious to me in Bonn. Pardon my intruding on you so long with my affairs, but all tat I have said was necessary for my own justification. \par 

I do entreat you not to deprive me of your valuable friendship; nothing do I wish so much as in any degree to become worthy of your regard. I am, with all esteem, your obedient servant and friend, 
\begin{flushright}
    L. V. BEETHOVEN,\\
    \textit{Cologne Court Organist.}
\end{flushright} \par 

\chapter[To the Elector Maximilian Francis]{To the Elector Maximilian Francis\raisebox{.3\baselineskip}{\normalsize\footnotemark}}
\footnotetext{An electoral decree was issued in compliance with this request on May 3, 1793.}
MOST ILLUSTRIOUS AND GRACIOUS PRINCE,-- \par 

Some years ago your Highness was pleased to grant a pension to my father, the Court tenor Van Beethoven, and further graciously to decree that 100 R. Thalers of his salary should be allotted to me, for the purpose of maintaining, clothing, and educating my two younger brothers, and also defraying the debts incurred by our father. It was my intention to present this decree to your Highness's treasurer, but my father earnestly implored me to desist from doing so, that he ight not be thus publicly proclaimed incapable himself of supporting his family, adding that he would engage to pay me the 25 R.T. quarterly, which he punctually did. After his death, however (in December last), wishing to reap the benefit of your Highness's gracious boon, by presenting the decree, I was startled to find that my father had destroyed it. \par 

I therefore, with all dutiful respect, entreat your Highness to renew this decree, and to order the paymaster of your Highness's treasury to grant me the last quarter of this benevolent addition to my salary (due the beginning of February). I have the honor to remain, \par 

Your Highness's most obedient and faithful servent, 
\begin{flushright}
    LUD. V. BEETHOVEN,\\
    \textit{Court Organist.}
\end{flushright} \par

\chapter{To Eleonore von Breuning,--Bonn}
MY HIGHLY ESTEEMED ELEONORE, MY DEAREST FRIEND,-- \par 

A year of my stay in this capital has nearly elapsed before you receive a letter from me, and yet the most vivid remembrance of you is ever present with me. I have often conversed in thought with you and your dear family, though not always in the happy mood I could have wished, for that fatal misunderstanding still hovered before me, and my conduct at that time is now hateful in my sight. But so it was, and how much would I give to have the power wholly to obliterate from my life a mode of acting so degrading to myself, and so contrary to the usual tenor of my character! \par 

Many circumstances, indeed, contributed to estrange us, and I suspect that those tale-bearers who repeated alternately to you and to me our mutual expressions were the chief obstacles to any good understanding between us. Each believed that what was said proceeded from deliberate conviction, whereas it arose only from anger, fanned by others; so we were both mistaken. Your good and noble disposition, my dear friend, is sufficient security that you have long since forgiven me. We are told that the best proof of sincere contrition is to acknowledge our faults; and this is what I wish to do. Let us now draw a veil over the whole affiar, learning one lesson from it,--that when friends are at variance, it is always better to employ no mediator, but to communicate directly with each other. \par 

With this you will receive a dedication from me [the variations on ``Se vuol ballare'']. My sole wish is that the work were greater and more worthy of you. I was applied to here publish this little work, and I take advantage of the opportunity, my beloved Eleonore, to give you a proof of my regard and friendship for yourself, and also a token of my enduring remembrance of your family. Pray then accept this trifle, and do not forget that it is offered by a devoted friend. Oh! if it only gives you pleasure, my wishes will be fulfilled. May it in some degree recall the time when I passed so many happy hours in your house! Perhaps it may serve to remind you of me till I return, though this is indeed a distant prospect. Oh! how we shall then rejoice together, my dear Eleonore! You will, I trust find your friend a happier man, all former forbidding, careworn furrows smoothed away by time and better fortune. \par 

\end{document}
