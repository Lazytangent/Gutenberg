\documentclass[oneside,12pt]{book}

\usepackage{mathtools,amsthm,amssymb,setspace}
\usepackage{fancyhdr,fullpage,array,graphicx,multicol}

\onehalfspace 
\setlength{\parskip}{12pt}
%\setlength{\parindent}{0pt}

\begin{document}
    
\frontmatter

\noindent The Project Gutenberg EBook of Matter, Ether, and Motion, Rev. ed., enl., by Amos Emerson Dolbear \par 

\noindent This eBook is for the use of anyone anywhere at no cost and with almost no restrictions whatsoever. You may copy it, give it away or re-use it under the terms
of the Project Gutenberg License included with this eBook or online at www.gutenberg.org \par 

\pagebreak

\noindent Prooduced by Andrew D. Hwang, Peter Vachuska, Chuck Greif and the Online Distributed Proofreading Team at http://www.pgdp.net \par 

\vfill

\begin{center}
    Transcriber's Note
\end{center}
Minor typographical corrections and presentational changes have been made without comment. Illustrations may have been moved slightly relative to the surrounding text. \par 

\noindent Aside from clear misspellings, every effort has been made to preserve variations of spelling and hyphenation from the original. \par 

\noindent This PDF file is optimized for screen viewing, but may easily be recompiled for printing. Please see the preamble of the \LaTeX\ source file for instructions. \par 
\begin{titlepage}
    \centering
    {\Huge \textbf{MATTER, ETHER, AND MOTION} \par}
    \vspace{2cm}
    {\LARGE \textit{THE FACTORS AND RELATIONS\\OF\\PHYSICAL SCIENCE} \par}
    \vspace{2cm}
    {\normalsize BY \par}
    {\Large A. E. DOLBEAR PH. D. \par}
    {\tiny PROFESSOR OF PHYSICS TUFTS COLLEGE\\AUTHOR OF ``THE TELEPHONE'' ``THE ART OF PROJECTING'' ETC. \par}
    \vspace{2cm}
    {\small REVISED EDITION, ENLARGED \par}
    \vfill
    {\large BOSTON\\LEE AND SHEPARD PUBLISHERS \par}
    {\small 10 MILK STREET \par} 
    {\large 1894 \par}
\end{titlepage}

\chapter{Preface to the Second Edition}
The issue of a new edition of this book gives me an opportunity to make some needed corrections, and enlarge it by the addition of three new chapters,
which I hope will make it more useful to such as have a taste for fundamental physical problems. The first of these, Prooperties of Matter as Modes of Motion,
presents the evidence that all the characteristic properties of matter are due to energy embodied in various forms of motion. The second, on The 
Implications of Physical Phenomena, points out what assumptions are made in explaining phenomena. It is the substance of a series of articles published in the 
\textit{Psychical Review} in 1892 and 1893. The third, on The Relations between Physical and Psychical Phenomena, was read as a paper befor the Psychical Congress
at the World's Fair in August, 1893. \par 

Judging from some of the comments made about my statement as to Modern Geometry on page 67 % Insert reference to correct page/item later
and as to Vital Force, p. 336, % Same as above 
I have thought it would be useful to some to see corroboratory statements; and I have therefore added, in an appendix, a few pages of quotations from some 
of the most eminent mathematicians and biologists on these subjects, and from them one may judge whether or not my statements are correct. \par 

As the work is a treatise on Physics, there is no special reason for going beyond it; but if this presentation of the subject is any approach to the truth,
there is an important conclusion to be drawn from it. If the ehter be the homogeneous and uniform medium it is believed with reason to be, then, in the absence
of what we call matter, no physical change which we call a phenomenon could possibly arise in it; for every such phenomenon is a product, and in the absence of
one of the essential factors, viz., matter, it could not be. If matter itself be a form of motion of the ether, the ehter must have existed prior to matter; also,
if the atom be a form of energy, then must energy have existed befor matter existed. Hence there must have been some other agency radically different from any physical 
energy we know, and independent of everything we know, which was capable of producing orderly physical phenomena, by acting upon the ether; for a homogeneous medium could not originate it. 
Some philosophers call this antecedent power The Unknowable; others call it God. If energy \textit{as we know it} implies antecedent energy as we do not know it,
so, likewise, mind as we know it implies antecedent mind under totally different conditions from those in which we find it embodied. \par 

In whatever direction one pursues physical science, he is at last confronted with a physical phenomenon with a superphysical antecedent where all physical methods of investigation are 
impotent. Such considerations raise the theistic hypothesis of creation to the rank of such physical theories as the nebula theory of the origin of the solar system, and the undulatory
theoryh of light. \par 

\chapter{Preface}
Within the past fifty years the advance in physical knowledge has not only been rapid, but it has been well-nigh revolutionary. Not that knowledge that was felt 
to be well grounded before has been set aside,--for it has not been,--but the fundamental prionciples of natural philosophy that were applied by Sir Isaac Newton 
and others to masses of visible magnitude have been applied to molecules; and it has thus been discovered that all kinds of phenomena are subject to the same mechanical
laws. It was thought before that physics embraces several distinct provinces of knowledge which were not necessarily related to each other, such as mechanics,
heat, electricity, etc. Such terms as imponderable matter, latent heat, electric fluid, forces of nature, and others in commone use in text-books and elsewhere, 
served to maintain the distinctions; and even to-day some of these obsolete physical agencies are to be met in books and places where one would hope not to find them. 
As all physical phenomena are reducible to the principles of mechanics, atoms and molecules are subejct to them as much as masses of visible magnitude; and it has become apparent 
that however different one phenomenon is from another, the factors of both are the same,--matter, ether, and motion; so that all the so-called forces of nature, considered as objective 
things controlling phenomena, are seen to have no existence; that all phenomena are reducible to nothing more mysterious than a push or a pull. \par 

% Continue here

\end{document}