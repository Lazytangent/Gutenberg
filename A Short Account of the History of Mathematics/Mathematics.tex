\documentclass[12pt,oneside]{book}

\usepackage{fancyhdr}
\usepackage{setspace}
\usepackage{mathtools,amssymb,amsthm}
\usepackage{array,graphicx,multicol,hhline,parskip}
\usepackage[margin=1in]{geometry}
\onehalfspace 

\setlength{\parskip}{12pt minus2pt}
\setlength{\parindent}{0pt}

\begin{document}
    
\frontmatter

The Project Gutenberg EBook of A Short Account of the History of Mathematics, by W. W. Rouse Ball \par 

This eBook is for the use of anyone anywhere at no cost and with almost no restrictions whatsoever. You may copy it, give it away or re-use it under
the terms of the Project Gutenberg License included with this eBook or online at www.gutenberg.org \par
\vspace{12pt}

\begin{titlepage}
    \centering
    \vspace*{2cm}
    {\LARGE A SHORT ACCOUNT \par}
    \vspace{1cm}
    {\Large OF THE \par}
    \vspace{1cm}
    {\Huge HISTORY OF MATHEMATICS \par}
    \vspace{2cm}
    {\Large BY \par}
    \vspace{1cm}
    {\Large W. W. ROUSE BALL \par}
    {\normalsize FELLOW OF TRINITY COLLEGE, CAMBRIDGE \par}
    \vfill
    {\Large DOVER PUBLICATIONS, INC. \par}
    {\Large NEW YORK \par}
\end{titlepage}

\begin{center}
    This new Dover edition, first published in 1960, is an unabridged and unaltered republication of the author's last revision -- the fourth edition
    which appeared in 1908. \par 
    \vspace{7cm}
    \textit{International Standard Book Number: 0-486-20630-0\\
    Library of Congress Catalog Card Number: 60-3187}
    \vfill
    Manufactured in the United States of America \\
    Dover Publications, Inc. \\
    180 Varick Street \\
    New York, N. Y. 10014 
\end{center}

\chapter{Preface}

The subject-matter of this book is a historical summary of the development of mathematics, illustrated by the lives and discoveries of those to
whom the progress of the science is mainly due. It may serve as an introduction to more elaborate works on the subject, but primarily it is intended
to give a short and popular account of those leading facts in the history of mathematics which many who are unwilling, or have not the time, to study
it systematically may yet desire to know. 

The first edition was substantially a transcript of some lectures which I delivered in the year 1888 with the object of giving a sketch of the history,
previous to the nineteenth century, that should be intelligible to any one acquainted with the elements of mathematics. In the second edition, issued in 
1893, I rearranged parts of it, and introduced a good deal of additional matter. \par 

The scheme of arrangement will be gathered from the table of contents at the end of this preface. Shortly it is as follows. The first chapter contains
a brief statement of what is known concerning the mathematics of the Egyptians and Phoenicians; that is introductory to the history of mathematics
under Greek influence. The subsequent history is divided into three periods: first, that under Greek influence, chapters II to VII; second, that of 
the middle ages and renaissance, chapters VIII to XIII; and lastly that of modern times, chapters XIV to XIX. \par 

In discussing the mathematics of these periods I have confined myself to giving the leading events in the history, and frequently have passed in silence 
over men or works whose influence was comparatively unimportant. Doubtless an exaggerated view of the discoveries of those mathematicians who are mentioned 
may be caused by the non-allusion to minor writers who preceded and prepared the way for them, but in all historical sketches this is to some extent inevitable,
and I have done my best to guard against it by interpolating remarks on the progress of the science at different times. Perhaps also I should here state that
generally I have not referred to the results obtained by practical astronomers and physicists unless there was some mathematical interest in them.
In quoting results I have commonly made use of modern notation; the reader must therefore recollect that, while the matter is the same as that of any writer
to whom allusion is made, his proof is sometimes translated into a more convenient and familiar language. \par 

The greater part of my account is a compilation from existing histories or memoirs, as indeed must be necessarily the case where the works discussed are
so numerous and cover so much ground. When authorities disagree I have generally stated only that view which seems to me to be the most probable; but if
the question be one of importance, I believe that I have always indicated that there is a difference of opinion about it. \par 

I think that it is undesirable to overload a popular account with a mass of detailed references or the authority for every particular fact mentioned.
For the history previous to 1758, I need only refer, once for all, to the closely printed pages of M.~Cantor's monumental \textit{Vorlesungen \"uber die Geschichte der Mathematik}
(hereafter alluded to as Cantor), which may be regarded as the standard treatise on the subject, but usually I have given references to the other leading 
authorities on which I have relied or with which I am acquainted. My account for the period subsequent to 1758 is generally based on the memoirs or monographs referred to
in the footnotes, but the main facts to 1799 have been also enumerated in a supplementary volume issued by Prof.~Cantor last year. I hope that my footnotes will supply 
the means of studying in detail the history of mathematics at any specified period should the reader desire to do so. \par 

My thanks are due to various friends and correspondents who have called my attention to points in the previous editions. I shall be grateful for notices
of additions or corrections which may occur to any of my readers. \par

\begin{flushright}
    {\Large W. W. ROUSE BALL}
\end{flushright}
{\footnotesize TRINITY COLLEGE, CAMBRIDGE}

\chapter{Note}
The fourth edition was stereotyped in 1908, but no material changes have been made since the issue of the second edition in 1893, other duties having,
for a few years, rendered it impossible for me to find time for any extensive revision. Such revision and incorporation of recent researches 
on the subject have now to be postponed till the cost of printing has fallen, though advantage has been taken of reprints to make trivial corrections
and additions. \par 

\begin{flushright}
    {\Large W. W. R. B.}
\end{flushright}
{\footnotesize TRINITY COLLEGE, CAMBRIDGE}

\tableofcontents

\mainmatter

\chapter{Egyptian and Phoenician Mathematics}

The history of mathematics cannot with certainty be traced back to any school or period before that of the Ionian Greeks.
The subsequent history may be divided into three periods, the distinctions between which are tolerably well marked.
The first period is that of the history of mathematics under Greek influence, that is discussed in chapters II to VII;
the second is that of the mathematics of the middle ages and the renaissance, this is discussed in chapters VIII to XIII;
the third is that of modern mathematics, and this is discussed in chapters XIV to XIX. 

Although the history of mathematics commences with that of the Ionian schools, there is no doubt that those Greeks who first paid attention 
to the subject were largely indebted to the previous investigations of the Egyptians and Phoenicians. Our knowledge of the mathematical
attainments of those races is imperfect and partly conjectural, but, such as it is, it is here briefly summarized. The definite
history begins with the next chapter.

On the subject of prehistoric mathematics, we may observe in the first place that, though all early races which have left records behind them knew something of numeration
and mechanics, and though the majority were also acquainted with the elements of land-surveying, yet the rules which they possessed were in general founded only on the 
results of observation and experiment, and were neither deduced from nor did they form part of any science. The fact then that various nations in the vicinity of Greece had reached
a high state of civilization does not justify us in assuming that they had studied mathematics. \par 

The only races with whom the Greeks of Asia Minor (amongst whom our history begins) were likely to have come into frequent contact were those inhabiting the eastern littoral of
the Mediterranean; and Greek tradition uniformly assigned the special development of geometry to the Egyptians, and that of the science of numbers either to the 
Egyptians or to the Phoenicians. I discuss these subjects separately. \par 

\part{First Period\\Mathematics under Greek Influence}

\textit{This period begins with the teaching of Tales, circ.} 600 B.C., \textit{and ends with the capture of Alexandria by the Mohammedans in or about} 641 A.D. \textit{The 
characteristic feature of this period is the development of Geometry.} \par 

It will be remembered that I commenced the last chapter by saying that the history of mathematics might be divided into three periods, namely, that of mathematics under Greek
influence, that of the mathematics of the middle ages and of the renaissance, and lastly that of modern mathematics. The next four chapters (chapters II, III, IV, and V) deal 
with the history of mathematics under Greek influence: to these it will be convenient to add one (chapter VI) on the Byzantine school, since through it the results of Greek mathematics
were transmitted to western Europe; and another (chapter VII) on the systems of numeration which were ultimately displaced by the system introduced by the Arabs. I should add that 
many of the dates mentioned in these chapters are not known with certainty, and must be regarded as only approximately correct. \par 

There appeared in December 1921, just before this reprint was struck off, Sir T. L. Heath's work in 2 volumes on the History of Greek Mathematics. This may now be taken as the 
standard authority for this period. \par 

\chapter[Ionian and Pythagorean Schools]{The Ionian and Pythagorean Schools\\Circ. 600 B.C.--400 B.C.}\footnotemark

With the foundation of the Ionian and Pythagorean schools we emerge from the region of antiquarian research and conjecture into the light of history. The materials at our disposal for estimating the knowledge of the philosophers of these schools previous to about the year 430 B.C. are, however, very scanty. Not only have all but fragments of the different mathematical treatises then written been lost, but we possess no copy of the history of mathematics written about 325 B.C. by Eudemus (who was a pupil of Aristotle). Luckily Proclus, who about 450 A.D. wrote a commentary on the earlier part of Euclid's \textit{Elements}, was familiar with Eudemus's work, and freely utilized it in his historical references. We have also a fragment of the \textit{General View of Mathematics} written by Geminus about 50 B.C., in which the methods of proof used by the early Greek geometricians are compared with those current at a later date. In addition to these general statements we have biographies of a few of the leading mathematicians, and some scattered notes in various writers in which allusions are made to the lives and works of others. The original authorities are criticized and discussed at length in the works mentioned in the footnote to the heading of the chapter. \par 

\footnotetext{The history of these schools has been discussed by G. Loria in his \textit{Le Scienze Esatte nell' Antica Grecia}, Moderna, 1893-1900; by Cantor, chaps. v-viii; by G.J. Allman in his \textit{Greek Geometry from Thales to Euclid}, Dublin, 1889; by J. Gow, in his \textit{Greek Mathematics}, Cambridge, 1884; by C.A. Bretschneider in his \textit{Die Geometrie und die Geometer vor Eukleides}, Leipzig, 1870; and partially by H. Hankel in his posthumous \textit{Geschichte der Mathematik}, Leipzig, 1874.}

\textbf{Thales.}\footnotemark\,  The founder of the earliest Greek school of mathematics and philosophy was \textit{Thales}, one of the seven sages of Greece, who was born about 640 B.C. at Miletus, and died in the same town about 550 B.C. The materials for an account of his life consist of little more than a few anecdotes which have been handed down by tradition. \par 

\footnotetext{See Loria, book I, chap. ii; Cantor, chap. v; Allman, chap. i.}

During the early part of his life Thales was engaged partly in commerce and partly in public affairs; and to judge by two stories that have been preserved, he was then as distinguished for shrewdness in business and readiness in resource as he was subsequently celebrated in science. It is said that once when transporting some salt which was loaded on mules, one of the animals slipping in a stream got its load wet and so caused some of the salt to be dissolved, and finding its burden thus lightened it rolled over at the next ford to which it came; to break it of this trick Thales loaded it with rags and sponges which, by absorbing the water, made the load heavier and soon effectually cured it of its troublesome habit. At another time, according to Aristotle, when there was a prospect of an unusually abundant crop of olives Thales got possession of all the olive-presses of the district; and, having thus ``cornered'' them, he was able to make his own terms for lending them out, or buying the olives, and thus realized a large sum. These tales may be apocryphal, but it is certain that he must have had considerable reputation as a man of affairs and as a good engineer, since he was employed to construct an embankment so as to diver the river Halys in such a way as to permit the construction of a ford. \par 

Probably it was as a merchant that Thales first went to Egypt, but during his leisure there he studied astronomy and geometry. He was middle-aged when he returned to Miletus; he seems then to have abandoned business and public life, and to have devoted himself to the study of philosophy and science---subjects which in the Ionian, Pythagorean, and perhaps also the Athenian schools, were closely connected: his views on philosophy do not here concern us. He continued to live at Miletus till his death circ. 550 B.C. \par 

We cannot form any exact idea as to how Thales presented his geometrical teaching. We infer, however, from Proclus that it consisted of a number of isolated propositions which were not arranged in a logical sequence, but that the proofs were deductive, so that the theorems were not a mere statement of an induction from a large number of special instances, as probably was the case with the Egyptian geometricians. The deductive character which he thus gave to the science is his chief claim to distinction. \par 

The following comprise the chief propositions that can now with reasonable probability be attributed to him; they are concerned with the geometry of angles and straight lines. \par 

\hfill\begin{minipage}{\dimexpr\textwidth-2cm}
(i) The angles at the base of an isosceles triangle are equal (Euc. I, 5). Proclus seems to imply that this was proved by taking another exactly equal isosceles triangle, turing it over, and then superposing it on the first---a sort of experimental demonstration. \par 

(ii) If two straight lines cut one another, the vertically opposite angles are equal (Euc. I, 15). Thales may have regarded this as obvious, for Proclus adds that Euclid was the first to give a strict proof of it. \par 

(iii) A triangle is determined if its base and base angles be given (\textit{cf.} Euc. I, 26). Apparently this was applied to find the distance of a ship at sea---the base being a tower, and the base angles being obtained by observation. \par 

(iv) The sides of equiangular triangles are proportionals (Euc. VI, 4, or perhaps rather Euc. VI, 2). This is said to have been used by Thales when in Egypt to find the height of a pyramid. In a dialogue given by Plutarch, the speaker, addressing Thales, says, ``Placing your stick at the end of the shadow of the pyramid, you made by the sun's rays two triangles, and so proved that the [height of the] pyramid was to the [length of the] stick as the shadow of the pyramid to the shadow of the stick.'' It would seem that the theorem was unknown to the Egyptians, and we are told that the king Amasis, who was present, was astonished at this application of abstract science. \par 

(v) A circle is bisected by any diameter. This may have been enunciated by Thales, but it must have been recognized as an obvious fact from the earlier times. \par 

(vi) The angle subtended by a diameter of a circle at any point in the circumference is a right angle (Euc. III, 31). This appears to have been regarded as the most remarkable of the geometrical achievements of Thales, and it is stated that on inscribing a right-angled triangle in a circle he sacrificed an ox to the immortal gods. It has been conjectured that he may have come to this conclusion by noting that the diagonals of a rectangle are equal and bisect one another, and that therefore a rectangle can be inscribed in a circle. If so, and if he went on to apply proposition (i), he would have discovered that the sum of the angles of a right-angled triangle is equal to two right angles, a fact with which it is believed that he was acquainted. It has been remarked that the shape of the tiles used in paving floors may have suggested these results. \par 
\end{minipage}

On the whole it seems unlikely that he knew how to draw a perpendicular from a point to a line; but if he possessed this knowledge, it is possible he was also aware, as suggested by some modern commentators, that the sum of the angles of any triangle is equal to two right angles. As far as equilateral and right-angled triangles are concerned, we know from Eudemus that the first geometers proved the general property separately for three species of triangles, and it is not unlikely that they proved it thus. The area about a point can be filled by the angles of six equilateral triangles or tiles, hence the proposition is true for an equilateral triangle. Again, any two equal right-angled triangles can be placed in juxtaposition so as to form a rectangle, the sum of whose angles is four right angles; hence the proposition is true for a right-angled triangle. Lastly, any triangle can by split into the sum of two right-angled triangles by drawing a perpendicular from the biggest angle on the opposite side, and therefore again the proposition is true. The first of these proofs is evidently included in the last, but there is nothing improbable in the suggestion that the early Greek geometers continued to teach the first proposition in the form above given. \par 

Thales wrote on astronomy, and among his contemporaries was more famous as an astronomer than as a geometrician. A story runs that one night, when walking out, he was looking so intently at the stars that he tumbled into a ditch, on which an old woman exclaimed, ``How can you tell what is going on in the sky when you can't see what is lying at your own feet?''---an anecdote which was often quoted to illustrate the unpractical character of philosophers. \par 

Without going in to astronomical details, it may be mentioned that he taught that a year contained about 365 days, and not (as is said to have been previously reckoned) twelve months of thirty days each. It is said that his predecessors occasionally intercalated a month to keep the seasons in their customary places, and if so they must have realized that the year contains, on the average, more than 360 days. There is some reason to think that he believed the earth to be a disc-like body floating on water. He predicted a solar eclipse which took place at or about the time he foretold; the actual date was either May 28, 585 B.C., or September 30, 609 B.C But though this prophecy and its fulfillment gave extraordinary prestige to his teaching, and secured him the name of one of the seven sages of Greece, it is most likely that he only made use of one of the Egyptian or Chaldaean registers which stated that solar eclipses recur at intervals of about 18 years 11 days. \par 

Among the pupils of Thales were \textbf{Anaximander, Anaximenes, Mamercus,} and \textbf{Mandryatus}. Of the three mentioned last we know next to nothing. \textit{Anaximander} was born in 611 B.C., and died in 545 B.C., and succeeded Thales as head of the school at Miletus. According to Suidas he wrote a treatise on geometry in which, tradition says, he paid particular attention to the properties of spheres, and dwelt at length on the philosophical ideas involved in the conception of inginity in space and time. He constructed terrestrial and celestial globes. \par 

Anaximander is alleged to have introduced the use of the \textit{style} or \textit{gnomon} into Greece. This, in principle, consisted only of a stick stuck upright in a horizontal piece of ground. It was originally used as a sun-dial, in which case it was placed at the centre of three concentric circles, so that every two hours the end of its shadow passed from one circle to another. Such sun-dials have been found at Pompeii and Tusculum. It is said that he employed these styles to determine his meridian (presumably by marking the lines of shadow cast by the style at sunrise and sunset on the same day, and taking the plane bisecting the angle so formed); and thence, by observing the time of year when the noon-altitude of the sun was greatest and least, he got the solstices; thence, by taking half the sum of the noon-altitudes of the sun at the two solstices, he found the inclination of the equator to the horizon (which determined the altitude of the place), and, by taking half their difference, he found the inclination of the ecliptic to the equator. There seems good reason to think that he did actually determine the latitude of Sparta, but it is more doubtful whether he really made the rest of these astronomical deductions. \par 

We need not here concern ourselves further with the successors of Thales. The school he established continued to flourish till about 400 B.C., but, as time went on, its members occupied themselves more and more with philosophy and less with mathematics. We know very little of the mathematicians comprised in it, by they would seem to have devoted most of their attention to astronomy. They exercised but slight influence on the further advance of the Pythagoreans, who not only immensely developed the science of geometry, but created a science of numbers. If Thales was the first to direct general attention to geometry, it was Pythagoras, says Proclus, quoting from Eudemus, who ``changed the study of geometry into the form of a liberal eduction, for he examined its principles to the bottom and investigated its theorems in an \dots intellectual manner;'' and it is accordingly to Pythagoras that we must now direct attention. \par 

\section{The Pythagorean School.}

\textbf{Pythagoras.}\footnotemark \textit{Pythagoras} was born at Samos about 569 B.C., perhaps of Tyrian parents, and died in 500 B.C. He was thus a contemporary of Thales. The details of his life are somewhat doubtful, but the following account is, I think, substantially correct. He studied first under Pherecydes of Syros, and then under Anaximander; by the latter he was recommended to go to Thebes, and there or at Memphis he spent some years. After leaving Egypt he travelled in Asia Minor, and then settled at Samos, where he gave lectures but without much success. ABout 529 B.C. he migrated to Sicily with his mother, and with a single disciple who seems to have been the sole fruit of his labours at Samos. Thence he went to Tarentum, but very shortly moved to Croton, a Dorian colony in the south of Italy. Here the schools that he opened were crowded with enthusiastic audiences; citizens of all ranks, especially those of the upper class, attended, and even the women broke a law which forbade their going to public meetings and flocked to hear him. Amongst his most attentive auditors was Theano, the young and beautiful daughter of his host Mile, whom, in spite of the disparity of their ages, he married. She wrote a biography of her husband, but unfortunately it is lost. \par 

\footnotetext{See Loria, book I, chap. iii; Cantor, chaps. vi, vii; Allman, chap. ii; Hankel, pp. 92-111; Hoefer, \textit{Histoire des math\'ematiques}, Paris, third edition, 1886, pp. 87-130; and various papers by S. P. Tannery. For an account of Pythagoras's life, embodying the Pythagorean traditions, see the biography of Iamblichus, of which there are two or three English translations. Those who are interested in esoteric literature may like to see a modern attempt to reproduce the Pythagorean teaching in \textit{Pythagoras}, by E. Schur\'e, Eng. trans., London, 1906.}

Pythagoras divided those who attended his lectures into two classes, whom we may term probationers and Pythagoreans. The majority were probationers, but it was only to the Pythagoreans that his chief discoveries were revealed. The latter formed a brotherhood with all things in common, holding the same philosophical and political beliefs, engaged in the same pursuits, and bound by oath not to reveal the teaching or secrets of teh school; their food was simple; their discipline severe; and their mode of life arranged to encourage self-command, temperance, purity, and obedience. This strict discipline and secret organisation gave the society a temporary supremacy in the state which brought on it the hatred of various classes; and, finally, instigated by his political opponents, the mob murdered Pythagoras and many of his followers. \par 
\end{document}