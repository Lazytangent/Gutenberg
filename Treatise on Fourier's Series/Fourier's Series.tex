\documentclass[oneside,12pt]{book}

\usepackage{mathtools,amssymb,amsthm,setspace,titlesec,chngcntr,gensymb}
\usepackage[margin=1in]{geometry}
%\usepackage{fourier}

\setlength{\parskip}{6pt}
\counterwithout{equation}{chapter}

\renewcommand{\theequation}{\Roman{equation}}
\titleformat{\chapter}[display]{\Huge\bfseries\centering}{Chapter \thechapter}{0pt}{\LARGE\uppercase}
\renewcommand{\thechapter}{\Roman{chapter}}
\renewcommand{\thesection}{\arabic{section}}
\titleformat{\section}[runin]{\normalsize\bfseries}{\thesection.}{0pt}{\Large}[]
%\pagestyle{headings}

\begin{document}
    
\frontmatter

\title{An Elementary Treatise on Fourier's Series\\and Spherical, Cylindrical, and Ellipsoidal Harmonics,\\with Applications to Problems in Mathematical Physics}
\author{William Elwood Byerly, Ph.D.,\\Professor of Mathematics in Harvard University}
\date{1893}
\maketitle

\chapter{Preface}
About ten years ago I gave a course of lectures on Trigonometric Series, following closely the treatment of that subject in Riemann's ``Partielle Differentialgleichungen,'' to accompany a short course on The Potential Function, given by Professor B. O. Peirce. \par 

My course has been gradually modified and extended until it has become an introduction to Spherical Harmonics and Bessel's and Lam\'e's Functions. \par 

Two years ago my lecture notes were lithographed by my class for their own use and were found so convenient that I have prepared them for publication, hoping that they may prove useful to others as well as to my own students. Meanwhile, Professor Peirce has published his lectures on ``The Newtonian Potential Function'' (Boston, Ginn \& Co.), and the two sets of lectures form a course (Math. 10) given regularly at Harvard, and intended as a partial introduction to modern Mathematical Physics. \par 

Students taking this course are supposed to be familiar with so much of the infinitesimal calculus as is contained in my ``Differential Calculus'' (Boston, Ginn \& Co.) and my ``Integral Calculus'' (second edition, same publishers), to which I refer in the present book as ``Dif. Cal.'' and ``Int. Cal.'' Here, as in the ``Calculus,'' I speak of a ``derivative'' rather than a ``differential coefficient,'' and use the notation $D_x$ instead of $\frac{\partial}{\partial x}$ for ``partial derivative with respect to $x$.'' \par 

The course was at first, as I have said, an exposition of Riemann's ``Partielle Differentialgleichungen.'' In extending it, I drew largely from Ferrer's ``Spherical Harmonics'' and Heine's ``Kugelfunctionen.'' and was somewhat indebted to Todhunter (``Functions of Laplace, Bessel, and Lam\'e''), Lord Rayleigh (``Theory of Sound''), and Forsyth (``Differential Equations''). \par 

In preparing the notes for publication, I have been greatly aided by the criticisms and suggestions of my colleagues, Professor B. O. Peirce and Dr. Maxime B\^ocher, and the latter has kindly contributed the brief historical sketch contained in Chapter IX. \par 

\begin{flushright}
    W. E. BYERLY.
\end{flushright}
\textsc{Cambridge, Mass.,} Sept. 1893.

\chapter{Analytical Table of Contents}
\begin{center}
    \uppercase{Chapter I.}
\end{center}
\begin{flushright}
    \textsc{pages}
\end{flushright}
\vspace{-0.6cm}
\textsc{Introduction\hfill 1--29} \par 
\textsc{Art. 1.} List of some important homogeneous linear partial differential equations of Physics.---\textsc{Arts. 2--4.} Distinction between the general solution and a particular solution of a differential equation. Need of additional data to make the solution of a differential equation determinate. Definition of linear and of linear and homogeneous.---\textsc{Arts. 5-6.} Particular solutions of homogeneous linear differential equations may be combined into a more general solution. Need of development in terms of normal forms.---\textsc{Art. 7.} Problem: Permanent state of temperatures in a thin rectangular plate. Need of a development in sine series. Example.---\textsc{Art. 8.} Problem: Transverse vibrations of a stretched elastic string. A development in sine series suggested. Example.---\textsc{Art. 9.} Problem: Potential function due to the attraction of a circular ring of small cross-section. Surface Zonal Harmonics (Legendre's Coefficients). Example.---\textsc{Art. 10.} Problem: Permanent state of temperatures in a solid sphere. Development in terms of Surface Zonal Harmonics suggested.---\textsc{Arts. 11-12}. Problem: Vibrations of a circular drumhead. Cylindrical Harmonics (Bessel's Functions). Recapitulation.---\textsc{Art. 13.} Method of making the solution of a linear partial differential equation depend upon solving a set of ordinary differential equations by assuming the dependent variable equal to a product of factors each of which involves but one of the independent variables. \textsc{Arts. 14-15} Method of solving ordinary homogeneous linear differential equations by development in power series. Applications.---\textsc{Art. 16.} Application to Legendre's Equation. Several forms of general solution obtained. Zonal Harmonics of the second kind.---\textsc{Art. 17.} Application to Bessel's Equation. General solution obtained for the case where $m$ is not an integer, and for the case where $m$ is zero. Bessel's Function of the second kind and zeroth order.---\textsc{Art. 18.} Method of obtaining the general solution of an ordinary linear differential equation of the second order from a given particular solution. Applications to the equations considered in Arts. 14-17. \par 

\begin{center}
    \uppercase{Chapter II.}
\end{center}
\textsc{Development in Trigonometric Series\hfill 30-55} \par 

\textsc{Arts. 19-22.} Determination of the coefficients of $n$ terms of a sine series so that the sum of the terms shall be equal to a given function of $x$ for $n$ given values of $x$. Numerical example.---

\mainmatter
\chapter{Introduction}
%\textbf{1.}
\section{}
In many important problems in mathematical physics we are obliged to deal with \textit{partial differential equations} of a comparatively simple form. \par

For example, in the Analytical Theory of Heat we have for the change of temperature of any solid due to the flow of heat within the solid, the equation 
\begin{equation}
    \label{eq1}
    D_tu=a^2(D_x^2u+D_y^2u+D_z^2u),\footnote{For the sake of brevity we shall often use the symbol $\nabla^2$ for the operation $D_x^2+D_y^2+D_z^2$; and with this notation equation \refeq{eq1} would be written $D_tu=a^2\nabla^2u$.}
\end{equation}
where $u$ represents the temperature at any point of the solid and $t$ the time. \par 

In the simplest case, that of a slab of infinite extent with parallel plane faces, where the temperature can be regarded as a function of one coordinate, \eqref{eq1} reduces to 
\begin{equation}
    \label{eq2}
    D_tu=a^2D_x^2u,
\end{equation}
a form of considerable importance in the consideration of the problem of the cooling of the earth's crust. \par 

In the problem of the permanent state of temperatures in a thin rectangular plate, the equation \eqref{eq1} becomes 
\begin{equation}
    D_x^2+D_y^2=0.
    \label{eq3}
\end{equation} \par 

In \textit{polar} or \textit{spherical coordinates} \eqref{eq1} is less simple, it is 
\begin{equation}
    D_tu=\frac{a^2}{r^2}\left[D_r(r^2D_ru)+\frac{1}{\sin\theta}D_\theta(\sin\theta D_\theta u)+\frac{1}{\sin^2\theta}D_\phi^2u\right].
    \label{eq4}
\end{equation} \par 
In the case where the solid in question is a sphere and the temperature at any point depends merely on the distance of the point from the centre \refeq{eq4} reduces to 
\begin{equation}
    \label{eq5}
    D_t(ru)=a^2D_r^2(ru). 
\end{equation} \par 

In \textit{cylindrical coordinates} \eqref{eq1} becomes 
\begin{equation}
    \label{eq6}
    D_tu=a^2[D_r^2u+\frac{1}{r}D_ru+\frac{1}{r^2}D_\phi^2u+D_z^2u].
\end{equation} \par 

In considering the flow of heat in a cylinder when the temperature at any point depends merely on the distance $r$ of the point from the axis \eqref{eq6} becomes 
\begin{equation}
    \label{eq7}
    D_tu=a^2(D_r^2u+\frac{1}{r}D_ru).
\end{equation} \par 
In Acoustics in several problems we have the equation
\begin{equation}
    \label{eq8}
    D_t^2y=a^2D_x^2y;
\end{equation}
for instance, in considering the transverse or the longitudinal vibrations of a stretched elastic string, or the transmission of plane sound waves through the air. \par 

If in considering the transverse vibrations of a stretched string we take account of the resistance of the air \eqref{eq8} is replaced by 
\begin{equation}
    \label{eq9}
    D_t^2y+2kD_ty=a^2D_x^2y. 
\end{equation} \par 

In dealing with the vibrations of a stretched elastic membrane, we have the equation 
\begin{equation}
    \label{eq10}
    D_t^2z=c^2(D_x^2z+D_y^2z),
\end{equation}
or in \textit{cylindrical coordinates}
\begin{equation}
    \label{eq11}
    D_t^2z=c^2(D_r^2z+\frac{1}{r}D_rz+\frac{1}{r^2}D_\phi^2z).
\end{equation} \par 

In the theory of \textit{Potential} we constantly meet Laplace's Equation 
\begin{equation}
    \label{eq12}
    D_x^2V+D_y^2V+D_z^2V=0
\end{equation}
or $$\nabla^2V=0$$ \\ 
which in \textit{spherical coordinates} becomes 
\begin{equation}
    \label{eq13}
    \frac{1}{r^2}\left[rD_r^2(rV)+\frac{1}{\sin\theta}D_\theta(\sin\theta D_\theta V)+\frac{1}{\sin^2\theta}D_\phi^2V=0\right],
\end{equation}
and in \textit{cylindrical coordinates}
\begin{equation}
    \label{eq14}
    D_r^2V+\frac{1}{r}D_rV+\frac{1}{r^2}D_\phi^2V+D_z^2V=0.
\end{equation} \par 

In \textit{curvilinear coordinates} it is 
\begin{equation}
    \label{eq15}
    h_1h_2h_3\left[D_{\rho_1}\left(\frac{h_1}{h_2h_3}D_{\rho_1}V\right)+D_{\rho_2}\left(\frac{h_2}{h_3h_1}D_{\rho_2}V\right)+D_{\rho_3}\left(\frac{h_3}{h_1h_2}D_{\rho_3}V\right)\right]=0;
\end{equation}
where $$f_1(x,y,z)=\rho_1,\ f_2(x,y,z)=\rho_2,\ f_3(x,y,z)=\rho_3$$ 
represent a set of surfaces which cut one another at right angles, no matter what values are given to $\rho_1,\ \rho_2,$ and $\rho_3$; and where 
\begin{align*}
    h_1^2&=(D_x\rho_1)^2+(D_y\rho_1)^2+(D_z\rho_1)^2 \\
    h_2^2&=(D_x\rho_2)^2+(D_y\rho_2)^2+(D_z\rho_2)^2 \\
    h_3^2&=(D_x\rho_3)^2+(D_y\rho_3)^2+(D_z\rho_3)^2,
\end{align*}
and, of course, must be expressed in terms of $\rho_1,\ \rho_2$, and $\rho_3$. \par 

If if happens that $\nabla^2\rho_1=0,\nabla^2\rho_2=0,\text{ and } \nabla^2\rho_3=0$, then Laplace's Equation \eqref{eq15} assumes the very simple form 
\begin{equation}
    h_1^2D_{\rho_1}^2V+h_2^2D_{\rho_2}^2V+h_3^2D_{\rho_3}^2V=0.
\end{equation} \par 

\section{} A \textit{differential equation} is an equation containing derivatives or differentials with or without the primitive variables from which they are derived. \par 

The \textit{general solution} of a differential equation is the equation expressing the most general relation between the primitive variables which is consistent with the given differential equation and which does not involved differentials or derivatives. A general solution will always contain arbitrary (\textit{i.e.}, undetermined) \textit{constants \emph{or} arbitrary functions.} \par 

A \textit{particular solution} of a differential equation is a relation between the primitive variables which is consistent with the given differential equation, but which is less general than the general solution, although included in it. \par 

Theoretically, every particular solution can be obtained from the general solution by substituting in the general solution particular values for the arbitrary constants or particular functions for the arbitrary functions; but in practice it is often easy to obtain particular solutions directly from the differential equation when it would be difficult or impossible to obtain the general solution. \par 

\section{} If a problem requiring for its solution the solving of a differential equation is \textit{determinate}, there must always be given in addition to the differential equation enough outside conditions for the determination of all the arbitrary constants or arbitrary functions that enter into the general solution of the equation; and in dealing with such a problem, if the differential equation can be readily solved the natural method of procedure is to obtain its general solution, and then to determine the constants or functions by the aid of the given conditions. \par 

It often happens, however, that the general solution of the differential equation in question cannot be obtained, and then, since hte problem \textit{if determinate} will be solved if by any means a solution of the equation can be found which will also satisfy the given outside conditions, it is worth while to try to get \textit{particular solutions} and so to combine them as to form a result which shall satisfy the given conditions without ceasing to satisfy the differential equation. \par 

\section{} A differential equation is \textit{linear} when it would be of the first degree if the dependent variable and all its derivatives were regarded as algebraic unknown quantities. If it is linear and contains no term which does not involve the dependent variable or one of its derivatives, it is said to be linear and \textit{homogeneous}. \par 

All the differential equations collected in Art. 1 are linear and homogeneous. \par 

\section{} \textit{if a value of the dependent variable has been found which satisfies a given homogeneous, linear, differential equation, the product formed by multiplying this value by any constant will also be a value of the dependent variable which will satisfy the equation.} \par 

For if all the terms of the given equation are transposed to the first member, the substitution of the first-named value must reduce that member to zero; substituting the second value is equivalent to multiplying each term of the result of the first substitution by the same constant factor, which therefore may be taken out as a factor of the whole first number. The remaining factor being zero, the product is zero and the equation is satisfied. \par 

\textit{If several values of the dependent variable have been found each of which satisfies the given differential equation, their sum will satisfy the equation}; for it the sum of the values in question is substituted in the equation each term of the sum will give rise to a set of terms which must be equal to zero, and therefore the sum of these sets must be zero. \par 

\section{} It is generally possible to get by some simple device \textit{particular solutions} of such differential equations as those we have collected in Art. 1. The object of the branch of mathematics with which we are about to deal is to find methods of so combining these particular solutions as to satisfy any given conditions which are consistent with the nature of the problem in question. \par 

This often requires us to be able to develop any given functions of the variables which enter into the expression of these conditions in terms of \textit{normal forms} suited to the problem with which we happen to be dealing, and suggested by the form of particular solution that we are able to obtain for the differential equation. \par 

These normal forms are frequently sines and cosines, but they are often much more complicated functions known as \textit{Legendre's Coefficients, \emph{or} Zonal Harmonics; Laplace's Coefficients, \emph{or} Spherical Harmonics: Bessel's Functions, \emph{or} Cylindrical Harmonics; Lam\'e's Functions, \emph{or} Ellipsoidal Harmonics}, \&c. \par 

\section{} As an illustration, let us take Fourier's problem of the permanent state of temperatures in a thin rectangular plate of breadth $\pi$ and of infinite length whose faces are impervious to heat. We shall suppose that the two long edges of the plate are kept at the constant temperature zero, that one of the short edges, which we shall call the base of the plate, is kept at the temperature unity, and that the temperatures of points in the plate decrease indefinitely as we recede from the base; we shall attempt to find the temperature at any point of the plate. \par 

Let us take the base as the axis of $X$ and one end of the base as the origin. Then to solve the problem we are to find the temperature $u$ of any point from the equation
\begin{equation}
    D_x^2u+D_y^2u=0
    \tag{III Art. 1}
    \label{eq3Art}
\end{equation}
subject to the conditions 
\renewcommand{\theequation}{\arabic{equation}}
\setcounter{equation}{0}
\begin{eqnarray}
    u=0 & \text{ when } & x=0 \label{con1}\\
    u=0 & \text{ `` } & x=\pi \label{con2}\\
    u=0 & \text{ `` } & y=\infty \label{con3}\\
    u=1 & \text{ `` } & y=0. \label{con4}
\end{eqnarray} \par 

We shall begin by getting a particular solution of \eqref{eq3Art}, and we shall use a device which always succeeds when the equation is \textit{linear \emph{and} homogeneous \emph{and has} constant coefficients.} \par 

Assume\footnote{This assumption mst be regarded as purely tentative. It must be tested by substituting in the equation, and is justified if it leads to a solution.\label{note2}} $u=e^{\alpha y+\beta x}$, where $\alpha$ and $\beta$ are constants, substitute in \eqref{eq3Art} and divide by $e^{\alpha y+\beta x}$, and we have $\alpha^2+\beta^2=0$. If, then, this condition is satisfied $u=e^{\alpha y+\beta x}$ is a solution. \par 

Hence $u=e^{\alpha y\pm\beta xi}$\footnote{We shall regularly use the symbol $i$ for $\sqrt{-1}$.} is a solution of \eqref{eq3Art}, no matter what value may be given to $\alpha$. \par 

This form is objectionable, since it involves an imaginary. We can, however, readily improve it. \par 

Take $u=e^{\alpha y}e^{\alpha xi}$, a solution of \eqref{eq3Art}, and $u=e^{\alpha y}e^{-\alpha xi}$, another solution of \eqref{eq3Art}; add these values of $u$ and divide the sum by 2 and we have $a^{\alpha y}\cos\alpha x$. (v. Int. Cal. Art. 35, [1].) Therefore by Art. 5
\begin{equation}
    u=e^{\alpha y}\cos\alpha x
    \label{equ5}
\end{equation}
is a solution of \eqref{eq3Art}. Take $u=e^{\alpha y}e^{\alpha xi}$ and $u=e^{\alpha y}e^{-\alpha xi}$, subtract the second value of $u$ from the first and divide by $2i$ and we have $e^{\alpha y}\sin\alpha x$. (v. Int. Cal. Art. 35, [2]). Therefore by Art. 5
\begin{equation}
    u=e^{\alpha y}\sin\alpha x
    \label{equ6}
\end{equation}
is a solution of \eqref{eq3Art}. \par 

Let us now see if out of these particular solutions we can build up a solution which will satisfy the conditions \eqref{con1}, \eqref{con2}, \eqref{con3}, and \eqref{con4}. \par 

Consider \hfill $u=e^{\alpha y}\sin\alpha x.$ \hspace{6.1cm} \eqref{equ6} \\
It is zero when $x=0$ for all values of $\alpha$. It is zero when $x=\pi$ if $\alpha$ is a whole number. It is zero when $y=\infty$ is negative. If, then, we write $u$ equal to a sum of terms of the form $Ae^{-my}\sin mx$, where $m$ is a positive integer, we shall have a solution of \eqref{eq3Art} which satisfies conditions \eqref{con1}, \eqref{con2}, and \eqref{con3}. Let this solution be 
\begin{equation}
    u=A_1e^{-y}\sin x+A_2e^{-2y}\sin 2x+A_3e^{-3y}\sin 3x+A_4e^{-4y}\sin 4x+\dots 
    \label{equ7}
\end{equation}
$A_1,\ A_2,\ A_3,\ A_4,$ \&c., being undetermined constants. \par 

When $y=0$ \eqref{equ7} reduces to 
\begin{equation}
    u=A_1\sin x+A_2\sin 2x+A_3\sin 3x+A_4\sin 4x+\dots.
    \label{equ8}
\end{equation} \par 

If now it is possible to develop unity into a series of the form \eqref{equ8}, our problem is solved; we have only to substitute the coefficients of that series for $A_1,\ A_2,\ A_3$. and \&c. in \eqref{equ7}. \par 

It will be proved later that 
\begin{equation*}
    1=\frac{4}{\pi}\left(\sin x + \frac{1}{3}\sin 3x+ \frac{1}{5}\sin 5x+\frac{1}{7}\sin 7x+\dots\right)
\end{equation*}
for all values of $x$ between 0 and $\pi$; hence our required solution is 
\begin{equation}
    u=\frac{4}{\pi}\left[e^{-y}\sin x+\frac{1}{3}e^{-3y}\sin 3x+\frac{1}{5}e^{-5y}\sin 5x+\frac{1}{7}e^{-7y}\sin 7x+\dots\right]
    \label{equ9}
\end{equation}
for this satisfies the differential equation and all the given conditions. \par 

If the given temperature of the base of the plate instead of being unity is a function of $x$, we can solve the problem as before if we can express the given function of $x$ as a sum of terms of the form $A\sin mx$, where $m$ is a whole number. \par 

The problem of finding the value of the \textit{potential function} at any point of a long, thin, rectangular conducting sheet, of breadth $\pi$, through which an electric current is flowing, when the two long edges are kept at potential zero, and one short edge at potential unity, is mathematically identical with the problem we have just solved. \par 
\begin{center}
    \textsc{Example.}
\end{center} \par
Taking the temperature of the base of the plate described above as 100\textdegree centigrade, and that of the sides of the plate 0\textdegree, compute the temperatures of the points 
$$(a)\ (\frac{\pi}{6},1);\ (b)\ (\frac{\pi}{3},2);\ (c)\ (\frac{\pi}{2},3),$$
correct to the nearest degree. \hfill \textit{Ans. (a) 26\textdegree;(b) 15\textdegree; (c) 6\textdegree.} \par 

\section{} As another illustration, we shall take the problem of the transverse vibrations of a stretched string fastened at the ends, initially distorted into some given curve and then allowed to swing. \par 

Let the length of the string be $l$. Take the position of equilibrium of the string as the axis of X, and one of the ends as the origin, and suppose the string initially distorted into a curve whose equation $y=f(x)$ is given. \par 

We have then to find an expression for $y$ which will be a solution of the equation 
\begin{equation}
    D_t^2y=a^2D_x^2y
    \tag{VIII Art. 1,}
    \label{eq8Art}
\end{equation}
while satisfying the conditions 
\setcounter{equation}{0}
\begin{eqnarray}
    y=0 & \text{when} & x=0 \label{con8.1}\\
    y=0 & \text{``} & x=l \label{con8.2}\\
    y=f(x) & \text{``} & t=0 \label{con8.3}\\
    D_ty=0 & \text{``} & t=0 \label{con8.4}, 
\end{eqnarray}
the last condition meaning merely that the string starts from rest. \par 

As in the last problem let\footnote{See note on page \pageref{note2}} $y=e^{\alpha x+\beta t}$ and substitute in \eqref{eq8Art}. Divide by $^{\alpha x+\beta t}$ and we have $\beta^2=a^2\alpha^2$ as the condition that our assumed value of $y$ shall satisfy the equation. 
\begin{equation}
    y=e^{\alpha x \pm a\alpha t}
    \label{equ8.5}
\end{equation}
is, then, a solution of \eqref{eq8Art} whatever the value of $\alpha$. \par 

It is more convenient to have a trigonometric than an exponential form to deal with, and we can readily obtain one by using an imaginary value for $\alpha$ in \eqref{equ8.5}. Replace $\alpha$ by $-\alpha i$ and \eqref{equ8.5} becomes $y=e^{-(x\pm at)\alpha i}$, another solution of \eqref{eq8Art}. Add these values of $y$ and divide by 2 and we have $\cos\alpha(x\pm at)$. Subtract the second value of $y$ from the first and divide by $2i$ and we have $\sin\alpha(x\pm at)$. 
\begin{align*}
    y&=\cos\alpha(x+at)\\
    y&=\cos\alpha(x-at)\\
    y&=\sin\alpha(x+at)\\
    y&=\sin\alpha(x-at)
\end{align*}
and, then, solutions of \eqref{eq8Art}. Writing $y$ successivly equal to half the sum of the first pair of values, helf their difference, half the sum of the last pair of values, and half their difference, we get the very convenient particular solutions of \eqref{eq8Art}. 
\begin{align*}
    y&=\cos\alpha x\cos\alpha at \\
    y&=\sin\alpha x\sin\alpha at \\
    y&=\sin\alpha x\cos\alpha at \\
    y&=\cos\alpha x\sin\alpha at.
\end{align*} \par 

If we take the third form 
$$y=\sin\alpha x\cos\alpha at$$ 
it will satisfy conditions \eqref{con8.1} and \eqref{con8.4}, no matter what value may be given to $\alpha$, and it will satisfy \eqref{con8.2} if $\alpha=\frac{m\pi}{l}$ where $m$ is an integer. \par 

If then we take 
\begin{equation}
    y=A_1\sin\frac{\pi x}{l}\cos\frac{\pi at}{l}+A_2\sin\frac{2\pi x}{l}\cos\frac{2\pi at}{l}+A_3\sin\frac{3\pi x}{l}\cos\frac{3\pi at}{l}+\dots
    \label{equ8.6}
\end{equation}
where $A_1$, $A_2$, $A_3$ \dots are undetermined constsants, we shall have a solution of \eqref{eq8Art} which satisfies \eqref{con8.1}, \eqref{con8.2}, and \eqref{con8.4}. When $t=0$ it reduces to 
\begin{equation}
    y=A_1\sin\frac{\pi x}{l}+A_2\sin\frac{2\pi x}{l}+A_3\sin\frac{3\pi x}{l} +\dots 
    \label{equ8.7}
\end{equation} \par 

If now it is possible to develop $f(x)$ into a series of the form \eqref{equ8.7}, we can solve our problem completely. We have only to take the coefficients of this series as values of $A_1$, $A_2$, $A_3$ \dots in \eqref{equ8.6}, and we shall have a solution of \eqref{eq8Art} which satisfies all our given conditions. \par 

In each of the preceding problems the \textit{normal function}, in terms of which a given function has to be expressed, is the sine of a simple multiple of the variable. It would be easy to modify the problem so that the \textit{normal form} shold be a cosine. \par 

We shall now take a couple of problems which are more complicated and where the normal function is an unfamiliar one. \par 

\section{} Let it be required to find the potential function due to a circular wire ring of small cross section and of given radius $c$, supposing the matter of the ring to attract according to the law of nature. \par 

We can readily find, by direct integration, the value of the potential function at any point of the axis of the ring. We get for it 
\setcounter{equation}{0}
\begin{equation}
    V=\frac{M}{\sqrt{c^2+x^2}}
    \label{eq9.1}
\end{equation}
where $M$ is the mass of the ring, and $x$ the distance of the point from the centre of the ring. \par 

Let us use spherical coordinates, taking the centre of the ring as origin and the axis of the ring as the polar axis. \par 

To obtain the value of the potential function at any point in space, we must satisfy the equation
\begin{equation}
    \label{equ8Art}
    \tag{XIII Art. 1}
    rD_r^2(rV)+\frac{1}{\sin\theta}D_\theta(\sin\theta D_\theta V)+\frac{1}{\sin^2\theta}D_\phi^2 V=0,
\end{equation}
subject to the condition
\setcounter{equation}{0}
\begin{equation}
    V=\frac{M}{(c^2+r^2)^{\frac{1}{2}}} \quad \text{when}\quad\theta=0.
\end{equation} \par 

From the symmetry of the ring, it is clear that the value of the potential function must be independent of $\phi$, so that \eqref{equ8Art} will reduce to 
\begin{equation}
    rD_r^2(rV)+\frac{q}{\sin\theta}D_\theta(\sin\theta D_\theta V)=0.
    \label{eq9.2}
\end{equation} \par 

We must now try to get particular solutions of \eqref{eq9.2}, and as the coefficients are not constant, we are driven to a new device. \par 

Let\footnote{See note on page \pageref{note2}.} $V=r^mP$, where $P$ is a function of $\theta$ only, and $m$ is a positive integer, and substitute in \eqref{eq2}, which becomes 
$$m(m+1)r^mP+\frac{r^m}{\sin\theta}D_\theta(\sin\theta D_\theta P)=0.$$
Divide by $r^m$ and use the notation of ordinary derivatives since $P$ depends upon $\theta$ only, and we have the equation 
\begin{equation}
    m(m+1)P+\frac{1}{\sin\theta}\frac{d\left(\sin\theta\frac{dP}{d\theta}\right)}{d\theta}=0,
    \label{eq9.3}
\end{equation}
from which to obtain $P$. \par 

Equation \eqref{eq9.3} can be simplified by changing the independent variable. Let $x=\cos\theta$ and \eqref{eq9.3} becomes 
\begin{equation}
    \frac{d}{dx}\left[(1-x^2)\frac{dP}{dx}\right]+m(m+1)P=0.
    \label{eq9.4}
\end{equation} \par 

Assume\footnote{See note on page \pageref{note2}} now that $P$ can be expressed as a sum or a series of terms involving whole powers of $x$ multiplied by constant coefficients. \par 

Let $P=\sum a_nx^n$ and substitute this value of $P$ in \eqref{eq9.4}. We get 
\begin{equation}
    \sum[n(n-1)a_nx^{n-2}-n(n+1)a_nx^n+m(m+1)a_nx^n]=0,
    \label{eq9.5}
\end{equation}
where the symbol $\sum$ indicates that we are to form all the terms we can by taking successive whole numbers for $n$. \par 

As \eqref{eq9.5} must be true no matter what the value of $x$, the coefficient of any given power of $x$, as for instance $x^k$, must vanish. Hence 
\begin{equation}
    (k+2)(k+1)a_{k+2}-k(k+1)a_k+m(m+1)a_k=0
    \label{eq9.6}
\end{equation}
and 
\begin{equation}
    a_{k+2}=-\frac{m(m+1)-k(k+1)}{(k+1)(k+2)}a_k.
    \label{eq9.7}
\end{equation}
If now any set of coefficients satisfying the relation \eqref{eq9.7} be taken, $P=\sum a_kx^k$ will be a solution of \eqref{eq9.4}. \par 

If \hspace{4cm} $k=m,\quad a_{k+2}=0,\quad a_{k+4}=0,\quad \&c.$ \\
Since it will answer our purpose if we pick out the simplest set of coefficients that will obey the condition \eqref{eq9.7}, we can take a set including $a_m$. \par 

Let us rewrite \eqref{eq9.7} in the form 
\begin{equation}
    a_k=-\frac{(k+2)(k+1)}{(m-k)(m+k+1)}a_{k+2}.
    \label{eq9.8}
\end{equation}
We get from \eqref{eq9.8}, beginning with $k=m-2$, 
\begin{align*}
    a_{m-2}&=-\frac{m(m-1)}{2\cdot(2m-1)}a_{k+2}. \\
    a_{m-4}&=-\frac{m(m-1)(m-2)(m-3)}{2\cdot4\cdot(2m-1)(2m-3)}a_m \\
    a_{m-6}&=-\frac{m(m-1)(m-2)(m-3)(m-4)(m-5)}{2\cdot4\cdot6\cdot(2m-1)(2m-3)(2m-5)}a_m,\quad \text{\&c.}
\end{align*}
If $m$ is even we see that the set will end with $a_0$, if $m$ is odd, with $a_1$. 
$$P=a_m\left[x^m-\frac{m(m-1)}{2\cdot(2m-1)}x^{m-2}+\frac{m(m-1)(m-2)(m-3)}{2\cdot4\cdot(2m-1)(2m-3)}x^{m-4}-\dots\right]$$
where $a_m$ is entirely arbitrary, is, then, a solution of \eqref{eq9.4}. It is found convenient to take $a_m$ equal to 
$$\frac{(2m-1)(2m-3)\dots1}{m!}$$
and it can be shown that with this value of $a_m\ P=1$ when $x=1$. \par 

$P$ is a function of $x$ and contains no higher powers of $x$ than $x^m$. It is usual to write it as $P_m(x)$. \par 

We proceed to compute a few values of $P_m(x)$ from the formula 
\begin{equation}
    \begin{split}
    P_m(x)&=\frac{(2m-1)(2m-3)\dots1}{m!}\biggl[x^m-\frac{m(m-1)}{2\cdot(2m-1)}x^{m-2} \\
    &+\frac{m(m-1)(m-2)(m-3)}{2\cdot4\cdot(2m-1)(2m-3)}x^{m-4}-\dots\biggr].
    \label{eq9.9}
    \end{split}
\end{equation} \par 

We have:
\begin{align*}
    P_0(x)&=1 & &\text{or} & P_0(\cos\theta)&=1 \\
    P_1(x)&=x & &\text{``} & P_1(\cos\theta)&=\cos\theta \\
    P_2(x)&=\frac{1}{2}(3x^2-1) & &\text{``} & P_2(\cos\theta)&=\frac{1}{2}(3\cos^2\theta-1) \\
    P_3(x)&=\frac{1}{2}(5x^3-3x) & &\text{``} & P_3(\cos\theta)&=\frac{1}{2}(5\cos^3\theta-3\cos\theta) \\
    P_4(x)&=\frac{1}{8}(35x^4-30x^2+3) & &\text{or} & P_4(\cos\theta)&=\frac{1}{8}(35\cos^4\theta-30\cos^2\theta+3) \\
    P_5(x)&=\frac{1}{8}(63x^5-70x^3+15x) & &\text{or} & P_5(\cos\theta)&=\frac{1}{8}(63\cos^5\theta-70\cos^3\theta+15\cos\theta).
    \tag{10} \label{eq9.10}
\end{align*}   \par %Figure out the large right bracket formatting here to group the equations as part of the (10). 

We have obtained $P=P_m(x)$ as a particular solution of \eqref{eq9.4} and $P=P_m(\cos\theta)$ as a particular solution of \eqref{eq9.3}. $P_m(x)$ or $P_m(\cos\theta)$ is a new function, known as a \textit{Legendre's Coefficient}, or as a \textit{Surface Zonal Harmonic}, and occurs as a normal form in many important problems. \par 

$V=r^mP_m(\cos\theta)$ is a particular solution of \eqref{eq9.2} and $r^mP_m(\cos\theta)$ is sometimes called a \textit{Solid Zonal Harmonic}. \par 

We can now proceed to the solution of our original problem. 
\setcounter{equation}{10}
\begin{equation}
    V=A_0r^0P_0(\cos\theta)+A_1r^1P_1(\cos\theta)+A_2r^2P_2(\cos\theta)+A_3r^3P_3(\cos\theta)+\dots 
    \label{eq9.11}
\end{equation}
where $A_0,\ A_1,\ A_2$, \&c., are entirely arbitrary, is a solution of \eqref{eq9.2} (v. Art. 5). When $\theta=0$ \eqref{eq9.11} reduces to 
$$V=A_0+A_1r+A_2r^2+A_3r^3+\dots,$$
since, as we have said, $P_m(x)=1$ when $x=1$, or $P_m(\cos\theta)=1$ when $\theta=0$. \par 

By our condition \eqref{con1}
$$V=\frac{M}{(c^2+r^2)^{\frac{1}{2}}}$$
when $\theta=0$. \par

By the Binomial Theorem 
\begin{equation*}
    \frac{M}{(c^2+r^2)^{\frac{1}{2}}}=\frac{M}{c}\left[1-\frac{1}{2}\frac{r^2}{c^2}+\frac{1\cdot3}{2\cdot4}\frac{r^4}{c^4}-\frac{1\cdot3\cdot5}{2\cdot4\cdot6}\frac{r^6}{c^6}+\dots\right]
\end{equation*}
provided $r<c$. Hence 
\begin{multline}
    V=\frac{M}{c}\left[P_\theta(\cos\theta)-\frac{1}{2}\frac{r^2}{c^2}P_2(\cos\theta)+\frac{1\cdot3}{2\cdot4}\frac{r^4}{c^4}P_4\linebreak(\cos\theta)-\frac{1\cdot3\cdot5}{2\cdot4\cdot6}\frac{r^6}{c^6}P_6(\cos\theta)+\dots\right]
    \label{eq9.12}
\end{multline}
is our required solution if $r<c$; for it is a solution of equation \eqref{eq9.2} and satisfies condition \eqref{con1}. \par 

\begin{center}
    \textsc{Example.}
\end{center}
Taking the mass of the ring as one pound and the radius of the ring as one foot, compute to two decimal places the value of the potential function due to the ring at the points 
\begin{align*}
    (a)\ (r&=.2,\ \theta=0); & (d)\ (r&=.6,\ \theta=0); & (f)\ (r&=.6,\ \theta=\frac{\pi}{3}); \\ 
    (b)\ (r&=.2,\ \theta=\frac{\pi}{4}); & (e)\ (r&=.6,\ \theta=\frac{\pi}{6}); & (g)\ (r&=.6,\ \theta=\frac{\pi}{2}); \\
    (c)\ (r&=.2,\ \theta=\frac{\pi}{2}); 
\end{align*} \textit{Ans. (a) \emph{.98}; (b) \emph{.99}; (c) \emph{1.01}; (d) \emph{.86}; (e) \emph{.90}; (f) \emph{1.00}; (g) \emph{1.10}.} \\  
The unit used is the potential due to a pound of mass concentrated at a point and attracting a second pound of mass concentrated at a point, the two points being a foot apart. \par 

\section{} A slightly different problem calling for development in terms of Zonal Harmonics is the following: \par 

Required the permanent temperatures within a solid sphere of radius 1, one half of the surface being kept at the constant temperature zero, and the other half at the constant temperature unity. \par 

Let us take the diameter perpendicular to the plane separating the unequally heated surfaces as our axis and let use use spherical coordinates. As in he last problem, we must solve the equation 

\begin{equation*}
    rD_r^2(ru)+\frac{1}{\sin\theta}D_\theta(\sin\theta D_\theta u)+\frac{1}{\sin^2\theta}D_\phi^2u=0
    \tag[XIII]{XIII Art. 1}
\end{equation*}
which as before reduces to 
\setcounter{equation}{0}
\begin{equation}
    rD_r^2(ru)+\frac{1}{\sin\theta}D_\theta(\sin\theta D_\theta u)=0
    \label{eq10.1}
\end{equation}
from the consideration that the temperatures must be independent of $\phi$. \par 

Our equation of condition is 
\begin{equation}
    u=1\text{ from }\theta=0\text{ to }\theta=\frac{\pi}{2}\text{ and }u=0\text{ from }\theta=\frac{\pi}{2}\text{ to }\theta=\pi,
    \label{eq10.2}
\end{equation}
when $r=1$. \par 

As we have seen $u=r^mP_m(\cos\theta)$ is a particular solution of \eqref{eq10.1}, $m$ being any positive whole number, and 
\begin{equation}
    u=A_0r^0P_0(\cos\theta)+A_1r^1P_1(\cos\theta)+A_2r^2P_2(\cos\theta)+A_3r^3P_3(\cos\theta)+\dots 
    \label{eq10.3}
\end{equation}
where $A_0$, $A_1$, $A_2$, $A_3$ \dots are undetermined constants, is a solution of \eqref{eq10.1}. \par 

When $r=1$ \eqref{eq10.3} reduces to 
\begin{equation}
    u=A_0P_0(\cos\theta)+A_1P_1(\cos\theta)+A_2P_2(\cos\theta)+A_3P_3(\cos\theta)+\dots 
    \label{eq10.4}
\end{equation}
If then we can develop our function of $\theta$ which enters into equation \eqref{eq10.2} in a series of the form \eqref{eq10.4}, we have only to take the coefficients of that series as the values of $A_0,\: A_1,\: A_2$, \&c., in \eqref{eq10.3} and we shall have our required solution. \par 

\setcounter{equation}{0}
\section{} As a last example we shall take the problem of the vibration of a stretched circular membrane fastened at the circumference, that is, of an ordinary drumhead. We shall suppose the membrane initially distorted into any given form which has circular symmetry\footnote{A function of the coordinates of a point has \textit{circular symmetry} about an axis when its value is not affected by rotating the point through any angle about the axis. A surface has circular symmetry about an axis when it is a surface of revolution about the axis.} about an axis through the centre perpendicular to the plane of the boundary, and then allowed to vibrate. \par 

Here we have to solve 
\begin{equation*}
    D_t^2z=c^2\left(D_r^2z+\frac{1}{r}D_rz+\frac{1}{r^2}d_\phi^2z \right)
    \tag{[XI] Art. 1}
\end{equation*}
subject to the conditions 
\begin{eqnarray}
    z=f(r) & \text{when} & t=0 \label{eq11.1} \\ 
    D_tz=0 & \text{``} & t=0 \label{eq11.2} \\ 
    z=0 & \text{``} & r=a. \label{eq11.3} 
\end{eqnarray} \par 

From the symmetry of the supposed initial distortion $z$ must be independent of $\phi$, therefore [XI] reduces to 
\begin{equation}
    D_t^2z=c^2\left(D_r^2z+\frac{1}{r}D_rz\right) \label{eq11.4}
\end{equation}
and this is the equation for which we wish to find a particular solution. \par 

We shall employ a device not unlike that used in Art. 9. \par 

Assume\footnote{See note on page \pageref{note2}} $z=R.T$ where $R$ is a function of $r$ alone and $T$ is a function of $t$ alone. Substitute this value of $z$ in \eqref{eq11.4} and we get 
\begin{equation*}
    RD_t^2T=c^2T\left(D_r^2R+\frac{1}{r}D_rR\right)
\end{equation*}
or 
\begin{equation}
    \frac{1}{c^2T}\frac{d^2T}{dt^2}=\frac{1}{R}\left(\frac{d^2R}{dr^2}+\frac{1}{r}\frac{dR}{dr}\right). \label{eq11.5}
\end{equation}
The second member of \eqref{eq11.5} does not involve $t$, therefore its equal the first member must be independent of $t$. The first member of \eqref{eq11.5} does not involve $r$, and consequently since it contains neither $t$ not $r$, it must be constant. Let it equal $-\mu^2$, where $\mu$ of course is an undetermined constant. \par 

Then \eqref{eq11.5} breaks up into the two differential equations 
\begin{equation}
    \frac{d^2T}{dt^2}+\mu^2c^2T=0 \label{eq11.6}
\end{equation}
\begin{equation}
    \frac{d^2R}{dr^2}+\frac{1}{r}\frac{dR}{dr}+\mu^2R=0 \label{eq11.7}
\end{equation}
\eqref{eq11.6} can be solved by familiar methods, and we get $t=\cos\mu ct$ and $t=\sin\mu ct$ as simple particular solutions (v. Int. Cal. p. 319, \S\, 21). \par 

To solve \eqref{eq11.7} is not so easy. We shall first simplify it by a change of independent variable. Let $r=\frac{x}{\mu}$. \eqref{eq11.7} becomes 
\begin{equation}
    \frac{d^2R}{dx^2}+\frac{1}{x}\frac{dR}{dx}+R=0. \label{eq11.8}
\end{equation} \par 

Assume, as in Art. 9, that $R$ can be expressed in terms of whole powers of $x$. Let $R=\sum a_nx^n$ and substitute in \eqref{eq11.8}. We get 
\begin{equation*}
    \sum[n(n-1)a_nx^{n-2}+na_nx^{n-2}+a_nx^n]=0,
\end{equation*}
an equation which must be true no matter what the value of $x$. The coefficient of any given power of $x$, as $x^{k-2}$, must, then, vanish, and 
$$k(k-1)a_k+ka_k+a_{k-2}=0$$ 
or $$k^2a_k+a_{k-2}=0$$  
whence we obtain 
\begin{equation}
    a_{k-2}=-k^2a_k  \label{eq11.9}
\end{equation}
as the only relation that need be satisfied by the coefficients in order that $R=\sum a_kx^k$ shall be a solution of \eqref{eq11.8}. \par 

If 
$$ k=0,\quad a_{k-2}=0,\quad a_{k-4}=0,\quad \& c.$$ 
We can then begin with $k=0$ as our lowest subscript. \par 

From \eqref{eq11.9} 
$$a_k=-\frac{a_k-2}{k^2}.$$
Then 
\begin{align*}
    a_2&=-\frac{a_0}{2^2} \\
    a_4&=\frac{a_0}{2^2\cdot4^2} \\
    a_6&=-\frac{a_0}{2^2\cdot4^2\cdot6^2},\, \text{\& c.}
\end{align*}
Hence 
$$R=a_0\left[1-\frac{x^2}{x}+\frac{x^4}{2^2\cdot4^2}-\frac{x^6}{2^2\cdot4^2\cdot6^2}+\dots\right]$$
where $a_0$ may be taken at pleasure, is a solution of \eqref{eq11.8}, provided the series is convergent. \par 



\end{document}