\documentclass[12pt,oneside]{book}

\usepackage[margin=1in]{geometry}
\usepackage{gensymb,mathtools,amssymb,graphicx,setspace,titlesec}

\setlength{\parindent}{0pt}
\setlength{\parskip}{6pt}

\titleformat{\chapter}[hang]{\bfseries}{}{0pt}{\Huge\uppercase}[]

\begin{document}
    
\frontmatter

Project Gutenberg's Legends of Babylon and Egypt, by Leonard W. King \par 

This eBook is for the use of anyone anywhere at no cost and with almost no restrictions whatsoever. You may copy it, give it away or re-use it under the terms of the Project Gutenberg License included with this eBook or online at www.gutenberg.org \par 

Title: Legends of Babylon and Egypt \par 
\hspace{1cm}In Relation to Hebrew Tradition \par 

Author: Leonard W. King \par 

Release Date: March 28, 2006 [EBook \#2030] \\
Last Updated: February 4, 2013 \par 

Language: English \par 

*** START OF THIS PROJECT GUTENBERG EBOOK LEGENDS OF BABYLON AND EGYPT ***

\vfill 

Produced by John Bickers, Dagny and David Widger \par 

\begin{titlepage}
    \centering 
    \vspace*{2cm}
    {\Huge \textbf{Legends of Babylon and Egypt}} \par 
    \vspace{3cm}
    {\Large \textbf{In Relation to Hebrew Tradition}}
\end{titlepage}

\begin{center}
    {\Large By Leonard W. King, M.A., Litt.D., F.S.A.} \\
    \vspace{1cm}
    {\large Assistant Keeper of Egyptian and Assyrian Antiquities in the British Museum} \\
    \vspace{1cm}
    {\large Professor in the University of London King's College} \\
    \vspace{1cm}
    {\large First Published 1918 by Humphrey Milford, Oxford University Press.} \\
    \vspace{1cm}
    {\large \uppercase{The British Academy \\ The Schweich Lectures 1916}}
    \vfill 
    Preparer's Note \\
\end{center}
This text was prepared from a 1920 edition of the book, hence the references to dates after 1916 in some places. Greek text has been transliterated within brackets ``{}'' using an Oxford English Dictionary alphabet table. Diacritical marks have been lost. \par 

\tableofcontents

\chapter{Preface}
In these lectures an attempt is made, not so much to restate familiar facts, as to accommodate them to new and supplementary evidence which has been published in America since the outbreak of the war. But even without the excuse of recent discovery, no apology would be needed for any comparison or contrast of Hebrew tradition with the mythological and legendary beliefs of Babylon and Egypt. Hebrew achievements in the sphere of religion and ethics are only thrown into stronger relief when studied against their contemporary background. \par 

The bulk of our new material is furnished by some early texts, written towards the close of the third millennium B.C. They incorporate traditions which extend in unbroken outline from their own period into the remote ages of the past, and claim to trace the history of man back to his creation. They represent the early national traditions of the Sumerian people, who preceded the Semites as the ruling race in Babylonia; and incidentally they necessitate a revision of current views with regard to the cradle of Babylonian civilization. The most remarkable of the new documents is one which relates in poetical narrative an account of the Creation, of Antediluvian history, and of the Deluge. It this exhibits a close resemblance in structure to the corresponding Hebrew traditions, a resemblance that is not shared by the Semitic-Babylonian Versions at present known. But in matter the Sumerian tradition is more primitive than any of the Semitic versions. In spite of the fact that the text appears to have been reached us in a magical setting, and to some extent in epitomized form, this early document enables us to tap the stream of tradition at a point far above any at which approach has hitherto been possible. \par 

Though the resemblance of early Sumerian tradition to that of the Hebrews is striking, it furnishes a still closer parallel to the summaries preserved from the history of Berossus. The huge figures incorporated in the latter's chronological scheme are no longer to be treated as a product of Neo-Babylonian speculation; they reappear in their original surroundings in another of these early documents, the Sumerian Dynastic List. The sources of Berossus had inevitably been semitized by Babylon; but two of his three Antediluvian cities find their place among the five of primitive Sumerian belief, and two of his ten Antediluvian kings rejoin their Sumerian prototypes. Moreover, the recorded ages of Sumerian and Hebrew patriarchs are strangely alike. It may be added that in Egypt a new fragment of the Palermo Stele has enabled us to verify, by a very similar comparison, the accuracy of Manetho's sources for his prehistoric period, while at the same time it demonstrates the way in which possible inaccuracies in his system, deduced from independent evidence, may have arisen in remote antiquity. It is clear that both Hebrew and Hellenistic traditions were modelled on very early lines. \par 

Thus our new material enables us to check the age, and in some measure the accuracy, of the traditions concerning the dawn of history which the Greeks reproduced from native sources, both in Babylonia and Egypt, after the conquests of Alexander had brought the Near East within the range of their intimate acquaintance. The third body of tradition, that of the Hebrews, thought unbacked by the prestige of secular achievement, has, through incorporation in the canons of two great religious systems, acquired an authority which the others have not enjoyed. In re-examining the sources of all three accounts, so far as they are affected by the new discoveries, it will be of interest to observe how the same problems were solved in antiquity by very different races, living under widely divergent conditions, but within easy reach of one another. Their periods of contact, ascertained in history or suggested by geographical considerations, will prompt the further question to what extent each body of belief was evolved in independence of the others. The close correspondence that has long been recognized and is now confirmed between the Hebrew and the Semitic-Babylonian systems, as compared with that of Egypt, naturally falls within the scope of our enquiry. \par 

Excavation has provided an extraordinarily full archaeological commentary to the legends of Egypt and Babylon; and when I received the invitation to deliver the Schweich Lectures for 1916, I was reminded of the terms of the Bequest and was asked to emphasize the archaeological side of the subject. Such material illustration was also calculated to bring out, in a more vivid manner than was possible with purely literary evidence, the contrasts and parallels presented by Hebrew tradition. Thanks to a special grant for photographs from the British Academy, I was enabled to illustrate by means of lantern slides many of the problems discussed in the lectures; and it was originally intended that the photographs than shown should appear as plates in this volume. But in view of the continued and increasing shortage of paper, it was afterwards felt to be only right that all illustrations should be omitted. This very necessary decision has involved a recasting of certain sections of the lectures as delivered, which in its turn has rendered possible a fuller treatment of the new literary evidence. To the consequent shifting of interest is also due a transposition of names in the title. On their literary side, and in virtue of the intimacy of their relation to Hebrew tradition, the legends of Babylon must be given precedence over those of Egypt. \par 

For the delay in the appearance of the volume I must plead the pressure of other work, on subjects far removed from archaeological study and affording little time and few facilities for a continuance of archaeological and textual research. It is hoped that hte insertion of references throughout, and the more detailed discussion of problems suggested by our new literary material, may incline the reader to add his indulgence to that already extended to me by the British Academy. \par 

L. W. King. 

\mainmatter

\chapter[Lecture I - Origins of Civilization]{Lecture I--Egypt, Babylon,\\ And Palestine, and Some \\Traditional Origins of Civilization}
At the present moment most of us have little time or thought to spare for subjects not connected directly or indirectly with the war. We have put aside our own interests and studies; and after the war we shall all have a certain amount of leeway to make up in acquainting ourselves with what has been going on in countries not yet involved in the great struggle. Meanwhile the most we can do is to glance for a moment at any discovery of exceptional interest that may come to light. \par 

The main object of these lectures will be to examine certain Hebrew traditions in the light of new evidence which has been published in America since the outbreak of the war. The evidence is furnished by some literary texts, inscribed on tablets from Nippur, one of the oldest and most sacred cities of Babylonia. They are written in Sumerian, the language spoken by the non-Semitic people whom the Semitic Babylonians conquered and displaced; and they include a very primitive version of the Deluge story and Creation myth, and some texts which throw new light on the age of Babylonian civilization and on the area within which it had its rise. In them we have recovered some of the material from which Berossus derived his dynasty of Antediluvian kings, and we are thus enabled to test the accuracy of the Greek tradition by that of the Sumerains themselves. So far then as Babylonia is concerned, these documents will necessitate a re-examination of more than one problem. \par 

The myths and legends of ancient Egypt are also to some extent involved. The trend of much recent anthropological research has been in the direction of seeking a single place of origin for similar beliefs and practices, at least among races which were bound to one another by political or commercial ties. And we shall have occasion to test, by means of our new data, a recent theory of Egyptian influence. The Nile Valley was, of course, one of the great centres from which civilization radiated throughout the ancient East; and, even when direct contact is unproved, Egyptian literature may furnish instructive parallels and contrasts in any study of Western Asiatic mythology. Moreover, by a strange coincidence, there has also been published in Egypt since the beginning of the way a record referring to the reigns of predynastic rulers in the Nile Valley. This, like some of the Nippur texts, take us back to that dim period before the dawn of actual history, and, though the information it affords is not detailed like theirs, it provides fresh confirmation of the general accuracy of Manetho's sources, and suggests some interesting points for comparison. \par 

But the people with whose traditions we are ultimately concerned are the Hebrews. In the first series of Schweich Lectures, delivered in the year 1908, the late Canon Driver showed how the literature of Assyria and Babylon had thrown light upon Hebrew traditions concerning the origin and early history of the world. The majority of the cuneiform documents, on which he based his comparison, date from a period no earlier than the seventh century B.C., and yet it was clear that the texts themselves, in some form or other, must have descended from a remote antiquity. He concluded his brief reference to the Creation and Deluge Tablets with these words: ``The Babylonian narratives are both polytheistic, while the corresponding biblical narratives (Gen. i and vi-xi) are made the vehicle of a pure and exalted monotheism; but in spite of this fundamental difference, and also variations in detail, the resemblances are such as to leave no doubt that the Hebrew cosmogony and the Hebrew story of the Deluge are both derived ultimately from the same original as the Babylonian narratives, only transformed by the magic tough of Israel's religion, and infused by it with a new spirit.''\footnote{Driver, \textit{Modern Research as illustrating the Bible} (The Schweich Lectures, 1908), p. 23.} Among the recently published documents from Nippur we have at last recovered one at least of those primitive originals from which the Babylonian accounts were derived, while others prove the existence of variant stories of the world's origin and early history which have not survived in the later cuneiform texts. In some of these early Sumerian records we may trace a faint but remarkable parallel with the BHebrew traditions of man's history between his Creation and the Flood. It will be our task, then, to examine the relations which the Hebrew narratives bear both to the early Sumerian and to the later Babylonian Versions, and to ascertain how far the new discoveries support or modify current views with regard to the contents of those early chapters of Genesis. \par 

I need not remind you that Genesis is the book of Hebrew origins, and that its contents mark it off to some extent from the other books of the Hebrew Bible. The object of the Pentateuch and the Book of Joshua is to describe in their origin the fundamental institutions of the national faith and to trace from the earliest times the course of events which led to the Hebrew settlement in Palestine. Of this national history the Book of Genesis forms the introductory section. Four centuries of complete silence lie between its close and the beginning of Exodus, where we enter on the history of a nation as contrasted with that of a family.\footnote{Cf., E.g., Skinner, \textit{A Critical and Exegetical Commentary on Genesis} (1912), p. ii f.; Driver, \textit{The Book of Genesis}, 10th ed. (1916), pp. 1 ff.; Ryle, \textit{The Book of Genesis} (1914), pp. x ff.} While Exodus and the succeeding books contain national traditions, Genesis is largely made up of individual biography. Chapters xii-l are concerned with the immediate ancestors of the Hebrew race, beginning with Abram's migration into Canaan and closing with Joseph's death in Egypt. But the aim of the book is not confined to recounting the ancestry of Israel. It seeks also to show her relation to other peoples in the world, and probing still deeper into the past it describes how the earth itself was prepared for man's habitation. Thus the patriarchal biographies are preceded, in chapters i-xi, by an account of the original of the world, the beginnings of civilization, and the distribution of the various races of mankind. It is, of course, with certain parts of this first group of chapters that such striking parallels have long been recognized in the cuneiform texts. \par 

In approaching this particular body of Hebrew traditions, the necessity for some caution will be apparent. It is not as though we were dealing with the reported beliefs of a Malayan or Central Australian tribe. In such a case there would be no difficulty in applying a purely objective criticism, without regard to ulterior consequences. But here our own feelings are involved, having their roots deep in early associations. The ground too is well trodden; and, had there been no new material to discuss, I think I should have preferred a less contentious theme. The new material is my justification for the choice of subject, and also the fact that, whatever views we may hold, it will be necessary for us to assimilate it to them. I shall have no hesitation in giving you my own reading of the evidence; but at the same time it will be possible to indicate solutions which will probably appeal to those who view the subject from more conservative standpoints. That side of the discussion may well be postponed until after the examination of the new evidence in detail. And first of all it will be advisable to clear up some general aspects of the problem, and to define the limits within which our criticism may be applied. \par 

It must be admitted that both Egypt and Babylon bear a bad name in Hebrew tradition. Both are synonymous with captivity, the symbols of suffering endured at the beginning and at the close of the national life. And during the struggle against Assyrian aggression, the disappointment at the failure of expected help is reflected in prophecies of the period. These great crises in Hebrew history have tended to obscure in the national memory the part which both Babylon and Egypt may have played in moulding the civilization of the smaller nations with whom they cam in contact. To such influence the races of Syria were, by geographical position, peculiarly subject. The country has often been compared to a bridge between the two great continents of Asia and Africa, flanked by the sea on one side and the desert on the other, a narrow causeway of highland and coastal plain connecting the valleys of the Nile and the Euphrates.\footnote{See G. A. Smith, \textit{Historical Geography of the Holy Land}, pp. 5 ff., 45 ff., and Myres, \textit{Dawn of History}, pp. 137 ff.; and cf. Hogarth, \textit{The Nearer East}, pp. 65 ff., and Reclus, \textit{Nouvelle G\'eographie universelle}, t. IX, pp. 685 ff.} For, except on the frontier of Egypt, desert and sea do not meet. Farther north the Arabian plateau is separated from the Mediterranean by a double mountain chain, which runs south from the Taurus at varying elevations, and encloses in its lower course the remarkable depression of the Jordan Valley, the Dead Sea, and the 'Arabah. The Judaean hills and the mountains of Moab are merely the southward prolongation of the Lebanon and Anti-Lebanon, and their neighbourhood to the sea endows this narrow tract of habitable country with its moisture and fertility. It thus formed the natural channel of intercourse between the two earliest centres of civilization, and was later the battle-ground of their opposing empires. \par 

The great trunk-roads of through communication run north and south, across the eastern plateaus of the Haur\^an and Moab, and along the coastal plains. The old highway from Egypt, which left the Delta at Pelusium, at first follows the coast, then trends eastward across the plain of Esdraelon, which breaks the coastal range, and passing under Hermon runes northward through Damascus and reaches the Euphrates at its most westerly point. Other through tracks in Palestine ran then as they do to-day, by Beesheba and Hebron, or along the 'Arabah and west of the Dead Sea, or through Edom and east of Jordan by the present Jahh route to Damascus. But the great highway from Egypt, the most westerly of the trunk-roads through Palestine, was that mainly followed, with some variant sections, by both caravans and armies, and was known by the Hebrews in its southern course as the ``Way of the Philistines'' and farther north as the ``Way of the East.'' \par 

The plain of Esraelon, where the road first trends eastward, has been the battle-ground for most invaders of Palestine from the north, and though Egyptian armies often fought in the southern coastal plain, they too have battled there when the held the southern country. Megiddo, which commands the main pass into the plain through the low Samaritan hills to the southeast of Carmel, was the site of Thothmes III's famous battle against a Syrian confederation, and it inspired the writer of the Apocalypse with his vision of an Armageddon of the future. But invading armies always followed the beaten track of caravans, and movements represented by the great campaigns were reflected in the daily passage of international commerce. \par 

With so much through traffic continually passing within her borders, it may be matter for surprise that far more striking evidence of its cultural effect should not have been revealed by archaeological research in Palestine. Here again the explanation is mainly of a geographical character. For though the plains and plateaus could be crossed by the trunk-roads, the rest of the country is so broken up by mountain and valley that it presented few facilities either to foreign penetration or to external control. The physical barriers to local intercourse, reinforced by striking differences in soil, altitude, and climate, while they precluded Syria herself from attaining national unity, always tended to protect her separate provinces, or ``kingdoms,'' from the full effects of foreign aggression. One city-state could be traversed, devastated, or annexed , without in the least degree affecting neighbouring areas. It is true that the population of Syria has always been predominantly Semitic, for she was on the fringe of the great breeding-ground of the Semitic race and her landward boundary was open to the Arabian nomad. Indeed, in the whole course of her history the only race that bade fair at one time to oust the Semite in Syria was the Greek. But the Greeks remained within the cities which they founded or rebuilt, and, as Robertson Smite pointed out, the death-rate in Eastern cities habitually exceeds the birth-rate; the urban population must be reinforced from the country if it is to be maintained, so that the type of population is ultimately determined by the blood of the peasantry.\footnote{See Robertson Smith, \textit{Religion of the Semites}, p. 12 f.; and cf. Smith, \textit{Hist. Geogr.}, p. 10 f.} Hence after the Arab conquest the Greek elements in Syria and Palestine tended rapidly to disappear. The Moslem invasion was only the last of a series of similar great inroads, which have followed on another since the dawn of history, and during all that time absorption was continually taking place from desert tribes that ranged the Syrian border. As we have seen, the country of his adoption was such as to encourage the Semitic nomad's particularism, which was inherent in his tribal organization. Thus the predominance of a single racial element in the population of Palestine and Syria did little to break down or overstep the natural barriers and lines of cleavage. \par 

These facts suffice to show why the influence of both Egypt and Babylon upon the various peoples and kingdoms of Palestine was only intensified at certain periods, when ambition for extended empire dictated the reduction of her provinces in detail. But in the long intervals, during which there was no attempt to enforce political control, regular relations were maintained along the lines of trade and barter. And in any estimate of the possible effect of foreign influence upon Hebrew thought, it is important to realize that some of the channels through which in later periods it may have acted had been flowing since the dawn of history, and even perhaps in prehistoric times. It is probable that Syria formed on e of the links by which we may explain the Babylonian elements that are attested in prehistoric Egyptian culture.\footnote{Cf. \textit{Sumer and Akkad}, pp. 322 ff.; and for a full discussion of the points of resemblance between the early Babylonian and Egyptian civilizations, see Sayce, \textit{The Archaeology of the Cuneiform Inscriptions}, chap. iv, pp. 101 ff.} But another possible line of advance may have been by way of Arabia and across the Red Sea into Upper Egypt. \par 

The latter line of contact is suggested by an interesting piece of evidence that has recently been obtained. A prehistoric flint knife, with a handle carved from the tooth of a hippopotamus, has been purchased lately by the Louvre,\footnote{See B\'en\'edite, ``Le couteau de Gebel al-'Arak'', in \textit{Foundation Eug\'ene Piot, Mon. et. M\'em.,} XXII. i. (1916).} and is said to have been found at Gebel el'Arak near Nage' Ham\^adi, which lies on the Nile not far below Koptos, where an ancient caravan-track leads by W\^adi Hamm\^am\^at to the Red Sea. On one side of the handle is a battle-scene including some remarkable representations of ancient boats. All the warriors are nude with the exception of a loin girdle, but, while one set of combatants have shaven heads or short hair, the others have abundant locks falling in a thick mass upon the shoulder. On the other face of the handle is carved a hunting scene, two hunters with dogs and desert animals being arranged around a central boss. But in the upper field is a very remarkable group, consisting of a personage struggling with two lions arranged symmetrically. The rest of the composition is not very unlike other examples of prehistoric Egyptian carving in low relief, but here attitude, figure, and clothing are quite un-Egyptian. The hero wears a sort of turban on his abundant hair, and a full and rounded beard descends upon his breast. A long garment clothes him from the waist and falls below the knees, his muscular calves ending in the claws of a bird of prey. There is nothing like this in prehistoric Egyptian art. \par 

Perhaps Monsieur B\'en\'edite is pressing his theme too far when he compares the close-cropped warriors on the handle with the shaven Sumerians and Elamites upon steles from Telloh and Susa, for their loin-girdles are African and quite foreign to the Euphrates Valley. And his suggestion that two of the boats, flat-bottomed and with high curved ends, seem only to have navigated the Tigris and Euphrates,\footnote{Op. cit., p. 32.} will hardly command acceptance. But there is no doubt that the heroic personage upon the other face is represented in the familiar attitude of the Babylonian hero Gilgamesh struggling with lions, which formed so favourite a subject upon early Sumerian and Babylonian seals. His garment is Sumerian or Semitic rather than Egyptian, and the mixture of human and bird elements in the figure, though not precisely paralleled at this early period, is not out of harmony with Mesopotamian or Susan tradition. His beard, too is quite different from that of the Libyan desert tribes which the early Egyptian kings adopted. Though the treatment of the lions is suggestive of proto-Elamite rather than of early Babylonian is suggestive of proto-Elamite rather than of early babylonian models, the design itself is unmistakably of Mesopotamian origin. This discovery intensifies the significance of other early parallels that have been noted between the civilizations of the Euphrates and the Nile, but its evidence, so far as it goes, does not point to Syria as the medium of prehistoric intercourse. Yet then, as later, there can have been no physical barrier to the use of the river-route from Mesopotamia into Syria and of the tracks thence southward along the land-bridge to the Nile's delta. \par 

In the early historic periods we have definite evidence that the eastern coast of the Levant exercised a strong fascination upon the rules of both Egypt and Babylonia. It may be admitted that Syria had little to give in comparison to what she could borrow, but her local trade in wine and oil must have benefited by an increase in the through traffic  which followed the working of copper in Cyprus and Sinai and of silver in the Taurus. Moreover, in the cedar forests of Lebanon and the north she possessed a product which was highly valued both in Egypt and the treeless plains of Babylonia. The cedars procured by Sneferu from Lebanon at the close of the IIIrd Dynasty were doubtless floated as rafts down the coast, and we may see in them evidence of a regular traffic in timber. It has long been known that the early Babylonian king Sharru-kin, or Sargon of Akkad, had pressed up the Euphrates to the Mediterranean, and we now have information that he too was fired by a desire for precious wood and metal. One of the recently published Nippur inscriptions contains copies of a number of his texts, collected by an ancient scribe from his statues at Nippur, and from these we gather additional details of his campaigns. We learn that after his complete subjugation of Southern Babylonia he turned his attention to the west, and that Enlil gave him the lands ``from the Upper Sea to the Lower Sea'', i.e. from the Mediterranean to the Persian Gulf. Fortunately this rather vague phrase, which survived in later tradition, is restated in greater detail in one of the contemporary version,s which records that Enlil ``gave him the upper land, Mari, Iarmuti, and Ibla, as far as the Cedar Forest and the Silver Mountains''.\footnote{See Poevel, \textit{Historical Texts} (Univ. of Penns. Mus. Publ., Bab. Sect., Vol. IV, No. 1, 1914), pp. 177 f., 222 ff.} \par 

Mari was a city on the middle Euphrates, but the name may here signify the district of Mari which lay in the upper course of Sargon's march. Now we know that the later Sumerian monarch Gudea obtained his cedar beams from the Amanus range, which he names \textit{Amanum} and describes as the ``cedar mountains''.\footnote{Thureau-Dangin, \textit{Les inscriptions de Sumer de d'Akkad}, p. 108 f., Statue B, col. v. 1. 28; Germ. ed., p. 68 f.} Doubtless he felled his trees on the eastern slopes of the mountain. But we may infer from his texts that Sargon actually reached the coast, and his ``Cedar Forest'' may have lain farther to the south, perhaps as far south as the Lebanon. The ``Silver Mountains'' can only be identified with the Taurus, where silver mines were worked in antiquity. The reference to Iarmuti is interesting, for it is clearly the same lace as Iarimuta or Iarimmuta, of which we find mention in the Tell el=Amarna letters. From the references to this district in the letters of Rib-Adda, governor of Byblos, we may infer that it was a level district on the coast, capable of producing a considerable quantity of grain for export, and that it was under GEgyptian control at the time of Amenophis IV. Hitherto its position has been conjecturally placed in the Nile Delta, but from Sargon's reference we must probably seek it on the North Syrian or possibly the Cilician coast. Perhaps, as Dr. Poebel suggests, it was the plain of Antioch, along the lower course and at the mouth of the Orontes. But his further suggestion that the term is used by Sargon for the whole stretch of country between the sea and the Euphrates is hardly probable. For the geographical references need not be treated as exhaustive, but as confined to the more important districts through which the expedition passed. The district of Ibla which is also mentioned by Nar\^am-Sin and Gudea, lay probably to the north of Iarmuti, perhaps on the southern slopes of Taurus. It, too, we may regard as a district of restricted extent rather than as a general geographical term for the extreme north of Syria. \par 

It is significant that Sargon does not allude to any battle when describing this expedition, nor does he claim to have devastated the western countries.\footnote{In some versions of his new records dSargon states that ``5,400 men daily eat bread before him'' (see Poebel, op. cot., p. 178); though the figure may be intended to convey an idea of the size of Sargon's court, we may perhaps see in it a not inaccurate estimate of the total strength of his armed forces.} Indeed, most of these early expeditions to the west appear to have been inspired by motives of commercial enterprise rather than of conquest. But increase of wealth was naturally followed by political expansion, and Egypt's dream of an Asiatic empire was realized by Pharaohs of the XVIIIth Dynasty. The fact that Babylonian should then have been adopted as the medium of official intercourse in Syria points to the closeness of the commercial ties which had already united the Euphrates Valley with the west. Egyptian control had passed from Canaan at the time of the Hebrew settlement, which was indeed a comparatively late episode in the early history of Syria. Whether or not we identify the Khabiri with the Hebrews, the character of the latter's incursion is strikingly illustrated by some of the Tell el-Amerna letters. We see a nomad folk pressing in upon settled peoples and gaining a foothold here and there.\footnote{See especially Professor Burney's forthcoming commentary on Judges (passim), and his forthcoming Schweich Lectures (now delivered, in 1917).} \par 

The great change from desert life consists in the adoption of agriculture, and when once that was made by the Hebrews any further advance in economic development was dictated by their new surroundings. The same process had been going on, as we have seen, in Syria since the dawn of history, the Semitic nomad passing gradually through the stages of agricultural and village life into that of the city. The country favoured the retention of tribal exclusiveness, but ultimate survival could only be purchased at the cost of some amalgamation with their new neighbours. Below the surface of Hebrew history these two tendencies may be traced in varying action and reaction. Some sections of the race engaged readily in the social and commercial life of Canaanite civilization with its rich inheritance from the past. Others, especially in the highlands of Judah and the south, at first succeeded in keeping themselves remote from foreign influence. During the  later periods of the national life the country was again subjected, and in an intensified degree, to those forces of political aggression from Mesopotamia and Egypt which we have already noted as operating in Canaan. But throughout the settled Hebrew community as a whole the spark of desert fire was not extinguished, and by kindling the zeal of the Prophets it eventually affected nearly all the white races of mankind. \par 

In his Presidential Address before the British Association at Newcastle,\footnote{``New Archaeological Lights on the Origins of Civilization in Europe,'' British Association, Newcastle-on-Tyne, 1916.} Sir Arthur Evans emphasized the part which recent archaeology has played in proving the continuity of human culture from the most remote periods. He showed how gaps in our knowledge had been bridged, and he traced the part which each great race had taken in increasing its inheritance. We have, in fact, ample grounds for assuming an interchange, not only of commercial products, but, in a minor degree, of ideas within areas geographically connected; and it is surely not derogatory to any Hebrew writer to suggest that he may have adopted, and used for his own purposes, conceptions current among his contemporaries. In other words, the vehicle of religious ideas may well be of composite origin; and, in the corse of our study of early Hebrew tradition, I suggest that we hold ourselves justified in applying the comparative method to some at any rate of the ingredients which went to form the finished product. The process is purely literary, but it finds an analogy in the study of Semitic art, especially in the later periods. And I think it will make my meaning clearer if we consider for a moment a few examples of sculpture produced by races of Semitic origin. I do not suggest that we should regard the one process as in any way proving the existence of the other. We should rather treat the comparison as illustrating in another medium the effect of forces which, it is clear, were operative at various periods upon races of the same stock from which the Hebrews themselves were descended. In such material products the eye at once detects the Semite's readiness to avail himself oforeign models. In some cases direct borrowing is obvious; in others, to adapt a metaphor from music, it is possible to trace extraneous \textit{motifs} in the design.\footnote{The necessary omission of plates, representing the slides shown in the lectures, has involved a recasting of most passages in which points of archaeological detail were discussed; see Preface. But the following paragraphs have been retained as the majority of the monuments referred to are well known.} \par 

Some of the most famous monuments of Semitic art date from the Persian and Hellenistic periods, and if we glance at them in this connexion it is in order to illustrate during its most obvious phase a tendency of which the earlier effects are less pronounced. In the sarcophagus of the Sidonian king Eshmu-'azar II, which is preserved in the Louvre,\footnote{\textit{Corp. Inscr. Semit.,} I. i, tab. II.} we have indeed a monument to which no Semitic sculptor can lay claim. Workmanship and material are Egyptian, and there is no doubt that it was sculptured in Egypt and transported to Sidon by sea. But the king's own engravers added the long Phoenician inscription, in which he adjures princes and men not to open his resting-place since there are no jewels therein, concluding with some potent curses against any violation of his tomb. One of the latter implores the holy gods to deliver such violators up ``to a mighty prince who shall rule over them'', and was probably suggested by Alexander's recent occupation of Sidon in 332 B.C. after his reduction and drastic punishment of Tyre. Kind Eshmun-'zar was not unique in his choice of burial in an Egyptian coffin, for he merely followed the example of his royal father, Tabn\^ith, ``priest of 'Ashtart and king of the Sidonians'', whose sarcophagus, preserved at Constantinople, still bears in addition to his own epitaph that of its former occupant, a certain Egyptian general Penptah. But more instructive than these borrowed memorials is a genuine example of Phoenician work, the stele set up by Yehaw-milk, king of Byblos, and dating from the fourth of fifth century B.C.\footnote{\textit{C.I.S.}, I. i, tab. I.} In the sculptured panel at the head of the stele the kind is represented in the Persian dress of the period standing in the presence of 'Ashtart or Astarte, his ``Lady, Mistress of Byblos''. There is no doubt that the stele is of native workmanship, but the influence of Egypt may be seen in the technique of the carving, in the winged disk above the figures, and still more in the representation of teh goddess in her character as t hte Egyptian Hathor, with disk and horns, vulture head-dress and papyrus-sceptre. The inscription records the dedication of an altar and shrine to the goddess, and these too we may conjecture were fashioned on Egyptian lines. \par 

The representation of Semitic deities under Egyptian forms and with Egyptian attributes was encouraged by the introduction of their cults into Egypt itself. In addition to Astarte of Byblos, Ba'al, Anath, and Reshef were all borrowed from Syria in comparatively early times and given Egyptian characters. The conical Syrian helmet of Reshef, a god of war and thunder, gradually gave place to the white Egyptian crown, so that as Reshpu he was represented as a royal warrior; and Wadesh, another form of Astarte, becoming popular with Egyptian women as a patroness of love and fecundity, was also sometimes modelled on Hathor.\footnote{See W. Max M\"uller, \textit{Egyptological Researches,} I, p. 32 f., pl. 41, and S. A. Cook, \textit{Religion of Ancient Palestine,} pp. 83 ff.} \par 

Semitic colonists on the Egyptian border were ever ready to adopt Egyptian symbolism in delineating the native gods to whom they owed allegiance, and a particularly striking example of this may be seen on a stele of the Persian period preserved in the Cairo Museum.\footnote{M\"uller, op. cit., p. 30 f., pl. 40. Numismatic evidence exhibits a similar readiness on the part of local Syrian cults to adopt the veneer of Hellenistic civilization while retaining in great measure their own individuality; see Hill, ``Some Palestinian Cults in the Graeco-Roman Age'', in \textit{Proceedings of the British Academy}, Vol. V (1912).} It was found at Tell Defenneh, on the right bank of the Pelusiac branch of the Nile, close to the old Egyptian highway into Syria, a site which may be identified with that of the biblical Tahpanhes and the Daphnae of the Greeks. Here it was that the Jewish fugitives, fleeing with Jeremiah after the fall of Jerusalem, founded a Jewish colony beside a flourishing Phoenician and Aramaean settlement. One of the local gods of Tahpanhes is represented on the Cairo monument, an Egyptian stele in the form of a naos with the winged solar disk upon its frieze. He stands on the back of a lion and is clothed in Asiatic costume with the high Syrian tiara crowning his abundant hair. The syrian workmanship is obvious, and the Syrian character of the cult ay be recognized in such details as the small brazen fire-altar before the god, and the sacred pillar which is being anointed by the officiating priest. But the god holds in his left hand a purely Egyptian sceptre and in his right an emblem as purely Babylonian, the weapon of Marduk and Gilgamesh which was also wielded by early Sumerian kings. \par 

The Elephantine papyri have shown that the early Jews of the Diaspora, though untrammeled by the orthodozy of Jerusalem, maintained the purity of their local cult in the fact of considerable difficulties. Hence the gravestones of their Aramaean contemporaries, which have been found in Egypt, can only be cited to illustrate the tempatations to which they were exposed.\footnote{It may be admitted that the Greek platonized cult of Isis and Osiris had its origin in the fusion of Greeks and Egyptians which took place in Ptolemaic times (cf. Scott-Moncrieff, \textit{Paganism and Christianity in Egypt}, p. 33 f.). But we may assume that already in the Persian period the Osiris cult had begun to acquire a tinge of mysticism, which, though it did not affect the mechanical reproduction of the native texts, appealed to the Oriental mind as well as to certain elements in Greek religion. Persian influence probably prepared the way for the Platonic exegesis of the Osiris and Isis legends which we find in Plutarch; and the latter may have been in great measure a development, and not, as is often assumed, a complete misunderstanding of the later Egyptian cult.} Such was the memorial erected by Abseli to the memory of his parents, Abb\^a and Ahatb\^u, in the fourth year of Xerxes, 481 B.C.\footnote{\textit{C.I.S.,} II. i, tab. XI, No. 122.} They had evidently adopted the religion of Osiris, and were buried at Saqq\^arah in accordance with the Wgyptian rites. The upper scene engraved upon the stele represents Abb\^a and his wife in the presence of Osiris, who is attended by Isis and Nephthys; and in the lower panel is the funeral scene, in which all the mourners with one exception are Asiatics. Certain details of the rites that are represented, and mistakes in the hieroglyphic version of the text, prove that the work is Aramaean throughout.\footnote{A very similar monument is the Carpentras Stele (\textit{C.I.S.,} II., i, tab. XIII, No. 141), commemborating Taba, daughter of Tahapi, an Aramaean lady who was also a convert to Osiris. It is rather later than that of Abb\^a and his wife, since the Aramaic characters are transitional from the archaic to the square alphabet; see Driver, \textit{Notes on the Hebrew Text of the Books of Samuel,} pp. xviii ff., and Cooke, \textit{North Semitic Inscriptions,} p. 205 f. The Vatican Stele (op. cit. tab. XIV. No. 142), which dates from the fourth century, represents inferior work.} \par 

If our examples of Semitic art were confined to the Persian and later periods, they could only be employed to throw light on their own epoch, when through communication had been organized, and there was consequently a certain poling of commercial and artistic products throughout the empire.\footnote{Cf. Beven, \textit{House of Seleucus,} Vol. I, pp. 5, 260 f. The artistic influence of Mesopotamia was even more widely spread than that of Egypt during the Persian period. This is suggested, for example, by the famous lion-weight discovered at Abydos in Mysia, the town on the Hellespont famed for the loves of Hero and Leander. The letters of its Aramaic inscription (\textit{C.I.S.,} II. i, tab. Vii, No. 108) prove by their form that it dates from the Persian period, and its provenance is sufficiently attested. Its weight moreover suggests that it was not merely a Babylonian or Persian importation, but casat for local use, yet in design and technique it is scarcely distinguishable from the beest Assyrian work of the seventh century.} It is true that under the Great King the various petty states and provinces were encouraged to manage their own affaris so long as they paid the required tribute, but their horizon naturally expanded with increase of commerce and the necessity for service in the king's armies. At this time Aramaic was the speech of Syria, and the population, especially in the cities, was still largely Aramaean. As early as the thirteenth century sections of this interresting Semitic race had begun to press into Northern Syria from the middle Euphrates, and they absorbed not only the old Canaanite population but also the Hittite immigrants from Capadocia. The latter indeed may for a time have furnished rulers to the vigorous North Syrian principalities which resulted from this racial combination, but the Aramaean element, thanks to continual reinforcement, was numerically dominant, and their art may legitimately be regarded as in great measure a Semitic product. Fortunately we have recovered examples of sculpture which prove that tendencies already notes in the Persian period were at work, though in a minor degree, under the alter Assyrian empire. The discoveries made at Zenjirli, for example, iluustrate the gradually increasing effect of Asyrian influence upon the artitic output of a small North Syrian state. \par 

This village in north-western Syria, on the road between Antioch and Mar'ash, marks the site of a town which lay near the southern border or just within the Syrian district of Sam'al. The latter is first mentioned in the Assyrian inscriptions by Shalmaneser III, the son and successor of the great conqueror, Ashur-nasir-pal; and in the first half of the eighth century, though within the radius of Assyrian influence, it was still an independent kingdom, It is to this period that we must assign the earliest of the inscribed monements discovered at Zenjirli and its neighbourhood. At Gerjin, not far to the north-west, was found the colossal statue of Dadad, chief god of the Aramaeans, which was fashioned and set up in his honour by Panammu I, son of Qaral and king of Ya'di.\footnote{See F. von Luschan, \textit{Sendschirli}, I. (1893), pp. 49 ff., pl. vi; and cf. Cooke, \textit{North Sem. Inscr.}, pp. 159 ff. The characters of the inscription on the statue are of the same archaic type as those of the Moabite Stone, though unlike them they are engraved in relief; so too are the unscriptions of Panammu's later successor Bar-rekub (see below). Gerjin was certainly in Ya'di, and Winckler's suggestion that Zenjirli itself also lay in that district but near the border of Sam'al may be provisionally accepted; the occurrence of the names in the inscriptions can be explained in more than one way (see Cooke, op. cit., p. 183).} In the long Aramaic inscription engraved upon the statue Panammu records the prosperity of his reign, which he ascribes to the support he has received from Hadad and his other gods, El, Reshef, Rekub-el, and Shamash. He had evidently been left in peace by Assyria, and the monument he erected to his god is of Aramaean workmanship and design. But the influence of Assyria may be traced in Hadad's beard and in his horned head-dress, modelled on that worn by Babylonian and Assyrian gods as the symbol of divine power. \par 

The political changes introduced into Ya'di and Sam'al by Tiglath-pileser IV are reflected in the inscriptions and monuments of Bar-rekub, a later king of the district. Internal strife had brought disaster upon Ya'di and the throne had been secured by Panammu II, son of Bar-sur, whose claims received Assyrian support. In the owrds of his son Bar-rekub, ``he laid hold of the skirt of his lord, the king of Assyria'', who was gracious to him; and it was probably at this time, and as a reward for his loyalty, that Ya'di was united with the neighbouring district of Sam'al. But Panammu's devotion to his foreign master led to his death, for he died at the siege of Damascus, in 733 or 732 B.C., ``in the camp, while following his lord, Tiglath-pileser, king of Assyria.'' His kinsfold and the whole camp bewailed him, and his body was sent back to Ya'di, where it was interred by his son, who set up an inscribed statue to his memory. Bar-rekub followed in his father's footsteps, as he leads us to infer in his palace-inscription found at Zanjirli: ``I ran at the wheel of my lord, the king of Assyria, in the midst of mighty kings, possessors of silver and possessors of gold.'' It is not strange therefore that his art should reflect Assyrian influence far more strikingly than that of Panammu I. The figure of himself which he caused to be carved in relief on the left side of the palace-inscription is in the Assyrian style,\footnote{\textit{Sendschirli,} IV (1911), pl. lxvii. Attitude and treatment of robes are both Assyrian, and so is the arrangement of divine symblos in the upper field, though some of the latter are given under unfamiliar forms. The king's close-fitting peaked cap was evidently the royal headdres of Sam'al; see the royal figure on a smaller stele of inferior design, op. cit., pl. lxvi.} and so too is another of his reliefs from Zenjirli. On the latter Bar-rekub is represented seated upon his throne with eunuch and scribe in attendance, while in the field is the emblem of full moon and crescent, here ascribed to ``Ba'al of Haran,'' the famous centre of moon-worship in Northern Mesopotamia.\footnote{Op. cit. pp. 257, 346 ff., and pl. lx. The general style of the sculpture and much of the detail are obviously Assyrian. Assyrian influence is particularly noticeable in Bar-rekub's throne; the details of its decoration are precisely similar to those of an Assyrian bronze throne in the British Museum. The full moon and cresent are not of the familiar form, but are mountaed on a standard with tassels.} \par 

The detailed history and artistic development of Sam'al and Ya'di convey a very vivid impression of the social and material effects upon the native population of Syria, which followed the westward advance of Assyria in the eighth century. We realize not only the readiness of one party in the state to defeat its rival wih the help of Assyrian support, but also the manner in which the life and activities of the nation as a whole were unavoidably affected by their action. Other Hittite-Aramaean and Phoenician monuments, as yet undocumented with literary records, exhibit a strange but not unpleasing mixture of foreign \textit{motifs,} sych as we see on the stele from Amrith\footnotemark in the inland district of Arvad. But perhaps the most remarkable example of Syrian art we possess is the king's gate recently discovered at Carchemish.\footnotemark The presence of the hierglyphic inscriptions points to the survival of Hittite tradition, but the figures represented in the reliefs are of Aramaean, not Hittite, type. Here the king is seen leading his eldest son by the hand in some stately ceremonial, and ranged in registers behind them are the younger members of the royal family, whose ages are indicated by their occupations.\footnotemark The employment of basalt in place of limestone does not disguise the sculptor's debt to Assyria. But the design is entirely his own, and the combined dignity and homeliness of the composition are refreshingly superior to the arrogant spirit and hard execution what mar so much Assyrian work. This example is particularly instructive, as it shows how a borrowed art may be developed in skilled hands and made to serve a purpose in complete harmony with its new environment. \par 

\footnotetext{\textit{Collection de Clercq,} t. II, pl. xxxvi. The stele is sculptured in relief with the figure of a North Syrian god. Here the winged disk is Egyptian, as well as the god's helmet with uraeus, and his loin-cloth; his attitude and his supporting lion are Hittite; and the lozenge-mountains, on which the lion stands, and the technique of the carbing are Assyrian. But in spite of its composite character the design is quite successful and not in the least incongruous.}

\footnotetext{Hogarth, \textit{Carchemish,} Pt. I (1914), pl. B. 7 f.}

\footnotetext{Two of the older boys play at knuckle-bones, others whip spinning-tops, and a little naked girl runs behind supporting herself with a stick, on the head of which is carved a bird. The procession is brought up by the queen-mother, who carries the youngest baby and leads a pet lamb.} 

Such monuments surely illustrate the adaptability of the Semitic craftsman among men of Phoenician and Aramaean strain. Excavation in Palestine has failed to furnish examples of Hebrew work. But Hebrew tradition itself justifies us in regarding this \textit{trait} as of more general application, or at any rate as not repugnant to Hebrew thought, when it relates that Solomon employed Tyrian craftsneb for work upon the Temple and its furniture; for Pheonician art was essentially Egyptian in its origin and general character. Even Eshmun-'zar's desire for burial in an Egyptian sarcophagus may be paralleled in Hebrew tradition of a much earlier period, when, in the last verse of Genesis,\footnotemark it was recorded that Joseph died, ``and they ebalmed him, and he was put in a coffin in Egypt.'' Since it formed the subject of prophetic denunciation, I refrain for the moment from citing the notorious adoption of Assyrian customs at certain periods othe later Judaean monarchy. The two records I have referred to will suffice, for we have in them cherished traditions, of which the Hebrews themselves were proud, concerning the most famous example of Hebrew religious architectre and the burial of one of the patriarchs of the race. A similar readiness to make use of the best available resources, even of foreign origin, may on analogy be regarded as at least possible in the composition of Hebrew literature. \par 

\footnotetext{Gen. l. 26, assigned by critics to E.}

We shall see that the problems we have to face concern the possible influence of Babylon, rather than of Egypt, upon Hebrew tradition. And one last example, drawn from the later period, will serve to demonstrate how Babylonian influence penetrated the ancient world and has even left some trace upon modern civilization. It is a fact, though one perhaps not generally realized, that the twelve divisions on the dials of our clocks and watches have a Babylonian, and ultimately a Sumerian, ancestry. For why is it we divide the day into twenty-four hours? We have a decimal system of reckoning, we count by tens; why then should we divide the day and night into twelve hours each, instead of ten or some multiple of ten? The reason is that the Babylonians divided the day into twelve double-hours; and the Greeks took over their ancient system of time-division along with their knowledge of astronomy and passed it on to us. So if we ourselves, after more than two thousand years, are making use of an old custom from Babylon, it would not be surprising if the Hebrews, a contemporary race, should have fallen under her influence even befroe they were carried away as captives and settled forcibly upon her river-banks. \par 

We may passon, then, to the site from which our new material has been obtained--the ancient city of Nippur, in central Babylonia. Though the place has been deserted for at least nine hundred years, its ancient name still lingers on in local tradition, and to this day \textit{Niffer \emph{or} Nuffar} is the name the Arabs give the mounds which cover its extensive ruins. No modern town or village has been built upon them or in their immediate neighbourhood. The nearest considerable town is D\^iw\^an\^iyah, on the left bank of the Hillah branch of the Euphrates, twenty miles to the south-west; but some four miles to the south of the ruins is the village of S\^uq el-'Afej, on the eastern edge of teh 'Afej marshes, which begin to the south of Nippur and stretch away westward. Protected by its swamps, the region contains a few primitive settlements of the wild 'Afej tribesmen, each a group of reed-nuts clustering around the ud fort of its ruling sheikh. Their chief enemies are the Shamm\^ar, who dispute with them possession of the pastures. In summer the marshes near the mounds are merely pools of water connected by channels through the reed-beds, but in sprin the flood-water converts them into a vast lagoon, and all that meets the eye are a few small hamlets built on rising knolls above the water-level. Thus Nippur may be almost isolated during the floods, but the mounds are protected from the waters' encroachment by an outer ring of former habitation which has lightly raised the level of the encircling area. The ruins of the city stand from thirty to seventy feet above the plain, and in the north-eastern corner there rose, before the excavations, a conical mound, known by the Arabs as \textit{Bint el-Em\^ir} or ``The Princess.'' This prominent landmark represents the temple-tower of Enlil's famous sancturary, and even after excavation it is still the first object that the approaching traveller sees on the horizon. When he has climbed its summit he enjoys an uninterrupted view over desert and swamp. \par 

The cause of Nippur's present desolation is to be traced to te change in the bed of Euphrates, which now lies far to the west. But in antiquity the stream flowed through the centre of the city, along the dry bed of the Shatt en-N\^il, which divides the mounds into an eastern and a western group. The latter covers the remains of the city proper and was occupied in part by the great business-shouses and bazaars. Here more than thriy thousand contracts and accounts, dating from the fourth millennium to the fifth century B.C., were found in houses along the former river-bank. In the eastern half of the city was Enlil's great temple Ekur, with its temple-tower Imkharsag rising in successive stages beside it. The huge temple-enclosure contained not only the sacrificial shrines, ut also the priests' apartments, store-chambers, and temple-magazines. Outside its enclosing wall, to the south-west, a large triangular mound, christened ``Tablet Hill'' by the excavators, yielded a further supply of records. In addition to business-documents of the First Dynasty of Babylon and of the later Assyrian, Neo-Babylonian, and Persian periods. between two and three thousand literary texts and fragments were discovered here, many of them dating from the Sumerian period. And it is possible that some of the early literary texts that have been published were obtained in other parts of the city. \par 

No less than twenty-one different strata, representing separated periods of occupation, have been noted by the American excavators at various levels within the Nippur mounds,\footnotemark the earliest descending to virgin soil some twenty feet below the present level of the surrounding plain. The remote date of Nippur's foundation as a city and cult-centre is attested by the fact that the pavement laid by Nar\^am-Sin in the south-eartern temple-court lies theirty feet above virgin soil, while only thriy-six feet of superimposed \textit{d\'ebris} represent the succeeding millennia of occupation down to Sassanian and early Arab times. In the period of the Hebrew captivity the city still raned as a great commercial market and as one of the most sacred repositories of Babylonian religious tradition. We know that not far off was Telabib, the seat of on of the colonies of Jewish exiles, for that lay ``by the river of Chebar,''\footnotemark which we may identify with the Kabaru Canal in Nippur's immediate neighbourhood. It was ``among the captives by the river Chebar' that Ezekiel lived and prophesied, and it was on Chebar's banks that he saw his first vision of the Cherubim.\footnotemark He and other of the Jewish exiles may perhaps have mingled with the motley crowd that once thronged the streets of Nippur, and they may often have gazed on the huge temple-tower which rose above the city's flat roofs. We know that the later population of Nippur itself included a considerable Jewish element, for the upper strata of the mounds have yielded numoerous clay bowls with Hebrew, Mandaean, and Syriac magical inscriptions;\footnotemark and not the least interesting of the objects recovered was the wooden box of a Jewish scribe, containing his pen and ink-vesse and a little scrap of crumbling parchment inscribed with a few Hebrew characters.\footnotemark \par 

\footnotetext{See Hilprecht, \textit{Explorations in Bible Lands,} pp. 289 ff., 540 ff.; and Fisher, \textit{Excavations at Nippur,} Pt. I (1905), Pt. II (1906).}

\footnotetext{Ezek. iii. 15.}

\footnotetext{Ezek. i. 1, 3; iii. 23; and cf. x. 15, 20, 22, and xliii. 3.}

\footnotetext{See J. A. Montgomery, \textit{Aramaic Incantation Texts from Nippur,} 1913}

\footnotetext{Hilprecht, \textit{Explorations,} p. 555 f.}

Of the many thousands of inscribed clay tablets which were found in the course of the expeditions, some were kept at Constantinople, while others were presented by the Sultan Abdul Hamid to the excavators, who had them conveyed to America. Since that time a large number have been published. The  work was necessarily slow, for many of the texts were found to bein an extremely bad state of preservation. So it happened that a great number of the boxes containing tablets remained until recently still packed up in the store-rooms of the Pennsylvania Museum. But under the present energetic Director of the Museum, Dr. G. B. Gordon, the process of arranging and publishing the mass of literary material has been ``speeded up.'' A staff of skilled workmen has been employred on the laborious task of cleaning the broken tablets and fitting the fragments together. At the same time the gelp of several Assyriologists was welcomed in the further task of running over and sorting the collections as they were prepared for study. Professor Clay, Professor Barton, Dr. Langdon, Dr. Edward Chiera, and Dr. Arno Poebel have all participated in the work. But the lion's share has fallen to the last-named scholar, who was given leave of absence by John Hopkins University in order to take up a temporary appointment at the Pennsylvania Musuem. The result of his labours was published by the Museum at the end of 1914.\footnotemark The texts thus made available for study are of very varied interest. A great body of them are grammatical and represent compilations made by Semitic scribes of the period of Hammurabi's dynasty for their study of the old Sumerian tongue. Containing, as most of them do, Semitic renderings of the Sumerian words and expressions collected, they are as great a help to us in our study of Sumerian language as they were to their compilers; in particular they have thrown much new light on the paradigms of the demonstrative and personal pronouns and on Sumeria verbal forms. But literary texts are aldo included in the recent publications. \par 

\footnotetext{Peobel, \textit{Historical Texts \emph{and} Historical and Grammatical Texts} (Univ. of Penns. Mus. Publ., Bab. Sect., Vol. IV, No. 1, and Vol. V), Philadelphia, 1914.}

When the Pennsylvania Museum sent out its first expedition, lively hopes were entertained that the site selected would yield material of interest from the biblical standpoin. The city of Nippur, as we have seen, was one of the most sacred and most ancient religious centres in the country, and Enlil, its city-god, was the head of the Babylonian pantheon. On such a site it seemed likely that we might find versions of the Babylonian leggends which were current at the dawn of history before the city of Babylonia and its Semitic inhabitants came upon the scene. This expectation has proved to be not unfounded, for the literary texts include the Sumerian Deluge Version and Creation myth to which I referred at the beginning of the lecture. Other texts of almost equal interest consist of early though fragmentary lists of historical and semi-mythical rules. They prove that Berossus and the later BAbylonians depended on material of quite early origin in compiling their dynasties of semi-mythical kings. In them we obtain a glimpse of ages more remote than any on which excavation in Babylonia has yet thrown light, and for the first time we have recovered genuine native tradition of early date with regard to the cradle of Babylonian culture. BEfore we approach the Sumerian legends themselves, it will be as well togday to trace back in this tradition the gradual merging of history into legend and myth, comparing at the same time the ancient Egyptian's picture of his own remote past. We will also ascertain whether any new ligth is thrown by our inquiry upon Hebrew traditions concerning the earliest history of the human race and the origins of civilization. \par 

In the study of both Egyptian and Babylonian chronology there has been a tendency of late years to reduce the very early dates that were formerly in fashion. But in Egypt, while the dynasties of Manetho have been telescoped in places, excavation has thrown light on predynasitc periods, and we can now trace the history of culture in the Nile Valley back, through an unbroken sequence, to its neolightic stage. Quite recently, too, as I mentioned just now, a fresh literary record of these early predynstic periods has been recovered, on a fragment of the famous Palermo Stele, our most valuable monument for early Egyptian history and chronology. Egypt presents a striking consrast to Babylonia in the comparatively small number of written records which have survived for the reconstruction of her history. We might well spare much of her religious literature, enshrined in endless temple-inscriptions and papyri, if we could but exchange it for some of the royal annals of Egyptian Pharaohs. That historical recods of this character were compiled by the Egyptian scribes, and that they were detailed ad precise in their information as those we have recovered from Assyrian sources, is clear from the few extracts from the annals of Thothmes III's wars which are engraved on the walls of the temple at Karnak.\footnotemark As in Babylonia and Assyria, such records must have formed the foundation on which summaries of chronicles of pasat Egyptian history were based. In the Palermo Stele it is recordnized that we possess a primitive chronicle of this character. \par 

\footnotetext{See Breasted, \textit{Ancient Records}, I, p. 4, II, pp. 163 ff.}

Drawn up as early as the Vth Dynasty, its historical summary proves that from the beginning of the dynastic age onward a yearly record was kapt of the most important achievements of the reigning Pharaoh. In this fragmentary byt invaluable epitome, recording in outline much of the history of the Old Kingdom,\footnotemark some interesting parallels have long been noted with Babylonian usage. The early system of time-reckoning, for example, was the same in both countries, each year being given an official title from the chief event that occurred in it. And altough in Babylonia we are still without material for tracing the procedss by which this cumbrous method gave place to that of reckoning by regnal years, the Palermo Stele demonstrates the way in which the latter system was evolved in Egypt. For the events from which the year was named came gradually to be confined to the fiscal ``numberings'' of casttle and land. And when these, which at first had taken place comparatively long intervals, had become annual events, the numbered sequence of their occurence corresponded precisely to the years of the king's reign. On the stele, during the dynastic period, each regnal year is allotted its own space or rectangle,\footnotemark arranged in horizontal sequence below the name aand titles of the ruling king. \par 

\footnotetext{Op. cit., I, pp. 57 ff.}

\footnotetext{The spaces are not strictly rectangles, as each is divided vertically from the next by the Egyptian hieroglyph for ``year.''}

The text, which is engraved on both sides of a great block of black basalt, teakes its name from the fact that the fragment hithero known has been preserved since 1877 at the Museum of Palermo. Five other fragments of the text have now been published, of which one undoubtedly belongs to the same monument as the Palermo fragment, while the others may represent parts of one or more duplicate copies of that famous text. One of the four Cairo fragments\footnotemark was found by a digger for \textit{sebakh} at Mitrah\^ineh (Memphis); the other three, which were purchased from a dealer, are said to have come from Minieh, while the fifth fragment, at University College, is also said to have come from Upper Egypt,\footnotemark though it was purchased by Professor Petrie while at Memphis. These reports suggest that a number of duplicate copies were engraved and set up in different Egyptian towns, and it is possible that the whole of the text may eventually be recovered. The choice of basalt for the records was obviously dictated by a desire for their preservation, but it has had the contrary effect; for the blocks of this hard and precious stone have been cut up and reused in later times. The largest and most interesting of the new fragments has evidently been employed as a door-sill, with the result that its surface is much rubbed has an important bearing on our knowledge of Egyptian predynastic history and on the traditions of that remote period which have come down to us from the history of Manetho. \par 

\footnotetext{See Gautier, \textit{Le Mus\'ee \'Egyptien}, III (1915), pp. 29 ff., pl. xxiv ff., and Foucart, \textit{Bulletin de l'Institut Fran\c{c}ais d'Arch\'eologie Orientale}, XII, ii (1916), pp. 161 ff.; and cf. Gardiner, \textit{Journ. of Egypt. Arch.,} III, pp. 143 ff., and Petrie, \textit{Ancient Egypt}, 1916, Pt. III, pp. 114 ff.}

\footnotetext{Cf. Petrie, op. cit., pp. 115, 120.}

From the fragment of the stele preserved at Palermo we already knew that its record went back beyond the Ist Dynasty into predynastic times. For part of the top band of tne inscription, which is there preserved, contains nine names borne by kings of Lower Egypt or the Delta, which, it had been conjectured, must follow the gods of Manetho and preced the ``Worshippers of Horus,'' the immediate predecessors of the Egyptian dynasties.\footnotemark But of contemporary rulers of Upper Egypt we had hitherto no knowledge, since the supposed royal names discovered at Abydos and assigned to the time of the ``Worshippers of Horus'' are probably not royal names at all.\footnotemark With the possible exception of two very archaic slate palettes, the first historical memorials recovered from the south do not date from an earlier period than the beginning of the Ist Dynasty. The largest of the Cairo fragments now hellps us to fill in this gap in our knowledge. \par 

\footnotetext{See Breasted, \textit{Anc. Rec.}, I, pp. 52, 57.}

\footnotetext{Cf. Hall, \textit{Ancient History of the Near East,} p. 99 f.}

On the top of the new fragment\footnotemark we meet the same band of rectangles as at Palermo,\footnotemark but here their upper portions are broken away, and there only remains at the base of each of them the outlined figure of a royal personage, seated in the same attitude as those on the Palermo stone. The remarkable fact about these figures is that, with the apparent exception of the third figure from the right,\footnotemark each wears, not the Crown of the North, as at Palermo, but the Crown of the South. We have then to do with kings of Upper Egypt, not the Delta, and it is no longer possible to suppose that the predynastic rulers of the Palermo Stele were confined to those of Lower Egypt, as reflecting northern tradition. Rulers of both halves of the country are represented, and Monsieur Gautier has shown,\footnotemark from data on the reverse of the inscription, that the kings of the Delta were arranged on the original stone before the rulers of the south who are outlined upon our new fragment. Moreover, we have now recovered definit proof that this band of the inscription is concerned with predynastic Egyptian princes; for the cartouche of the king, whose years are enumerated in the second band immediately below the kings of the south, reads Athet, a name we may with certainty identify with Athothes, the second successor of Menes, founder of the Ist Dynasty, which is already given under the form Ateth in the Abydos List of Kings.\footnotemark It is thus quite certain that the first band of the inscription relates to the earlier periods before the two halves of the country were brought together under a single ruler. \par 

\footnotetext{Cairo No. 1; see Gautier, \textit{Mus. \'Egypt.,} III, pl. xxiv f.}

\footnotetext{In this upper band the spaces are true rectangles,being separated by vertical lines, not by the hieroglyph for ``year'' as in the lower bands; and each rectangle is assigned to a separate king, and not, as in the other bands, to a year of a king's reign.}

\footnotetext{The difference in the crown worn by this figure is probably only apparent and not intentional; M. Foucart, after a careful examination of the fragment, concludes that it is due to subsequent damage or to an original defect in the stone; cf. \textit{Bulletin,} XII, ii, p. 162.}

\footnotetext{Op. cet., p. 32 f.}

\footnotetext{In Manetho's list he corresponds to \{Kenkenos\}, the second successor of Menes according to both Africanus and Eusebius, who assign the name Athothis to the second ruler of the dynasty only, the Teta of the Abydos List. The form Athothes is preserved by Eratosthenes for both of Menes' immediate successors.}

Though the tradition of these remote times is here recorded on a monument of the Vth Dynasty, there is no reason to doubt its general accuracy, or to suppose that we are dealing with purely mthological personages. It is perhaps possible, as Monsieur Foucart suggests, that missing portions of the text may have carried the record back through purely mythical periods to Ptah and the Creation. In that case we should have, as we shall see, a striking parallel to early Sumerian tradition. But in the first extant portions of the Palermo texte we are already in the realm of genuine tradition. The names preserved appear to be those of individuals, not of mythological creations, and we may assume that their owners really existed. For though the invention of writing had not at that time been achieved, its place was probably taken by oral tradition. We know that with certain tribes of Africa at the present day, who possess no knowledge of writing ,there are functionaries charged with the duty of preserving tribal traditions, who transmit orally to their successors a remembrance of past chiefs and some details of events that occurred centuries before.\footnotemark The predynastic Egyptians may well have adopted similar means for preserving a remembrance of their past history. \par 

\footnotetext{M. Foucart illustrates this point by citing the case of the Bushongos, who have in this way preserved a list of no less than a hundred and twenty-one of their past kings; op. cit., p. 182, and cf. Tordey and Joyce, ``Les Bushongos,'' in \textit{Annales du Mus\'ee dy Congo Belge,} s\'er. III, t. II, fasc. i (Brussels, 1911).}

Moreover, the new text furnishes fresh proof of the general accuracy of Manetho, even when dealing with traditions of this prehistoric age. On the stele there is no definite indication that these two sets of predynastic kings were contemporaneous rulers of Lower and Upper Egypt respectively; and since elsewhere the lists assign a single soverign to each epoch, it has been suggested that we should regard them as successive representatives of the legitimate kingdom.\footnotemark Now Manetho, after his dynasties of gods and demi-gods, states that thirty Memphite kings reigned fro 1,790 years, and were followed by ten Thinite kings whose reigns covered a period of 350 years. Neglecting the figures as obviously erroneous, we may well admit that the Greek historian here alludes to our two pre-Menite dynasties. But the fasct that he should regard them as ruling consecutively does not preclude the other alternative. The modern convention of arranging lines of contemporaneous rulers in parallel columns had not been evolved in antiquity, and without some such method of distinction contemporaneous rulers, when enumerated in a list, can only be registered consecutively. It would be natural to assume that, before the unification of Egypt by the founder of the Ist Dynasty, the rulers of North and South were independent princes, possessing no traditions of a united throne on which any claim to hegemony could be based. On the assumption that this was so, their arrangement in a consecutive series would not have deceived their immediate successors. But it owuld undoubtedly tend in course of time to obliterate the tradition of their true order, which even at the period of the Vth Dynasty may have been completely forgotten. Manetho would thus have introduced no strange or novel confusion; and this explanation would of course apply to other sections of his system where the dynasties he enumerates appear to by too many for their period. But his reproduction of two lines of predynastic rulers, supported as it now is by the early evidence of the Palermo text, only serves to increase our confidence in the general accuracy of his sources, while at the same time it illustrates very effectively the way in which possible insccuracies, deduced from independent data, may have arisen in quite early times. \par 

\footnotetext{Fourcart, loc. cit.}


\end{document}