\documentclass[12pt,oneside]{book}

\usepackage[margin=1in]{geometry}
\usepackage{gensymb,mathtools,amssymb,graphicx,setspace,titlesec}

\setlength{\parindent}{0pt}
\setlength{\parskip}{6pt}

\titleformat{\chapter}[hang]{\bfseries}{}{0pt}{\Huge\uppercase}[]

\begin{document}
    
\frontmatter

Project Gutenberg's Legends of Babylon and Egypt, by Leonard W. King \par 

This eBook is for the use of anyone anywhere at no cost and with almost no restrictions whatsoever. You may copy it, give it away or re-use it under the terms of the Project Gutenberg License included with this eBook or online at www.gutenberg.org \par 

Title: Legends of Babylon and Egypt \par 
\hspace{1cm}In Relation to Hebrew Tradition \par 

Author: Leonard W. King \par 

Release Date: March 28, 2006 [EBook \#2030] \\
Last Updated: February 4, 2013 \par 

Language: English \par 

*** START OF THIS PROJECT GUTENBERG EBOOK LEGENDS OF BABYLON AND EGYPT ***

\vfill 

Produced by John Bickers, Dagny and David Widger \par 

\begin{titlepage}
    \centering 
    \vspace*{2cm}
    {\Huge \textbf{Legends of Babylon and Egypt}} \par 
    \vspace{3cm}
    {\Large \textbf{In Relation to Hebrew Tradition}}
\end{titlepage}

\begin{center}
    {\Large By Leonard W. King, M.A., Litt.D., F.S.A.} \\
    \vspace{1cm}
    {\large Assistant Keeper of Egyptian and Assyrian Antiquities in the British Museum} \\
    \vspace{1cm}
    {\large Professor in the University of London King's College} \\
    \vspace{1cm}
    {\large First Published 1918 by Humphrey Milford, Oxford University Press.} \\
    \vspace{1cm}
    {\large \uppercase{The British Academy \\ The Schweich Lectures 1916}}
    \vfill 
    Preparer's Note \\
\end{center}
This text was prepared from a 1920 edition of the book, hence the references to dates after 1916 in some places. Greek text has been transliterated within brackets ``{}'' using an Oxford English Dictionary alphabet table. Diacritical marks have been lost. \par 

\tableofcontents

\chapter{Preface}
In these lectures an attempt is made, not so much to restate familiar facts, as to accommodate them to new and supplementary evidence which has been published in America since the outbreak of the war. But even without the excuse of recent discovery, no apology would be needed for any comparison or contrast of Hebrew tradition with the mythological and legendary beliefs of Babylon and Egypt. Hebrew achievements in the sphere of religion and ethics are only thrown into stronger relief when studied against their contemporary background. \par 

The bulk of our new material is furnished by some early texts, written towards the close of the third millennium B.C. They incorporate traditions which extend in unbroken outline from their own period into the remote ages of the past, and claim to trace the history of man back to his creation. They represent the early national traditions of the Sumerian people, who preceded the Semites as the ruling race in Babylonia; and incidentally they necessitate a revision of current views with regard to the cradle of Babylonian civilization. The most remarkable of the new documents is one which relates in poetical narrative an account of the Creation, of Antediluvian history, and of the Deluge. It this exhibits a close resemblance in structure to the corresponding Hebrew traditions, a resemblance that is not shared by the Semitic-Babylonian Versions at present known. But in matter the Sumerian tradition is more primitive than any of the Semitic versions. In spite of the fact that the text appears to have been reached us in a magical setting, and to some extent in epitomized form, this early document enables us to tap the stream of tradition at a point far above any at which approach has hitherto been possible. \par 

Though the resemblance of early Sumerian tradition to that of the Hebrews is striking, it furnishes a still closer parallel to the summaries preserved from the history of Berossus. The huge figures incorporated in the latter's chronological scheme are no longer to be treated as a product of Neo-Babylonian speculation; they reappear in their original surroundings in another of these early documents, the Sumerian Dynastic List. The sources of Berossus had inevitably been semitized by Babylon; but two of his three Antediluvian cities find their place among the five of primitive Sumerian belief, and two of his ten Antediluvian kings rejoin their Sumerian prototypes. Moreover, the recorded ages of Sumerian and Hebrew patriarchs are strangely alike. It may be added that in Egypt a new fragment of the Palermo Stele has enabled us to verify, by a very similar comparison, the accuracy of Manetho's sources for his prehistoric period, while at the same time it demonstrates the way in which possible inaccuracies in his system, deduced from independent evidence, may have arisen in remote antiquity. It is clear that both Hebrew and Hellenistic traditions were modelled on very early lines. \par 

Thus our new material enables us to check the age, and in some measure the accuracy, of the traditions concerning the dawn of history which the Greeks reproduced from native sources, both in Babylonia and Egypt, after the conquests of Alexander had brought the Near East within the range of their intimate acquaintance. The third body of tradition, that of the Hebrews, thought unbacked by the prestige of secular achievement, has, through incorporation in the canons of two great religious systems, acquired an authority which the others have not enjoyed. In re-examining the sources of all three accounts, so far as they are affected by the new discoveries, it will be of interest to observe how the same problems were solved in antiquity by very different races, living under widely divergent conditions, but within easy reach of one another. Their periods of contact, ascertained in history or suggested by geographical considerations, will prompt the further question to what extent each body of belief was evolved in independence of the others. The close correspondence that has long been recognized and is now confirmed between the Hebrew and the Semitic-Babylonian systems, as compared with that of Egypt, naturally falls within the scope of our enquiry. \par 

Excavation has provided an extraordinarily full archaeological commentary to the legends of Egypt and Babylon; and when I received the invitation to deliver the Schweich Lectures for 1916, I was reminded of the terms of the Bequest and was asked to emphasize the archaeological side of the subject. Such material illustration was also calculated to bring out, in a more vivid manner than was possible with purely literary evidence, the contrasts and parallels presented by Hebrew tradition. Thanks to a special grant for photographs from the British Academy, I was enabled to illustrate by means of lantern slides many of the problems discussed in the lectures; and it was originally intended that the photographs than shown should appear as plates in this volume. But in view of the continued and increasing shortage of paper, it was afterwards felt to be only right that all illustrations should be omitted. This very necessary decision has involved a recasting of certain sections of the lectures as delivered, which in its turn has rendered possible a fuller treatment of the new literary evidence. To the consequent shifting of interest is also due a transposition of names in the title. On their literary side, and in virtue of the intimacy of their relation to Hebrew tradition, the legends of Babylon must be given precedence over those of Egypt. \par 

For the delay in the appearance of the volume I must plead the pressure of other work, on subjects far removed from archaeological study and affording little time and few facilities for a continuance of archaeological and textual research. It is hoped that hte insertion of references throughout, and the more detailed discussion of problems suggested by our new literary material, may incline the reader to add his indulgence to that already extended to me by the British Academy. \par 

L. W. King. 

\mainmatter

\chapter[Lecture I - Origins of Civilization]{Lecture I--Egypt, Babylon,\\ And Palestine, and Some \\Traditional Origins of Civilization}
At the present moment most of us have little time or thought to spare for subjects not connected directly or indirectly with the war. We have put aside our own interests and studies; and after the war we shall all have a certain amount of leeway to make up in acquainting ourselves with what has been going on in countries not yet involved in the great struggle. Meanwhile the most we can do is to glance for a moment at any discovery of exceptional interest that may come to light. \par 

The main object of these lectures will be to examine certain Hebrew traditions in the light of new evidence which has been published in America since the outbreak of the war. The evidence is furnished by some literary texts, inscribed on tablets from Nippur, one of the oldest and most sacred cities of Babylonia. They are written in Sumerian, the language spoken by the non-Semitic people whom the Semitic Babylonians conquered and displaced; and they include a very primitive version of the Deluge story and Creation myth, and some texts which throw new light on the age of Babylonian civilization and on the area within which it had its rise. In them we have recovered some of the material from which Berossus derived his dynasty of Antediluvian kings, and we are thus enabled to test the accuracy of the Greek tradition by that of the Sumerains themselves. So far then as Babylonia is concerned, these documents will necessitate a re-examination of more than one problem. \par 

The myths and legends of ancient Egypt are also to some extent involved. The trend of much recent anthropological research has been in the direction of seeking a single place of origin for similar beliefs and practices, at least among races which were bound to one another by political or commercial ties. And we shall have occasion to test, by means of our new data, a recent theory of Egyptian influence. The Nile Valley was, of course, one of the great centres from which civilization radiated throughout the ancient East; and, even when direct contact is unproved, Egyptian literature may furnish instructive parallels and contrasts in any study of Western Asiatic mythology. Moreover, by a strange coincidence, there has also been published in Egypt since the beginning of the way a record referring to the reigns of predynastic rulers in the Nile Valley. This, like some of the Nippur texts, take us back to that dim period before the dawn of actual history, and, though the information it affords is not detailed like theirs, it provides fresh confirmation of the general accuracy of Manetho's sources, and suggests some interesting points for comparison. \par 

But the people with whose traditions we are ultimately concerned are the Hebrews. In the first series of Schweich Lectures, delivered in the year 1908, the late Canon Driver showed how the literature of Assyria and Babylon had thrown light upon Hebrew traditions concerning the origin and early history of the world. The majority of the cuneiform documents, on which he based his comparison, date from a period no earlier than the seventh century B.C., and yet it was clear that the texts themselves, in some form or other, must have descended from a remote antiquity. He concluded his brief reference to the Creation and Deluge Tablets with these words: ``The Babylonian narratives are both polytheistic, while the corresponding biblical narratives (Gen. i and vi-xi) are made the vehicle of a pure and exalted monotheism; but in spite of this fundamental difference, and also variations in detail, the resemblances are such as to leave no doubt that the Hebrew cosmogony and the Hebrew story of the Deluge are both derived ultimately from the same original as the Babylonian narratives, only transformed by the magic tough of Israel's religion, and infused by it with a new spirit.''\footnote{Driver, \textit{Modern Research as illustrating the Bible} (The Schweich Lectures, 1908), p. 23.} Among the recently published documents from Nippur we have at last recovered one at least of those primitive originals from which the Babylonian accounts were derived, while others prove the existence of variant stories of the world's origin and early history which have not survived in the later cuneiform texts. In some of these early Sumerian records we may trace a faint but remarkable parallel with the BHebrew traditions of man's history between his Creation and the Flood. It will be our task, then, to examine the relations which the Hebrew narratives bear both to the early Sumerian and to the later Babylonian Versions, and to ascertain how far the new discoveries support or modify current views with regard to the contents of those early chapters of Genesis. \par 

I need not remind you that Genesis is the book of Hebrew origins, and that its contents mark it off to some extent from the other books of the Hebrew Bible. The object of the Pentateuch and the Book of Joshua is to describe in their origin the fundamental institutions of the national faith and to trace from the earliest times the course of events which led to the Hebrew settlement in Palestine. Of this national history the Book of Genesis forms the introductory section. Four centuries of complete silence lie between its close and the beginning of Exodus, where we enter on the history of a nation as contrasted with that of a family.\footnote{Cf., E.g., Skinner, \textit{A Critical and Exegetical Commentary on Genesis} (1912), p. ii f.; Driver, \textit{The Book of Genesis}, 10th ed. (1916), pp. 1 ff.; Ryle, \textit{The Book of Genesis} (1914), pp. x ff.} While Exodus and the succeeding books contain national traditions, Genesis is largely made up of individual biography. Chapters xii-l are concerned with the immediate ancestors of the Hebrew race, beginning with Abram's migration into Canaan and closing with Joseph's death in Egypt. But the aim of the book is not confined to recounting the ancestry of Israel. It seeks also to show her relation to other peoples in the world, and probing still deeper into the past it describes how the earth itself was prepared for man's habitation. Thus the patriarchal biographies are preceded, in chapters i-xi, by an account of the original of the world, the beginnings of civilization, and the distribution of the various races of mankind. It is, of course, with certain parts of this first group of chapters that such striking parallels have long been recognized in the cuneiform texts. \par 

In approaching this particular body of Hebrew traditions, the necessity for some caution will be apparent. It is not as though we were dealing with the reported beliefs of a Malayan or Central Australian tribe. In such a case there would be no difficulty in applying a purely objective criticism, without regard to ulterior consequences. But here our own feelings are involved, having their roots deep in early associations. The ground too is well trodden; and, had there been no new material to discuss, I think I should have preferred a less contentious theme. The new material is my justification for the choice of subject, and also the fact that, whatever views we may hold, it will be necessary for us to assimilate it to them. I shall have no hesitation in giving you my own reading of the evidence; but at the same time it will be possible to indicate solutions which will probably appeal to those who view the subject from more conservative standpoints. That side of the discussion may well be postponed until after the examination of the new evidence in detail. And first of all it will be advisable to clear up some general aspects of the problem, and to define the limits within which our criticism may be applied. \par 

It must be admitted that both Egypt and Babylon bear a bad name in Hebrew tradition. Both are synonymous with captivity, the symbols of suffering endured at the beginning and at the close of the national life. And during the struggle against Assyrian aggression, the disappointment at the failure of expected help is reflected in prophecies of the period. These great crises in Hebrew history have tended to obscure in the national memory the part which both Babylon and Egypt may have played in moulding the civilization of the smaller nations with whom they cam in contact. To such influence the races of Syria were, by geographical position, peculiarly subject. The country has often been compared to a bridge between the two great continents of Asia and Africa, flanked by the sea on one side and the desert on the other, a narrow causeway of highland and coastal plain connecting the valleys of the Nile and the Euphrates.\footnote{See G. A. Smith, \textit{Historical Geography of the Holy Land}, pp. 5 ff., 45 ff., and Myres, \textit{Dawn of History}, pp. 137 ff.; and cf. Hogarth, \textit{The Nearer East}, pp. 65 ff., and Reclus, \textit{Nouvelle G\'eographie universelle}, t. IX, pp. 685 ff.} For, except on the frontier of Egypt, desert and sea do not meet. Farther north the Arabian plateau is separated from the Mediterranean by a double mountain chain, which runs south from the Taurus at varying elevations, and encloses in its lower course the remarkable depression of the Jordan Valley, the Dead Sea, and the 'Arabah. The Judaean hills and the mountains of Moab are merely the southward prolongation of the Lebanon and Anti-Lebanon, and their neighbourhood to the sea endows this narrow tract of habitable country with its moisture and fertility. It thus formed the natural channel of intercourse between the two earliest centres of civilization, and was later the battle-ground of their opposing empires. \par 

The great trunk-roads of through communication run north and south, across the eastern plateaus of the Haur\^an and Moab, and along the coastal plains. The old highway from Egypt, which left the Delta at Pelusium, at first follows the coast, then trends eastward across the plain of Esdraelon, which breaks the coastal range, and passing under Hermon runes northward through Damascus and reaches the Euphrates at its most westerly point. Other through tracks in Palestine ran then as they do to-day, by Beesheba and Hebron, or along the 'Arabah and west of the Dead Sea, or through Edom and east of Jordan by the present Jahh route to Damascus. But the great highway from Egypt, the most westerly of the trunk-roads through Palestine, was that mainly followed, with some variant sections, by both caravans and armies, and was known by the Hebrews in its southern course as the ``Way of the Philistines'' and farther north as the ``Way of the East.'' \par 

The plain of Esraelon, where the road first trends eastward, has been the battle-ground for most invaders of Palestine from the north, and though Egyptian armies often fought in the southern coastal plain, they too have battled there when the held the southern country. Megiddo, which commands the main pass into the plain through the low Samaritan hills to the southeast of Carmel, was the site of Thothmes III's famous battle against a Syrian confederation, and it inspired the writer of the Apocalypse with his vision of an Armageddon of the future. But invading armies always followed the beaten track of caravans, and movements represented by the great campaigns were reflected in the daily passage of international commerce. \par 

With so much through traffic continually passing within her borders, it may be matter for surprise that far more striking evidence of its cultural effect should not have been revealed by archaeological research in Palestine. Here again the explanation is mainly of a geographical character. For though the plains and plateaus could be crossed by the trunk-roads, the rest of the country is so broken up by mountain and valley that it presented few facilities either to foreign penetration or to external control. The physical barriers to local intercourse, reinforced by striking differences in soil, altitude, and climate, while they precluded Syria herself from attaining national unity, always tended to protect her separate provinces, or ``kingdoms,'' from the full effects of foreign aggression. One city-state could be traversed, devastated, or annexed , without in the least degree affecting neighbouring areas. It is true that the population of Syria has always been predominantly Semitic, for she was on the fringe of the great breeding-ground of the Semitic race and her landward boundary was open to the Arabian nomad. Indeed, in the whole course of her history the only race that bade fair at one time to oust the Semite in Syria was the Greek. But the Greeks remained within the cities which they founded or rebuilt, and, as Robertson Smite pointed out, the death-rate in Eastern cities habitually exceeds the birth-rate; the urban population must be reinforced from the country if it is to be maintained, so that the type of population is ultimately determined by the blood of the peasantry.\footnote{See Robertson Smith, \textit{Religion of the Semites}, p. 12 f.; and cf. Smith, \textit{Hist. Geogr.}, p. 10 f.} Hence after the Arab conquest the Greek elements in Syria and Palestine tended rapidly to disappear. The Moslem invasion was only the last of a series of similar great inroads, which have followed on another since the dawn of history, and during all that time absorption was continually taking place from desert tribes that ranged the Syrian border. As we have seen, the country of his adoption was such as to encourage the Semitic nomad's particularism, which was inherent in his tribal organization. Thus the predominance of a single racial element in the population of Palestine and Syria did little to break down or overstep the natural barriers and lines of cleavage. \par 

\end{document}