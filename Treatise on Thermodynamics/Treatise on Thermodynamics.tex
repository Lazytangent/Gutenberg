\documentclass[oneside,12pt]{book}

\usepackage{mathtools,amsthm,amssymb,setspace,titlesec,graphicx,gensymb}
\usepackage[margin=1in]{geometry}

\setlength{\parskip}{6pt}
%\setlength{\parindent}{0pt}
\onehalfspace
\newcommand{\iit}[1]{\textit{#1}}
%\titleformat{\chapter}[hang]{\Huge\bfseries}{\thechapter.\hspace{20pt}}{0pt}{\Huge\bfseries}
\titleformat{\chapter}[hang]{\Huge\bfseries}{Chapter \thechapter.\hspace{10pt}}{0pt}{\Huge\bfseries}

\usepackage[symbol]{footmisc}
\usepackage[version=4]{mhchem}
\renewcommand{\thefootnote}{\fnsymbol{footnote}}

\counterwithout{equation}{chapter}

\begin{document}

\frontmatter

\vspace*{1cm}
\vfill
\begin{center}
    Transcriber's Note
\end{center}
\noindent The camera-quality files for this public-domain ebook may be downloaded \iit{gratis} at
\begin{center}
    www.gutenberg.org/ebooks/50880.
\end{center} \par

\noindent This ebook was produced using scanned images and OCR text generously provided by the University of California, Berkeley, through the Internet Archive. \par

\noindent Minor typographical corrections and presentational changes have been made without comment. \par

\noindent This PDF file is optimized for screen viewing, but may be recompiled for printing. Please consult the preamble of the \LaTeX\ source file for instructions and other particulars. \par

\pagebreak

\begin{titlepage}
    \centering
    \uppercase{
    {\LARGE treatise\par}
    {\large on\par}
    {\Huge thermodynamics\par}
    {\normalsize by\par}
    {\Large Dr. Max Planck \par}
    {\small professor of theoretical physics in the University of Berlin \par}
    \vspace{1cm}
    {\large \textit{Translated with the Author's Sanction} \par}
    {\normalsize By \par}
    {\large Alexander Ogg, M.A., B.Sc., Ph.D. \par }
    {\tiny Late 1851 exhibition scholar and university assistant, Aberdeen University\par}
    {\tiny Assistant Master, Royal Naval engineering college, Devonport \par}
    \vspace{2cm}
    {\LARGE Longmans, Green, and Co. \par }
    {\large 39 Paternoster Row, London \par}
    {\normalsize New York and Bombay \par}
    {\Large 1903 \par }
    {\normalsize \textit{All rights reserved}}}
\end{titlepage}

\chapter{Translator's Notice}
The modern developments of Thermodynamics, and the applications to physical and chemical problems, have become so important, that I have ventured to translate Professor Planck's book, which presents the whole subject from a uniform point of view. \par

A few noes have been added to the present English edition by Professor Planck. He has not found it necessary to change the original text in any way. \par

To bring the notation into conformity with the usual English notation, several symbols have been changed. This has been done with the author's sanction. Here I have followed J. J. van Laar and taken $\Psi$ to signify what he calls the \iit{Planck'sches Potential}, \iit{i.e.} the thermodynamic potential of Gibbs and Duhem divided by $-\theta$. \par

Professor Planck's recent paper, ``\"Uber die Grundlage der L\"osungstheorie'' (Ann. d. Phys. \textbf{10}, p. 436, 1903), ought to be read in connection with his thermodynamical theory of solution. \par

I am indebted to Herren Veit \& Co., Leipzig, for kindly supplying the blocks of the five figures in the text. \par

\begin{flushright}
    A. O.
\end{flushright}
\textsc{Devonport,} \par
\textit{June,} 1903

\chapter{Preface}
The oft-repeated requests either to publish my collected papers on Thermodynamics, or to work them up into a comprehensive treatise, first suggested the writing of this book. Although the first plan would have been the simpler, especially as I found no occasion to make any important changes in the line of thought of my original papers, yet I decided to rewrite the whole subject-matter, with the intention of giving at greater length, and with more detail, certain general considerations and demonstrations too concisely expressed in these papers. My chief reason, however, was that an opportunity was thus offered of presenting the entire field of Thermodynamics from a uniform point of view. This, to be sure, deprives the work of the character of an original contribution to science, and stamps it rather as an introductory text-book on Thermodynamics for students who have taken elementary courses in Physics and Chemistry, and are familiar with the elements of the Differential and Integral Calculus. \par

Still, I do not think that this book will entirely supersede my former publications on the same subject. Apart from the fact that these contain, in a sense, a more original presentation, there may be found in them a number of details expended at greater length than seemed advisable in the more comprehensive treatment here required. To enable the reader to revert in particular cases to the original form for comparison, a list of my publications on Thermodynamics has been appended, with a reference in each case to the section of the book which deals with the same point. \par

The numerical values in the examples, which have been worked, as applications of the theory, have, almost all of them, been taken from the original papers; only a few, that have been determined by frequent measurement, have been taken from the tables in Kohlrausch's ``Leitfaden der praktischen Physik.'' It should be emphasized, however, that the numbers used, notwithstanding the care taken, have not undergone the same amount of critical sifting as the more general propositions and deductions. \par

Three distinct methods of investigations may be clearly recognized in the previous development of Thermodynamics. The first penetrates deepest into the nature of the processes considered, and, were it possible to carry it out exactly, would be designated as the most perfect. Heat, according to it, is due to the definite motions of the chemical molecules and atoms considered as distinct masses, which in the case of gases possess comparatively simple properties, but in the case of solids and liquids can be only very roughly sketched. This kinetic theory, founded by Joule, Waterston, Kr\"onig and Clausius, has been greatly extended mainly by Maxwell and Boltzmann. Obstacles, at present unsurmountable, however, seem to stand in the way of its further progress. These are due not only to the highly complicated mathematical treatment, but principally to essential difficulties, not to be discusses here, in the mechanical interpretation of the fundamental principles of Thermodynamics. \par

Such difficulties are avoided by the second method, developed by Helmholtz. It confines itself to the most important hypothesis of the mechanical theory of heat, that heat is due to motion, but refuses on principle to specialize as to the character of this motion. This is a safer point of view than the first, and philosophically quite as satisfactory as the mechanical interpretation of nature in general, but it does not as yet offer a foundation of sufficient breadth upon which to build a detailed theory. Starting from this point of view, all that can be obtained is the verification of some general laws which have already been deduced in other ways direct from experience. \par

A third treatment of Thermodynamics has hitherto proved the most fruitful. This method is distinct from the other two, in that it does not advance the mechanical theory of heat, but, keeping aloof from definite assumptions as to its nature, starts direct from a few very general empirical facts, mainly the two fundamental principles of Thermodynamics. From these, by pure logical reasoning, a large number of new physical and chemical laws are deduced, which are capable of extensive application, and have hitherto stood the test without exception. \par

This last, more inductive, treatment, which is used exclusively in this book, corresponds best to the present state of the science. It cannot be considered as final, however, but may have in time to yield to a mechanical, or perhaps an electromagnetic theory. Although it may be of advantage for a time to consider the activities of nature--Heat, Motion, Electricity, etc.--as different in quality, and to suppress the question as to their common nature, still our aspiration after a uniform theory of nature, on a mechanical basis or otherwise, which has derived such powerful encouragement from the discovery of the principle of the conservation of energy, can never by permanently repressed. Even at the present day, a recession from the assumption that all physical phenomena are of a common nature would be tantamount to renouncing the comprehension of a number of recognized laws of interaction between different spheres of natural phenomena. Of course, even then, the result s we have deduced from the two laws of Thermodynamics would not be invalidated, but these two laws would not be introduced as independent, but would be deduced from other more general propositions. At present, however, no probable limit can be set to the time which it will take to reach this goal. \par
\begin{flushright}
    \uppercase{The Author.}
\end{flushright}
\textsc{Berlin,} \par
\iit{April,} 1897

\tableofcontents

\mainmatter
\part{Fundamental Facts and Definitions}
\chapter{Temperature}

\textbf{\S 1.} The conception of ``heat'' arises from that particular sensation of warmth or coldness which is immediately experienced on touching a body. This direct sensation, however, furnishes no quantitative scientific measure of a body's state with regard to heat; it yields only qualitative results, which vary according to external circumstances. For quantitative purposes we utilize the change of volume which takes place in all bodies when heated under constant pressure, for this admits of exact measurement. Heating produces in most substances an increase of volume, and thus we can tell whether a body gets hotter or colder, not merely by the sense of touch, but also by a purely mechanical observation affording a much greater degree of accuracy. We can also tell accurately when a body assumes a former state of heat. \par

\textbf{\S 2.} If two bodies, one of which feels warmer than the other, be brought together (for example, a piece of heated metal and cold water), it is invariably found that the hotter body is cooled, and the colder one is heated up to a certain point, and then all change ceases. The two bodies are then said to be in \iit{thermal equilibrium}. Experience shows that such a state of equilibrium finally sets in, not only when two, but also when any number of differently heated bodies are brought into mutual contact. From this follows the important proposition: \iit{If a body, A, be in thermal equilibrium with two other bodies, B and C, then B and C are in thermal equilibrium with one another.}\footnote[1]{As is well known, there exists no corresponding proposition for electrical equilibrium. For if we join together the substances \ce{Cu|CuSO4aq\text{.}|ZnSO4aq\text{.}|Zn} to form a conducting ring, no electrical equilibrium is possible.} For, if we bring $A,\ B,\ and C$ together so that each touches the other two, then, according to our supposition, there will be equilibrium at the points of contact $AB$ and $AC$, and, therefore, also at the contact $BC$. If it were not so, no general thermal equilibrium would be possible, which is contrary to experience. \par

\textbf{\S 3.} These facts enable us to compare the degree of heat of two bodies, $B$ and $C$, without bringing them into contact with one another; namely, by bringing each body into contact with an arbitrarily selected standard body, $A$ (for example, a mass of mercury enclosed in a vessel terminating in a fine capillary tube). By observing the volume of $A$ in each case, it is possible to tell whether $B$ and $C$ are in thermal equilibrium or not. If they are not in thermal equilibrium, we can tell which of the two is the hotter. The degree of heat of $A$, or of any body in thermal equilibrium with $A$, can thus be very simply defined by the volume of $A$, or, as is usual, by the difference between the volume of $A$ and its volume when in thermal equilibrium with melting ice under atmospheric pressure. This volumetric difference, which, by an appropriate choice of unit, is made to read 100 when $A$ is in contact with steam under atmospheric pressure, is called the \textit{temperature} in degrees Centigrade with regard to $A$ as thermometric substance. Two bodies of equal temperature are, therefore, in thermal equilibrium, and \textit{vice vers\^a}. \par

\textbf{\S 4.} The temperature readings of no two thermometric substances agree, in general, except at 0\textdegree and 100\textdegree. The definition of temperature is therefore somewhat arbitrary. This we may remedy to a certain extent by taking gases, in particular those hard to condense, such as hydrogen, oxygen, nitrogen, and carbon monoxide, as thermometric substances. They agree almost completely within a considerable range of temperature, and their readings are sufficiently in accordance for most purposes. Besides, the coefficient of expansion of these different gases is the same, inasmuch as equal volumes of them expand under constant pressure by the same amount--about $\frac{1}{273}$ of their volume--when heated from 0\textdegree C. to 1\textdegree C. Since, also, the influence of the external pressure on the volume of these gases can be represented by a very simple law, we are led to the conclusion that these regularities are based on a remarkable simplicity in their constitution, and that, therefore, it is reasonable to define the common temperature given by them simply as temperature. We must consequently reduce the readings of other thermometers to those of the gas thermometer, and preferably to those of the hydrogen thermometer. \par

\textbf{\S 5.} The definition of temperature remains arbitrary in cases where the requirements of accuracy cannot be satisfied by the agreement between the readings of the different gas thermometers, for there is no sufficient reason for the preference of any one of these gases. A definition of temperature completely independent of the properties of any individual substance, and applicable to all stages of heat and cold, becomes first possible on the basis of the \textit{second law of thermodynamics} (\S 160, etc.). In the mean time, only such temperature will be considered as are defined with sufficient accuracy by the gas thermometer. \par

\textbf{\S 6.} In the following we shall deal chiefly with homogeneous, isotropic bodies of any form, possessing throughout their substance the same temperature and density, and subject to a uniform pressure acting everywhere perpendicular to the surface. They, therefore, also exert the same pressure outwards. Surface phenomena are thereby disregarded. The condition of such a body is determined by its chemical nature; its mass, $M$; its volume, $V$; and its temperature, $t$. On these must depend, in a definite manner, all other properties of the particular state of the body, especially the pressure, which is uniform throughout, internally and externally, The pressure, $p$, is measured by the force acting on the unit of area--in the C.G.S. system, in dynes per square centimeter, a \textit{dyne} being the force which imparts to a mass of one gramme in one second a velocity of one centimeter per second. \par

\textbf{\S 7.} As the pressure is generally given in atmospheres, the value of an atmosphere in absolute C.G.S. units is here calculated. The pressure of an atmosphere is the weight of a column of mercury at 0\textdegree C., 76 cm. high, and 1 sq. cm. in cross-section, when placed in mean geographical latitude. This latter condition must be added, because of the weight, \textit{i.e.} the force of the earth's attraction, varies with the locality. The volume of the column of mercury is $76 cm.^3$; and sine the density of mercury at 0\textdegree C. is 13.596, the mass is $76 \times 13.596$. Multiplying the mass by the acceleration of gravity in mean latitude, we find the pressure of one atmosphere in absolute units to be
$$76 \times 13.596 \times 981 = 1,013,650\quad\frac{\text{dynes}}{\text{cm.}^2}\quad \text{or} \quad \frac{\text{gr.}}{\text{cm. -sec.}^2}.$$
This, then, is the factor for converting atmospheres into absolute units. If, as was formerly the custom in mechanics, we use as the unit of force the weight of a gramme in mean geographical latitude instead of the dyne, the pressure of an atmosphere would be $76 \times 13.596 = 1033.3$ grms. per square centimeter. \par

\textbf{\S 8.} Since the pressure in a given substance is evidently controlled by its internal physical condition only, and not by its form or mass, it follows that $p$ depends only on the temperature and tha ratio of the mass $M$ to the volume $V$ (\textit{i.e.} the density), or on the reciprocal of the density, the volume of unit mass---which is called the specific volume of the substance. For every substance, then, there exists a characteristic relation---
$$p=f(v,t),$$
which is called the \textit{characteristic equation} of the substance. For gases, the function $f$ is invariably positive; for liquids and solids, however, it may have also negative values under certain circumstances. \par

\setcounter{equation}{0}
\textbf{\S 9. Perfect Gases.}---The characteristic equation assumes its simplest form for the substances which we used in \S 4 for the definition of temperature. If the temperature be kept constant, then, according to the Boyle-Mariotte law, the product of the pressure and the specific volume remains constant for gases---
\begin{equation}
    pb=T,
    \label{eq1}
\end{equation}
where $T$, for a given gas, depends only on the temperature. \par

But if the pressure be kept constant, then, according to \S 3, the temperature is proportional to the difference between the present volume $v$ and the volume $v_0$ at 0\textdegree; \textit{i.e.}---
\begin{equation}
    t=(v-v_0)P,
    \label{eq2}
\end{equation}
where $P$ depends only on the pressure $p$. Equation \eqref{eq1} for $v_0$ becomes
\begin{equation}
    pv_0=T_0,
    \label{eq3}
\end{equation}
where $T_0$ is the value of the function $T$, when $t=0$\textdegree C. \par

Finally, as has already been mentioned in \S 4, the expansion of all permanent gases on heating from 0\textdegree C. to 1\textdegree C. is the same fraction $\alpha$ (about $\frac{1}{273}$) of their volume at 0\textdegree (Gay-Lussac's law). Putting $t=1$, we have $v-v_0=\alpha v_0$, and equation \eqref{eq2} becomes
\begin{equation}
    1=\alpha v_0P.
    \label{eq4}
\end{equation}
By eliminating $P$, $v_0$, and $v$ from \eqref{eq1}, \eqref{eq2}, \eqref{eq3}, \eqref{eq4}, we obtain the temperature function of the gas---
$$T=T_0(1+\alpha t),$$
which is seen to be linear function of $t$. The characteristic equation \eqref{eq9.1} becomes
$$p=\frac{T_0}{v}(1+\alpha t).$$ \par

\textbf{\S 10.} The form of this equation is considerably simplified by shifting the zero of temperature, arbitrarily fixed in \S 3, by $\frac{1}{\alpha}$ degrees, and calling the melting point of ice, not 0\textdegree C., but $\frac{1\degree}{\alpha}$ C. (\textit{i.e.} about 273\textdegree C.). For, putting $t+\frac{1}{\alpha}=\theta$ (absolute temperature), and the constant $\alpha T_0=C$, the characteristic equation becomes
\begin{equation}
    p=\frac{C}{v}\theta=\frac{CM}{V}\theta.
    \label{eq5}
\end{equation}
This introduction of \textit{absolute} temperature is evidently tantamount to measuring temperature no longer, as in \S 3, by a change of volume, but by the volume itself. \par

\textbf{\S 11.} The constant $C$, which is characteristic for the perfect gas under consideration, can be calculated, if the specific volume $v$ be known for any pair of values of $\theta$ and $p$ (\textit{e.g.} 0\degree and 1 atmosphere). For different gases, taken at the same temperature and pressure, the constants $C$ evidently vary directly as the specific volumes, or inversely as the densities $\frac{1}{v}$. It may be affirmed, then, that, taken at the same temperature and pressure, the densities of all perfect gases bear a constant ratio to one another. A gas is, therefore, often characterized by the constant ratio which its density bears to that of a normal gas at hte same temperature and pressure (\textit{specific density} relative to air or hydrogen). At 0\degree C. ($\theta=273\degree$) and 1 atmosphere pressure, the densities of the following gases are:
\begin{center}
\begin{tabular}{ll}
    Hydrogen & 0.00008988 $\frac{gr.}{cm.^3}$ \\
    Oxygen & 0.0014291 \\
    Nitrogen & 0.0012507 \\
    Atmospheric nitrogen & 0.0012571 \\
    Air & 0.0012930
\end{tabular}
\end{center}
whence the corresponding values of $C$ in absolute units can be readily calculated. \par

All questions with regard to the behaviour of a substance when subjected to changes of temperature, volume, and pressure are completely answered by the characteristic equation of the substance. \par

\textbf{\S 12. Behaviour under Constant Pressure} (Isopiestic or Isobaric Changes).---\textit{Coefficient of expansion} is the name given to the ratio of the increase of volume for a rise of tempearture of 1\degree C. to the volume at 0\degree C. This increase for a perfect gas is, according to \eqref{eq5}, $\frac{CM}{p}$. The same equation \eqref{eq5} gives the volume of the gas at 0\degree C. as $\frac{CM}{p}\times 273$, hence the ratio of the two quantities, or the coefficient of expansion, is $\frac{1}{273}=\alpha$. \par

\textbf{\S 13. Behaviour at Constant Volume} (Isochoric or Isopycnic Changes).---The \textit{pressure coefficient} is the ratio of the increase of pressure for a rise of temperature of 1\degree to the pressure at 0\degree C. For a perfect gas, this increase, according to equation \eqref{eq10.5}, is $\frac{CM}{V}$. The pressure at 0\degree C. is $\frac{CM}{V}\times 273$, whence the required ratio, \textit{i.e.} the pressure coefficient, is $\frac{1}{273}$, therefore equal to the coefficient of expansion $\alpha$. \par

\textbf{\S 14. Behaviour at Constant Temperature} (Isothermal Changes).---\textit{Coefficient of elasticity} is the ratio of an infinitely small increase of pressure to the resulting contraction of unit volume of the substance. In a perfect gas, according to equation \eqref{eq5}, the contraction of volume $V$, in consequence of an increase of pressure $dp$, is
\begin{equation*}
    dV=\frac{CM\theta}{p^2}dp=\frac{V}{p}dp.
\end{equation*}
The contraction of unit volume is therefore
\begin{equation*}
    -\frac{dV}{V}=\frac{dp}{p},
\end{equation*}
and the coefficient of elasticity of the gas is
\begin{equation*}
    \frac{dp}{\frac{dp}{p}}=p,
\end{equation*}
that is, equal to the pressure. \par

The reciprocal of the coefficient of elasticity, \textit{i.e.} the ratio of an infinitely small contraction of unit volume to the corresponding increase of pressure, is called the \textit{coefficient of compressibility}. \par

\textbf{\S 15.} The three coefficients which characterize the behaviour of a substance subject to isopiestic, isopycnic, and isothermal changes are not independent of one another, but are in every case connected by a definite relation. The general characteristic equation, on being differentiated, gives
$$dp=\left(\frac{\partial p}{\partial \theta}\right)_\nu d\theta+\left(\frac{\partial p}{\partial\nu}\right)_\theta d\nu,$$
where the suffixes indicate the variables to be kept constant while performing the differentiation. By putting $dp=0$ we impose the condition of an isopiestic processes:---
\begin{equation}
    \left(\frac{\partial\nu}{\partial\theta}\right)_p=-\frac{\left(\frac{\partial p}{\partial\theta}\right)_\nu}{\left(\frac{\partial p}{\partial\nu}\right)_\theta}.
    \label{eq6}
\end{equation} \par

For every state of a substance, one of the three coefficients, viz. of expansion, of pressure, or of compressibility, may therefore be calculated from the other two. \par

Take, for example, mercury at 0\degree C. and under atmospheric pressure. Its coefficient of expansion is (\S 12)
\begin{equation*}
    \left(\frac{\partial\nu}{\partial\theta}\right)_p\cdot\frac{1}{\nu_0}=0/00018,
\end{equation*}
its coefficient of compressibility in atmospheres (\S 14) is
$$-\left(\frac{\partial\nu}{\partial p}\right)_\theta\cdot\frac{1}{\nu_0}=0.000003,$$
therefore its pressure coefficient in atmospheres (\S 13) is
$$\left(\frac{\partial p}{\partial\theta}\right)_\nu=-\left(\frac{\partial p}{\partial\nu}\right)_\theta\cdot\left(\frac{\partial \nu}{\partial\theta}\right)_p=-\frac{\left(\frac{\partial\nu}{\partial\theta}\right)_p}{\left(\frac{\partial\nu}{\partial p}\right)_\theta}=\frac{0.00018}{0.000003}=60.$$
This means that an increase of pressure of 60 atmospheres is required to keep the volume of mercury constant when heated from 0\degree C. to 1\degree C. \par

\textbf{\S 16. Mixture of Perfect Gases.}---If any quantities of the \textit{same} gas at the same temperatures and pressures be at first separated by partitions, and then allowed to come suddenly in contact with another by the removal of these partitions, it is evident that the volume of the entire system will remain the same and be equal to the sum-total of the partial volumes. Starting with quantities of \textit{different} gases, experience still shows that, when pressure and temperature are maintained uniform and constant, the total volume continues equal to the sum of the volumes of the constituents, notwithstanding the slow process of intermingling---diffusion---which takes place in this case. Diffusion goes on until the mixture has become at every point of precisely the same composition, \textit{i.e.} physically homogeneous. \par

\textbf{\S 17.} Two views regarding the constitution of mixtures thus formed present themselves. Either we might assume that the individual gases, while mixing, split into a large number of small portions, all retaining their original volumes and pressures, and that these small portions of the different gases, without penetrating each other, distribute themselves evenly throughout the entire space. In the end each gas would still retain its original volume (partial volume), and all the gases would have the same common pressure. Or, we might suppose---and this view will be shown below (\S 32) to be the correct one---that the individual gases change and interpenetrate in every infinitesimal portion of the volume, and that after diffusion each individual gas, in so far as one may speak of such, fills the total volume, and is consequently under a lower pressure than before, diffusion. This so-called partial pressure of a constituent of a gas mixture can easily be calculated. \par

\textbf{\S 18.} Denoting the quantities referring to the individual gases by suffixes---$\theta$ and $p$ requiring no special disgnation, as they are supposed to be the same for all the gases,---the characteristic equation \eqref{eq5} gives for each gas before diffusion
$$p=\frac{C_1M_1\theta}{V_1};\quad p=\frac{C_2M_2\theta}{V_2};\quad \dots .$$
The total volume,
$$V=V_1+V_2+\dots ,$$
remains constant during diffusion. After diffusion we ascribe to each gas the total volume, and hence the partial pressures become
\begin{equation}
    p_1=\frac{C_1M_1\theta}{V}=\frac{V_1}{V}p;\quad p_2=\frac{C_2M_2\theta}{V}=\frac{V_2}{V}p;\quad \dots ,
    \label{eq7}
\end{equation}
and by addition
\begin{equation}
    p_1+p_2+\dots=\frac{V_1+V_2+\dots}{V}p=p.
    \label{eq8}
\end{equation}
This is Dalton's law, that in a homogeneous mixture of gases the pressure is equal to the sum of hte partial pressures of the gases. It is also evident that
\begin{equation}
    p_1\, :\, p_2\,:\, \dots=V_1 : V_2 :\dots=C_1M_1:C_2M_2:\dots,
    \label{eq9}
\end{equation}
\textit{i.e.} the partial pressures are proportional to the volumes of the gases before diffusion, or to the partial volumes which the gases would have according to the first view of diffusion given above. \par

\textbf{\S 19.} The characteristic equation of the mixture, according to \eqref{eq7} and \eqref{eq8}, is 
\begin{align*}
    p&=(C_1M_1+C_2M_2+\dots)\frac{\theta}{V} \\
    &=\left(\frac{C_1M_1+C_2M_2+\dots}{M}\right)\frac{M}{C}\theta 
    \tag{10}
    \label{eq10}
\end{align*}
\setcounter{equation}{10}
which corresponds to the characteristic equation of a perfect gas with the following characteristic constant:---
\begin{equation}
    C=\frac{C_1M_1+C_2M_2+\dots}{M_1+M_2+\dots}. 
    \label{eq11}
\end{equation}
Hence the question as to whether a perfect gas is a chemically simple one, or a mixture of chemically different gases, cannot in any case be settled by the investigation of the characteristic equation. \par 

\textbf{\S 20.} The composition of a gas mixture is defined, either by the ratios of the masses, $M_1$, $M_2$, \dots or by the ratios of the partial pressures $p_1,\, p_2$ \dots or the partial volumes $V_1$, $V_2$, \dots of the individual gases. Accordingly we speak of per cent. by weight or by volume. Let us take for example atmospheric air, which is a mixture of oxygen (1) and ``atmospheric'' nitrogen (2). \par 

The ratio of the densities of oxygen, ``atmospheric'' nitrogen and air is, according to \S 11,
$$0.0014291:0.0012571:0.0012930=\frac{1}{C_1}:\frac{1}{C_2}:\frac{1}{C_3}.$$
Taking into consideration the relation \eqref{eq11}---
$$C=\frac{C_1M_1+C_2M_2}{M_1+M_2},$$
we find the ratio $M_1:M_2=0.2998$, \textit{i.e.} 23.1 per cent. by weight of oxygen and 76.9 per cent. of nitrogen. Furthermore, 
$$C_1M_1:C_2M_2=p_1:p_2=V_1:V_2=0.2637$$
\textit{i.e.} 20.9 per cent. by volume of oxygen and 79.1 per cent. of nitrogen. \par 

\textbf{\S 21. Characteristic Equation of Other Substances.}---The characteristic equation of perfect gases, even in the case of the substances hitherto discussed, is only an approximation, though a close one, to the actual facts. A still further deviation from the behaviour of perfect gases is shown by the other gaseous bodies, especially by those easily condensed, which for this reason were formerly classed as \textit{vapours}. For these a modification in the form of the characteristic equation is necessary. It is worthy of notice, however, that the more rarefied the state in which we observe these gases, the less does their behaviour deviate from that of perfect gases, so that al gaseous substances, when sufficiently rarefied, may be said in general to act like perfect gases. The general characteristic equation of gases and vapours, for very large values of $v$, will pass over, therefore, into the special form for perfect gases. \par 

\textbf{\S 22.} We may obtain by various graphical methods an idea of the character and magnitude of the deviations from the ideal gaseous state. An isothermic curve may, \textit{e.g.}, be drawn, taking $\nu$ and $p$ for some given temperature as the abscissa and ordinate, respectively, of a point in a plane. The entire system of isotherms gives us a complete representation of the characteristic equation. The more the behaviour of the vapour in question approaches that of a perfect gas, the closer do the isotherms approach those of equilateral hyperbolae having the rectangular co-ordinate axes for asymptotes, for $p\nu=$ const. is the equation of an isotherm of a perfect gas. The deviation from the hyperbolic form yields at the same time a measure of the departure from the ideal state. \par 

\textbf{\S 23.}
\end{document}